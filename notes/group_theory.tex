\chapter{Gruppentheorie}



\section{Nebenklassen und Satz von Lagrange}

Es sei $G$ eine Gruppe und $H \subgroup G$ eine Untergruppe.

\begin{definition}
  Für alle $a \in G$ ist $aH \defined \{ ah \suchthat h \in H \}$ die \emph{Linksnebenklasse von $a$ bezüglich $H$}, und $Ha \defined \{ ha \suchthat h \in H \}$ die \emph{Rechtsnebenklasse von $a$ bezüglich $H$}.
\end{definition}

Für alle $a \in G$ gilt dabei
\[
        aH = H
  \iff  a \in H
  \iff  Ha = H \,.
\]
Auf $G$ werden nun durch
\[
        a \sim_L b
  \iff  aH = bH
  \quad\text{und}\quad
        a \sim_R b
  \iff  Ha = Hb
\]
zwei Äquivalenzrelationen definiert.
Dabei gilt für alle $a, b \in G$, dass
\[
        a \sim_L b
  \iff  aH = bH
  \iff  b^{-1} a H = H
  \iff  b^{-1} a \in H
  \iff  a^{-1} b \in H \,,
\]
sowie analog
\[
        a \sim_R b
  \iff  Ha = Hb
  \iff  H = H b a^{-1}
  \iff  b a^{-1} \in H
  \iff  a b^{-1} \in H \,.
\]
Dabei ist $a^{-1} b$ das eindeutige Gruppenelement mit $a \cdot a^{-1} b = b$.
Es folgt somit, dass
\begin{align*}
      [a]_{\sim_L}
   =  \{ b \in G \suchthat a \sim_L b \}
  &=  \{ b \in G \suchthat a^{-1} b \in H \}  \\
  &=  \{ b \in G \suchthat \exists h \in H : ah = b \}
   =  \{ ah \suchthat h \in H \}
   =  aH \,,
\end{align*}
sowie analog, dass $[a]_{\sim_R} = Ha$.
Die Äquivalenzklassen von $\sim_L$ und $\sim_R$ sind also genau die Linksnebenklassen Rechtsnebenklassen bezüglich $H$, also verschobene Versionen von $H$ selbst.

Dabei ist für jedes $a \in H$ die Abbildung
\[
          H
  \to     aH \,,
  \quad   h
  \mapsto ah
\]
bijektiv, weshalb $\card{aH} = \card{H}$ gilt, sowie analog auch $\card{Ha} = \card{H}$.

Es ist nun $G$ die disjunkte Vereinigug Äquivalenzklassen von $\sim_L$, d.h.\ $G$ zerfällt in disjunkte verschobene Versionen von $H$.
Ist $G$ endlich, so folgt somit, dass $\card{G}$ ein Vielfaches von $\card{H}$ ist.

\begin{definition}
  Die \emph{Ordnung} einer Gruppe $G$ ist $\ord{G} \defined \card{G}$.
\end{definition}

\begin{corollary}[Satz von Lagrange]
  Ist $G$ endlich, so gilt $\ord{H} \divides \ord{G}$.
\end{corollary}

\begin{definition}
  Es ist $G/H \defined G/{\sim_L} = \{aH \suchthat a \in H\}$ die Menge der Linksnebenklassen und $H \backslash G \defined G/{\sim_R} = \{Ha \suchthat a \in G\}$ die Menge der Rechtsnebenklassen.
\end{definition}


Für die Inversions-Abbildung $(-)^{-1} \colon G \to G$, $g \mapsto g^{-1}$ gilt
\[
    (aH)^{-1}
  = H^{-1} a^{-1}
  = H a^{-1} \,,
\]
weshalb $(-)^{-1}$ eine Bijektion $G/H \to H \backslash G$ induziert.
Es gibt deshalb gleich viele Links-\ und Rechtsnebenklassen, d.h.\ es gilt $\card{G/H} = \card{H \backslash G}$.

\begin{definition}
  Der \emph{Index} von $H$ in $G$ ist $\groupindex{G}{H} \defined \card{G/H} = \card{H \backslash G}$.
\end{definition}

\begin{corollary}
  Ist $G$ endlich, so gilt $\ord{G} = \ord{H} \groupindex{G}{H}$, sowie äquivalent $\groupindex{G}{H} = \ord{G}/{\ord{H}}$.
  Insbesondere ist auch $\groupindex{G}{H}$ ein Teiler von $\ord{G}$.
\end{corollary}





\section{Normalteiler und Quotientengruppen}



\subsection{Definition von Normalteilern}

Für eine Untergruppe $N \subgroup G$ sind die folgenden Bedingungen äquivalent:

\begin{enumerate}
  \item
    Für alle $a \in G$ gilt $aN = Na$.
  \item
    Für alle $a \in G$ gilt $aNa^{-1} = N$.
  \item
    Für alle $a \in G$ gilt $aNa^{-1} \subseteq N$.
\end{enumerate}

\begin{definition}
  Eine Untergruppe $N \subgroup G$, die eine \textup(und damit alle\textup) der obigen Bedingungen erfüllt, ist \emph{normal}, bzw.\ ein \emph{Normalteiler}.
  Dies wird mit $N \normalgroup G$ notiert.
\end{definition}

\begin{example}
  \begin{enumerate}
    \item
      Ist $G$ abelsch, so ist jede Untergruppe $N \subgroup G$ normal.
    \item
      Für $N \normalgroup G$ und $H \subgroup G$ mit $N \subgroup H$ gilt auch $N \normalgroup H$
    \item
      Für jeden Gruppenhomomorphismus $\phi \colon G \to H$ ist $\ker(\phi)$ ein Normalteiler.
  \end{enumerate}
\end{example}



\subsection{Konstruktion von Quotientengruppen}

Es gilt auch die Umkehrung des obigen Beispiels:
Ist $N \normalgroup G$ ein Normalteiler, so lässt sich auf $G/N$ durch
\[
            gN
  \cdot     hN
  \defined  (gh)N
\]
eine Gruppenstruktur definieren.

\begin{definition}
  Für $N \normalgroup G$ ist $G/N$ wie oben die \emph{Quotientengruppe} von $G$ nach $N$.
\end{definition}

\begin{remark}
  Es handelt sich bei dem obigen Vorgehen um eine von mehrenen möglichen Vorgehensweisen, Quotientengruppen zu konstruieren.
  Inbesondere lassen sich Quotientengruppen auch ohne Verwendung von Nebenklassen konstruieren.
  Entscheident ist für Quotientengruppen nur, dass die \emph{kanonische Projektion}
  \[
              p
    \colon    G
    \to       G/N
    \quad     g
    \mapsto   \class{g}
    \defined  gN
  \]
  ein Gruppenhomomorphismus mit $\ker p = N$ ist, welcher die folgende universelle Eigenschaft besitzt:
\end{remark}



\subsection{Universelle Eigenschaft der Quotientengruppe}

\begin{theorem}[Homomorphiesatz]
  Ist $f \colon G \to H$ ein Gruppenhomomorphismus mit $N \subseteq \ker(f)$, so induziert $f$ einen eindeutigen Gruppenhomomorphismus $\induced{f} \colon G/N \to H$ mit $f = \induced{f} \circ p$, d.h.\ so dass das folgende Diagramm kommutiert:
  \[
    \begin{tikzcd}
        G
        \arrow{rr}[above]{f}
        \arrow{dr}[below left]{p}
      & {}
      & H
      \\
        {}
      & G/N
        \arrow[dashed]{ru}[below right]{\induced{f}}
      & {}
    \end{tikzcd}
  \]
  In anderen Worten:
  Es ergibt sich eine Bijektion
  \begin{align*}
                            \{ \text{Gruppenhomo.\ $\induced{f} \colon G/N \to H$} \}
    &\xlongrightarrow{\sim} \{ \text{Gruppenhomo.\ $f \colon G \to H$ mit $N \subseteq \ker(f)$} \} \,,  \\
                            \induced{f}
    &\mapsto                \induced{f} \circ p \,.
  \end{align*}
\end{theorem}

\begin{corollary}[1.\ Isomorphiesatz]
  Jeder Gruppenhomomorphismus $f \colon G \to H$ induziert einen Isomorphismus
  \[
                            G/{\ker(f)}
    \xlongrightarrow{\sim}  \im(f)
    \quad                   \class{g}
    \mapsto                 f(g)
  \]
\end{corollary}

\begin{corollary}[2.\ Isomorphiesatz]
  Es seien $N, K \normalgroup G$ zwei Normalteiler mit $N \subgroup K$.
  Dann ist $K/N$ normal in $G/N$, und es gibt einen wohldefinierten Isomorphismus
  \[
                            (G/N)/(K/N)
    \xlongrightarrow{\sim}  G/K \,,
    \quad                   \class{ \class{g} }
    \mapsto                 \class{g} \,.
  \]
\end{corollary}

\begin{corollary}[3.\ Isomorphiesatz]
  Es sei $H \subgroup G$ eine Untergruppe und $N \normalgroup G$ eine normale Untergruppe.
  Dann ist $HN = \{hn \suchthat h \in H, n \in N\}$ eine Untergruppe von $G$, $H \cap N$ eine normale Untergruppe von $H$, und es gibt einen wohldefinierten Isomorphismus
  \[
                            H/(H \cap N)
    \xlongrightarrow{\sim}  HN/N \,,
    \quad                   \class{h}
    \mapsto                 \class{h} \,.
  \]
\end{corollary}



\subsection{Korrespondenz von (normalen) Untergruppen}

\begin{proposition}
  Es sei $N \normalgroup G$ eine normale Untergruppe und $p \colon G \to G/N$, $g \mapsto \class{g}$ die kanonische Projektion.
  \begin{enumerate}
    \item
      Es gibt eine 1:1-Korrespondenz von Untergruppen
      \begin{align*}
        \{ \text{Untergruppen $H \subgroup G$ mit $N \subgroup H$} \}
        &\xleftrightarrow{1:1}
        \{ \text{Untergruppen $H' \subgroup G/N$} \} \,,
        \\
        H
        &\longmapsto
        p(H)
        =
        H/N \,,
        \\
        p^{-1}(H')
        &\longmapsfrom
        H' \,.
      \end{align*}
    \item
      Es gibt eine eingeschränkte 1:1-Korrespondenz zwischen normalen Untergruppen
      \begin{align*}
        \{ \text{Normalteiler $K \normalgroup G$ mit $N \subgroup K$} \}
        &\xleftrightarrow{1:1}
        \{ \text{Normalteiler $K' \normalgroup G/N$} \} \,,
        \\
        K
        &\longmapsto
        p(K)
        =
        K/N \,,
        \\
        p^{-1}(K')
        &\longmapsfrom
        K' \,.
      \end{align*}
    \item
      Für jeden Normalteiler $K \normalgroup G$ mit $N \subgroup K$ gilt dabei für den zugehörigen Normalteiler $K' = p(K) = K/N$ nach dem 2.\ Isomorphiesatz, dass
      \[
              G/K
        \cong (G/N)/(N/K)
        \cong (G/N)/K' \,.
      \]
      Auf beiden Seiten der obigen 1:1-Korrespondenz erhält man somit \textup(bis auf Isomorphie\textup) die gleichen Quotientengruppen.
  \end{enumerate}
\end{proposition}





\section{Erzeugte Untergruppen}



\subsection{Definition erzeugter Untergruppen}

Es sei $G$ eine Gruppe und $S \subseteq G$ eine Teilmenge.
Für eine Untergruppe $H \subseteq G$ sind die folgenden Bedingungen äquivalent:

\begin{enumerate}
  \item
    Es ist $H$ die kleinste Untergruppe von $G$, die $S$ enthält, d.h.\ es gilt $S \subseteq H$, und für jede Untergruppe $K \subgroup G$ mit $S \subseteq K$ gilt $H \subgroup K$.
  \item
    Es gilt $H = \bigcap_{K \subgroup G, S \subseteq K} K$.
  \item
    Es gilt
    $
        H
      = \{
          s_1^{\varepsilon_1} \dotsm s_n^{\varepsilon_n}
        \suchthat
          n \in \Natural,
          s_i \in S,
          \varepsilon_i = \pm 1
        \}
    $.
\end{enumerate}

\begin{definition}
  Die Untergruppe $H$, die eine \textup(und damit alle\textup) der obigen Bedingungen erfüllt, ist die \emph{von $S$ erzeugte} Untergruppe, und wird mit $\generated{S}$ notiert.
  Es ist $S$ ein \emph{Erzeugendensystem} von $H$.
\end{definition}

\begin{definition}
  Eine Gruppe $G$ ist \emph{endlich erzeugt}, wenn es eine endliche Teilmenge $S \subseteq G$ mit $G = \generated{S}$ gibt, und \emph{zyklisch}, wenn es ein $g \in G$ mit $G = \generated{g}$ gibt.
\end{definition}



\subsection{Klassifikation zyklischer Gruppen}

\begin{example}
  \begin{enumerate}
    \item
      Die Gruppe $\Integer$ ist zyklisch mit Erzeuger $1$.
    \item
      Wird $G$ von $g \in G$ zyklisch erzeugt, so wird für jeden Normalteiler $N \normalgroup G$ die Quotientengruppe $G/N$ von $\class{1}$ zyklisch erzeugt.
    \item
      Insbesondere ist für alle $n \in \Integer$ ist die Quotientengruppe $\Integer/n \defined \Integer/n\Integer = \Integer/\generated{n}$ zyklisch mit Erzeuger $\class{1} \in \Integer/n$.
  \end{enumerate}
\end{example}

Es gilt auch die Umkehrung der obigen Beispiele, d.h.\ jede zyklische Gruppe ist zu genau einer der Gruppen $\Integer/n$ mit $n \geq 0$ isomorph.

\begin{lemma}
\label{lemma: subgroups of Z}
  Jede Untergruppe $H \subgroup G$ ist von der Form $H = \generated{n} = n\Integer$ für ein eindeutiges $n \in \Natural$.
  Für $H = \{0\}$ gilt $n = 0$, und sonst gilt $n = \min \{k > 0 \suchthat k \in H\}$.
\end{lemma}

Ist $G$ zyklisch mit Erzeuger $g \in G$, so ist die Abbildung
\[
          f
  \colon  \Integer
  \to     G \,,
  \quad   n
  \mapsto g^n
\]
ein surjektiver Gruppenhomomorphismus, und induziert somit einen Isomorphismus
\[
          \induced{f}
  \colon  \Integer/{\ker(f)}
  \to     G \,,
  \quad   \class{n}
  \mapsto g^n \,.
\]
Es gibt es eindeutiges $n \geq 0$ mit $\ker(f) = n\Integer$.
Gilt $n = 0$, so ist gilt $\Integer \cong G$, und $G$ ist unendlich.
Ansonsent gilt $n > 1$ und somit $G \cong \Integer/n$ mit $\ord{G} = n$.

\begin{corollary}[Klassifikation zyklischer Gruppen]
  Jede zyklische Gruppe $G$ ist zu genau einer der Gruppen $\Integer$, $\Integer/n$ mit $n \geq 1$ isomorph.
  Dabei gilt
  \[
          G
    \cong \begin{cases}
              \Integer
            & \text{falls $G$ unendlich ist} \,,
          \\
              \Integer/n
            & \text{falls $n = \ord{G}$ endlich ist} \,.
          \end{cases}
  \]
\end{corollary}

\begin{corollary}
  Untergruppen zyklischer Gruppen sind ebenfalls zyklisch.
\end{corollary}

\begin{definition}
  Für $g \in G$ ist $\ord{g} = \ord{\generated{g}}$ die \emph{Ordnung von $g$}.
\end{definition}

\begin{lemma}
  Es sei $g \in G$.
  \begin{enumerate}
    \item
      Ist $G$ eine endliche Gruppe, so gilt $\ord{g} \divides \ord{G}$.
    \item
      Hat $g$ endliche Ordnung, so gilt $\ord{g} = \min \{ n > 0 \suchthat g^n = 1 \}$.
  \end{enumerate}
\end{lemma}

\begin{example}[Gruppen von Ordnung $p$]
  Es sei $p$ prim und $G$ eine Gruppe von Ordnung $p$.
  Dann gibt es ein nicht-triviales Element $g \in G$.
  Dann ist $\ord{g} \neq 1$ ein Teiler von $\ord{G} = p$, und somit $\ord{g} = \ord{G}$, also $\generated{g} = G$.
  Somit ist $G$ zyklisch, also $G \cong \Integer/p$.
\end{example}

%TODO: Exercise: subgroups of S3





\section{Gruppenwirkungen}



\subsection{Grundlegende Definitionen}

\begin{definition}
  Es sei $G$ eine Gruppe und $X$ eine Menge.
  Eine \emph{\textup(Gruppen\textup)wirkung} von $G$ auf $X$ ist eine Abbildung $G \times X \to X$, $(g,x) \mapsto g.x$ mit $1.x = x$ und $g.(h.x) = (gh).x$ für alle $g, h \in G$, $x \in X$.
  
  Eine \emph{$G$-Menge} ist eine Menge $X$ zusammen mit einer Wirkung von $G$ auf $X$.
\end{definition}

Es sei $G$ eine Gruppe und $X$ eine Menge.
Wirkt $G$ auf $X$, so ist für jedes $g \in G$ die Abbildung $\lambda_g \colon X \to X$, $x \mapsto g.x$ eine Bijektion, und die Abbildung $G \to \symmetric{X}$, $g \mapsto \lambda_g$ ist ein Gruppenhomomorphismus.
Ist andererseits $\varphi \colon G \to \symmetric{X}$ ein Gruppenhomomorphismus, so definiert $g.x \defined \varphi(g)(x)$ eine Wirkung von $G$ auf $X$.
Diese beiden Konstruktionen sind invers zueinander, weshalb eine Wirkung von $G$ auf $X$ einem Gruppenhomomorphismus $G \to \symmetric{X}$ entspricht.

\begin{definition}
  Es sei $X$ eine $G$-Menge und $x \in X$.
  \begin{enumerate}
    \item
      Die \emph{$G$-Bahn} von $x$ ist $G.x \defined \{g.x \suchthat g \in G\}$.
    \item
      Es ist $X/G = \{G.x \suchthat x \in X\}$ die Menge der $G$-Bahnen.
    \item
      Der \emph{Stabilisator} von $x$ ist $G_x \defined \{g \in G \suchthat g.x = x\}$.
    \item
      Die \emph{Fixpunktmenge} von $X$ ist $X^G \defined \{x \in X \suchthat \forall g \in G: g.x = x\}$.
  \end{enumerate}
\end{definition}

\begin{lemma}
  Ist $X$ eine $G$-Menge, so gilt $G_x \subgroup G$ für jedes $x \in X$.
\end{lemma}

\begin{definition}
  Es sei $X$ eine $G$-Menge.
  \begin{enumerate}
    \item
      Die Wirkung von $G$ auf $X$ ist \emph{transitiv}, wenn es für alle $x, y \in X$ ein $g \in G$ mit $g.x = y$ gibt.
    \item
      Die Wirkung von $G$ auf $X$ ist \emph{treu}, wenn es für alle $g, h \in G$ aus
      \[
        \text{$g.x = h.x$ für alle $x \in X$}
      \]
      folgt, dass $g = h$ gilt.
  \end{enumerate}
\end{definition}

\begin{remark}
  Eine Wirkung von $G$ auf $X$ ist genau dann treu, wenn der zugehörige Gruppenhomomorphismus $G \to \symmetric{X}$ injektiv ist.
\end{remark}




\subsection{Die Bahnenformel}

Es sei $X$ eine $G$-Menge.
Durch
\[
        x \sim y
  \iff  \exists g \in G: g.x = y
\]
wird eine Äquivalenzrelation auf $X$ definiert.
Für alle $x \in X$ gilt dann $[x]_{\sim} = G.x$.
Je zwei Bahnen $G.x$ und $G.y$ sind also entweder gleich oder disjunkt, und $X$ ist die disjunkte Vereinigung der $G$-Bahnen.
Ist $(x_i)_{i \in I}$ ein Repräsentantensystem der $G$-Bahnen von $X$, so gilt deshalb
\begin{equation}
\label{equation: orbit formula alpha}
    \card{X}
  = \sum_{i \in I} \card{G.x_i} \,.
\end{equation}
Für jedes $x \in X$ ist dabei die Abbildung $G \to G.x$, $g \mapsto g.x$ surjektiv, und es gilt
\[
        g.x = h.x
  \iff  h^{-1}g.x = x
  \iff  h^{-1} g \in G_x
  \iff  g G_x = h G_x \,.
\]

\begin{lemma}
  Für jedes $x \in G$ ist die Abbildung
  \[
            G/G_x
    \to     G.x \,,
    \quad   g G_x
    \mapsto g.x
  \]
  eine wolhdefinierte Bijektion.
  Insbesondere gilt $\card{G.x} = \card{G/G_x} = \groupindex{G}{G_x}$.
\end{lemma}

Gleichung~\eqref{equation: orbit formula alpha} lässt sich somit zu
\[
    \card{X}
  = \sum_{i \in I} \groupindex{G}{G_{x_i}} \,
\]
umschreiben.
Dabei gilt genau dann $|G.x_i| = \groupindex{G}{G_{x_i}} = 1$, wenn $x_i$ ein Fixpunkt ist.
Durch umschreiben der obigen Summe ergibt sich damit die Bahnenformel:

\begin{corollary}[Bahnenformel]
  Ist $X$ eine $G$-Menge und $(x_i)_{i \in I}$ ein Repräsentantensystem der $G$-Bahnen von $X$, so gilt
  \[
      \card{X}
    =   \card*{X^G}
      + \sum_{\substack{i \in I \\ x_i \notin X^G}} \groupindex{G}{G_{x_i}}.
  \]

\end{corollary}





\section{\texorpdfstring{$p$}{p}-Gruppen und Sylowsätze}

Es sei $G$ eine endliche Gruppe und $p$ prim.

\begin{definition}
  Gilt $\ord{G} = p^n$ für ein $n \in \Natural$, so ist $G$ eine \emph{$p$-Gruppe}.
\end{definition}

\begin{definition}
  Eine \emph{$p$-Sylow\-unter\-gruppe} von $G$ eine $p$-Untergruppe $S \subgroup G$ mit $\ord{S} = p^r$ und $p \notdivides \groupindex{G}{S}$, d.h.\ es gilt $\ord{G} = p^r m$ mit $p \notdivides m$.
\end{definition}

\begin{theorem}[Sylowsätze]
  Es gelte $G = p^r m$ mit $p \notdivides m$.
  \begin{enumerate}
    \item
      Jede $p$-Untergruppe $H \subgroup G$ ist in einer $p$-Sylowuntergruppe von $G$ enthalten.
      Inbesondere ergibt sich für $H = \{1\}$, dass $G$ eine $p$-Sylowuntergruppe besitzt.
    \item
      Je zwei $p$-Sylowuntergruppen $S, S' \subseteq G$ sind konjugiert zueinander, d.h.\ es gibt ein $g \in G$ mit $g S g^{-1} = S'$.
    \item
      Bezeichnet $n_p$ die Anzahl der $p$-Sylowuntergruppen von $G$, so gilt
      \[
                n_p
        \equiv  1
        \pmod{p}
        \quad\text{und}\quad
        n_p \divides m \,.
      \]

  \end{enumerate}
\end{theorem}



\section{Klassifikation endlicher abelscher Gruppen}

\begin{theorem}
  Ist $G$ abelsch, so gibt es genau eine $p$-Sylowuntergruppe $S_p \subgroup G$.
  Ist die Primfaktorzerlegung von $\ord{G}$ durch $\ord{G} = p_1^{n_1} \dotsm p_r^{n_r}$ gegeben, so ist die Abbildung
  \[
                            S_{p_1} \times \dotsb \times S_{p_r}
    \xlongrightarrow{\sim}  G \,,
    \quad                   (g_1, \dotsc, g_r)
    \mapsto                 g_1 \dotsm g_r
  \]
  ein Isomorphismus.
\end{theorem}

\begin{proposition}
  Es sei $p$ prim.
  Jede abelsche $p$-Gruppe $G$ ist isomorph zu einem Produkt zyklischer $p$-Gruppen, d.h.\ es gilt
  \[
          G
    \cong \Integer/p^{n_1} \times \dotsm \times \Integer/p^{n_r}
  \]
  mit $n_1, \dotsc, n_r \geq 1$.
\end{proposition}

\begin{corollary}[Klassifikation endlicher abelscher Gruppen]
  Jede endliche abelsche Gruppe ist isomorph zu einem Produkt zyklischer $p$-Gruppen, d.h.\ es gilt
  \[
          G
    \cong \Integer/p_1^{n_1} \times \dotsm \times \Integer/p_r^{n_r}
  \]
  mit $p_1, \dotsc, p_r$ prim und $n_1, \dotsc, n_r \geq 1$.
\end{corollary}

\begin{remark}
  Tatsächlich ist diese Zerlegung bereits eindeutig bis auf Permutation der Faktoren, d.h.\ die Paare $(p_1, n_1), \dotsc, (p_r, n_r)$ sind eindeutig bis auf Permuation.
  Diese Eindeutigkeit wurde in der Vorlesung allerdings nicht gezeigt.
\end{remark}




