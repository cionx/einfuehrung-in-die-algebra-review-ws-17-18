\chapter{Körpererweiterungen}

\begin{definition}
  Es seien $K$, $L$ zwei Körper.
  Ist $K$ ein Unterring von $L$, so ist $K$ ein \emph{Unterkörper} von $L$.
  Dann ist $L$ eine \emph{Körpererweiterung} von $K$, notiert als $L/K$.
\end{definition}

\begin{definition}
  Ein Ringhomomorphismus $K \to L$ zwischen Körpern $K$, $L$ ist ein \emph{Körperhomomorphismus}.
\end{definition}

\begin{lemma}
  Jeder Körperhomomorphismus ist injektiv.
\end{lemma}

Ist $\varphi \colon K \to L$ ein Körperhomomorphismus, so induziert $\varphi$ also einen Isomorphismus von Körpern $\induced{\varphi} \colon  K \to \im(\varphi)$.
Indem man $K$ mit dem Unterkörper $\im(\varphi)$ von $L$ identifiziert, lässt sich $K$ als ein Unterkörper von $L$ auffassen.

\begin{definition}
  Ist $K$ ein Körper und $S \subseteq K$ eine Teilmenge, so ist
  \[
              \generated{S}_{\text{Körper}}
    \defined  \bigcap_{\substack{\text{Unterkörper} \\ K' \subseteq K \\S \subseteq K'}} K'
  \]
  der \emph{von $S$ erzeugte Unterkörper}.
  Für jeden Unterkörper $K' \subseteq K$ mit $S \subseteq K'$ gilt also $\generated{S}_{\text{Körper}} \subseteq K'$.

  Ist $L/K$ eine Körpererweiterung und $S \subseteq L$ eine Teilmenge, so ist
  \[
              K(S)
    \defined  \generated{K \cup S}_{\text{Körper}}
    =         \bigcap_{\substack{\text{Zwischenkörper} \\ K \subseteq L' \subseteq L \\ S \subseteq L'}} L'
  \]
  die \emph{von $S$ erzeugte Zwischenerweiterung}.
  Für jeden Unterkörper $L' \subseteq L$ mit $S \subseteq L'$ gilt also $K(S) \subseteq L$.
\end{definition}





\section{Der Primkörper}

Es sei $K$ ein Körper.
Dann ist $P \defined \bigcap_{\text{Unterkörper $K' \subseteq K$}} K'$ ein Unterkörper von $K$.
Es handelt sich um den kleinsten Unterkörper, der in $K$ enthalten ist, d.h.\ für jeden anderen Unterkörper $K' \subseteq L$ gilt $P \subseteq K'$.

\begin{definition}
  Der Körper $K$ wie oben ist der \emph{Primkörper} von $K$.
\end{definition}

Zur Bestimmung des Primkörpers betrachtet man den Ringhomomorphismus
\[
          \varphi
  \colon  \Integer
  \to     K \,,
  \quad   n
  \mapsto n \cdot 1_K \,.
\]
Gilt $\ringchar{K} = p > 0$, so gilt $\ker(\varphi) = \genideal{p}$, weshalb $\varphi$ einen Körperhomomorphismus
\[
          \induced{\varphi}
  \colon  \Finite_p
  \to     K \,,
  \quad   \class{n}
  \mapsto n \cdot 1_K \,.
\]
induziert.
Gilt $\ringchar{K} = 0$, so induziert $\varphi$ hingegen einen Körperhomomorphismus
\[
          \induced{\varphi}
  \colon  \Rational
  \to     K \,,
  \quad   \frac{n}{m}
  \mapsto \frac{n \cdot 1_K}{m \cdot 1_K} \,.
\]
In beiden Fällen ist $\im(\induced{\varphi})$ der von $1_K$ erzeugte Unterkörper von $K$, also der Primkörper $K$.
Es gilt also
\[
          P
  \cong \begin{cases}
          \Finite_p & \text{falls $\ringchar{K} = p > 0$} \,, \\
          \Rational & \text{falls $\ringchar{K} = 0$} \,.
        \end{cases}
\]
Mithilfe des obigen Isomorphismus $\induced{\varphi}$ wird dann $P$ mit $\Finite_p$, bzw.\ $\Rational$ identifiziert.

\begin{corollary}
  Sind $K$ und $L$ zwei Körper mit $\ringchar{K} \neq \ringchar{L}$, so gibt es keinen Körperhomomorphismus $K \to L$.
\end{corollary}





\section{Der Grad einer Körpererweiterung}

Es seien $M/L/K$ Körpererweiterungen.

\begin{definition}
  Der \emph{Grad} einer Körpererweiterung $L/K$ ist $\fieldindex{L}{K} \defined \dim_K(L)$.
\end{definition}

\begin{lemma}
  Es sei $V$ ein $L$-Vektorraum mit $L$-Basis $(v_i)_{i \in I}$ und $(b_j)_{j \in J}$ eine $K$-Basis von $L$.
  Dann ist $(b_j v_i)_{i \in I, j \in J}$ eine $K$-Basis von $V$, also $\dim_K(V) = \dim_K(L) \dim_L(V)$.
\end{lemma}

\begin{example}
  Für jeden $\Complex$-Vektorraum $V$ gilt $\dim_\Real(V) = 2 \dim_\Complex(V)$ mit $2 = \dim_\Real(\Complex)$.
\end{example}

\begin{corollary}[Multiplikativität des Grades]
  Es gilt $\fieldindex{M}{K} = \fieldindex{M}{L} \fieldindex{L}{K}$.
\end{corollary}

\begin{definition}
  Die Körpererweiterung $L/K$ ist \emph{endlich}, wenn $\fieldindex{L}{K}$ endlich ist.
\end{definition}





\section{Algebraizität}

Es seien $M/L/K$ eine Körpererweiterung.

\begin{lemma}
  Für $a \in L$ sind die folgenden Bedingungen äquivalent:
  \begin{enumerate}
    \item
      Die Körpererweiterung $K(a)/K$ ist endlich.
    \item
      Es gibt ein Polynom $p \in K[t]$ mit $p \neq 0$ und $p(a) = 0$.
    \item
      Es gilt $K[a] = K(a)$.
  \end{enumerate}
\end{lemma}

\begin{definition}
  Ein Element $a \in L$ ist \emph{algebraisch \textup(über $K$\textup)}, wenn es eine \textup(und damit alle\textup) der obigen Bedingungen erfüllt;
  andernfalls ist $a$ \emph{transzendent \textup(über $K$\textup)}.
  
  Die Körpererweiterung $L/K$ ist \emph{algebraisch}, wenn jedes $a \in L$ algebraisch über $K$ ist;
  andernfalls ist $L/K$ \emph{transzendent}.
\end{definition}

\begin{example}
% actual examples
  Ist $M/K$ algebraisch, so sind auch $M/L$ und $L/K$ algebraisch.
\end{example}

\begin{lemma}
  Ist die Erweiterung $L/K$ endlich, so ist sie algebraisch.
\end{lemma}

\begin{corollary}
  Sind $a, b \in L$ algebraisch, so sind auch $a + b$ und $a \cdot b$ algebraisch.
  Gilt $a \neq 0$, so ist auch $1/a$ algebraisch.
\end{corollary}

\begin{corollary}
  Es ist $(L/K)^\alg \defined \{ x \in L \suchthat \text{$x$ ist algebraisch über $K$} \}$ ein Zwischenkörper der Erweiterung $L/K$.
\end{corollary}

\begin{corollary}
  Für die Erweiterung $L/K$ sind die folgenden beiden Bedigungen äquivalent:
  \begin{enumerate}
    \item
      $L/K$ ist endlich.
    \item
      $L/K$ wird von endlich vielen algebraischen Elementen $a_1, \dotsc, a_n \in L$ erzeugt.
  \end{enumerate}
\end{corollary}

\begin{corollary}
  Die folgenden beiden Bedingungen sind äquivalent:
  \begin{enumerate}
    \item
      Die Erweiterung $L/K$ ist algebraisch.
    \item
      Die Erweiterung $L/K$ wird von algebraischen Elementen erzeugt, d.h.\ es gibt algebraische Elemente $a_i \in L$, $i \in I$ mit $L = K(a_i \suchthat i \in I)$.
  \end{enumerate}
\end{corollary}

\begin{corollary}[Transitivität von Algebraizität]
  Sind die Erweiterungen $M/L$ und $L/K$ beide algebraisch, so ist auch $M/K$ algebraisch.
\end{corollary}





\section{Einfache Körpererweiterungen}

Es sei $L/K$ eine Körpererweiterung.

\begin{definition}
  $L/K$ ist \emph{einfach}, wenn es ein $a \in L$ mit $L = K(a)$ gibt.
\end{definition}

Es sei zunächst $a \in L$ algebraisch über $K$.
Dann gilt $K(a) = K[a]$, weshalb $K(a)$ das Bild des Ringhomomorphismus
\[
          \varphi
  \colon  K[t]
  \to     L \,,
  \quad   p
  \mapsto p(a)
\]
ist.
Dann ist $\ker(\varphi)$ ein Ideal in $K[t]$, weshalb es ein eindeutiges normiertes Polynom $m_a \in K[t]$ mit $\ker(\varphi) = \genideal{m_a}$ gibt.

\begin{definition}
  Ist $a \in L$ algebraisch über $K$, so ist $m_a \in K[t]$ wie oben das \emph{Minimalpolynom von $a$ \textup(über $K$\textup)}.
\end{definition}

Der Ringhomomorphismus $\varphi$ induziert einen Ringisomorphismus
\[
                          K[t]/\genideal{m_a}
  \xlongrightarrow{\sim}  K(a) \,,
  \quad                   \class{p}
  \mapsto                 p(a) \,.
\]
Insbesondere gilt $K[t]/\genideal{m_a} \cong K(a)$, weshalb $K[t]/\genideal{m_a}$ ein Körper ist.
Das Ideal $\genideal{m_a}$ ist also maximal, und das Polynom $m_a$ somit irreduzibel.
Ist $p \in K[t]$ ein weiteres irreduzibles normiertes Polynom mit $p(a) = 0$, so gilt $m_a \divides p$ und somit $p = m_a$.

\begin{corollary}
  Ist $a \in L$ algebraisch über $K$, so ist $m_a \in K[t]$ das eindeutige normierte irreduzible Polynom, dass $a$ als Nullstelle hat.
\end{corollary}

Der Körperisomorphismus $\induced{\varphi}$ ist auch $K$-linear, und somit ein Isomorphismus von $K$-Vektorräumen.
Dabei ist für $\deg(m_a) = d$ eine $K$-Basis von $K[t]/(m_a)$ durch $\class{1}, \class{t}, \dotsc, \class{t^{d-1}}$ gegeben.

\begin{corollary}
  Ist $a \in L$ algebraisch über $K$, so gilt $\fieldindex{K(a)}{K} = \deg(m_a)$.
\end{corollary}

Ist $a \in L$ transzendent über $K$, so ist der Ringhomomorphismus $\varphi$ injektiv, und induziert deshalb einen Körperhomomorphismus
\[
          \induced{\varphi}
  \colon  K(t)
  \to     L \,,
  \quad   \frac{p}{q}
  \mapsto \frac{p(a)}{q(a)}
\]
dessen Bild gerade $K(a)$ ist.
Also gilt dann $K(a) \cong K(t)$.





\section{Das Kompositum}

Es sei $L/K$ eine Körpererweiterung, und es seien $L_1, L_2$ zwei Zwischenkörper dieser Erweiterung, d.h.\ es seien $L_1, L_2 \subseteq L$ Unterkörper mit $K \subseteq L_1, L_2$.

\begin{definition}
  Das \emph{Kompositum} der Zwischenkörper $L_1, L_2$ ist der Zwischenkörper
  \[
              L_1 L_2
    \defined  K(L_1 \cup L_2)
    =         L_1(L_2)
    =         L_2(L_1) \,.
  \]
\end{definition}

\begin{lemma}
  \begin{enumerate}
    \item
      Es gelten $\fieldindex{L_1}{K}, \fieldindex{L_2}{K} \divides \fieldindex{L_1 L_2}{K}$.
    \item
      Es gilt $\fieldindex{L_1 L_2}{K} \leq \fieldindex{L_1}{K} \fieldindex{L_2}{K}$.
  \end{enumerate}
\end{lemma}





\section{\texorpdfstring{$K$}{K}-Homomorphismen}

Es seien $L/K$ und $L'/K$ zwei Körpererweiterungen.

\begin{definition}
  Ein \emph{$K$-Homomorphismus} $\varphi \colon L_1 \to L_2$ ist ein $K$-linearer Körperhomomorphismus .
\end{definition}

\begin{remark}
  Die $K$-Linearität von $\varphi$ ist äquivalent dazu, dass $\restrict{\varphi}{K} = \id_K$ gilt.
\end{remark}

\begin{example}
  Ist $P$ der Primkörper von $K$, so ist jeder Körperhomomorphismus $L_1 \to L_2$ bereits $P$-linear, also ein $P$-Homomorphismus.
\end{example}

\begin{lemma}
  Es sei $a \in L$.
  \begin{enumerate}
    \item
      Ist $\varphi \colon L \to L'$ ein $K$-Homomorphismus, so gilt für jedes Polynom $f \in K[t]$, dass $f(\varphi(a)) = \varphi(f(a))$.
    \item
      Es ist $a$ genau dann eine Nullstelle von $f$, wenn $\varphi(a)$ eine Nullstelle von $f$ ist.
    \item
      Es ist $a$ genau dann algebraisch über $K$, wenn $\varphi(a)$ algebraisch über $K$ ist, und es gilt dann $m_a = m_{\varphi(a)}$.
  \end{enumerate}
\end{lemma}

\begin{remark}
  Es seien allgemeiner $L/K$ und $L'/K'$ Körpererweiterungen und $\psi \colon K \to K'$, $\varphi  \colon L \to L'$ Körperhomomorphismen mit $\restrict{\varphi }{K} = \psi$, d.h.\ so dass das folgende Diagramm kommutiert:
  \[
    \begin{tikzcd}
        L
        \arrow{r}[above]{\varphi}
      & L'
      \\
        K
        \arrow[hook]{u}
        \arrow{r}[above]{\psi}
      & K'
        \arrow[hook]{u}
    \end{tikzcd}
  \]
  Dann ist $a \in L$ genau dann eine Nullstelle von $f \in K[t]$, wenn $\varphi(a)$ eine Nullstelle von $\psi_*(f) \in L'[t]$ ist, wobei $\psi_*$ der von $\psi$ induzierte Ringhomomorphismus
  \[
            \psi_*
    \colon  K[t]
    \to     K'[t] \,,
    \quad   \sum_i a_i t^i
    \mapsto \sum_i \psi(a_i) t^i
  \]
  ist.
\end{remark}





\section{Der Algebraische Abschluss}

Es sei $K$ ein Körper.



\subsection{Hinzuadjungieren von Nullstellen}

Ist $f \in K[t]$ irreduzibel, so ist $L \defined K[t]/\genideal{f}$ ein Körper, und durch den Körperhomomorphismus $K \to L$, $x \mapsto \class{x}$ ergibt sich eine Körpererweiterung $L/K$.
Dabei gilt für das Element $a \defined \class{t} \in L$, dass $f(a) = 0$.
Es lässt sich also eine Körperweiterung $L/K$ konstruieren, in der $f$ eine Nullstelle hat.



\subsection{Definition des algebraischen Abschluss}

\begin{lemma}
  Die folgenden Bedingungen sind äquivalent:
  \begin{enumerate}
    \item
      Jedes nicht-konstante Polynom $p \in K[t]$ besitzt eine Nullstelle in $K$.
    \item
      Jedes Polynom $p \in K[t]$ zerfällt in Linearfaktoren.
    \item
      Für jede algebraische Körpererweiterung $L/K$ gilt bereits $L = K$.
  \end{enumerate}
\end{lemma}

\begin{definition}
  Erfüllt $K$ eine \textup(und damit alle\textup) der obigen Bedingungen, so ist $K$ \emph{algebraisch abgeschlossen}.
\end{definition}

\begin{example}
  Der \emph{Fundamentalsatz der Algebra} besagt, dass der Körper der komplexen Zahlen $\Complex$ algebraisch abgeschlossen ist.
\end{example}

\begin{definition}
  Ist $L/K$ eine Körpererweiterung, so ist $L$ ein \emph{algebraischer Abschluss von $K$}, wenn
  \begin{itemize}
    \item
      die Erweiterung $L/K$ algebraisch ist, und
    \item
      der Körper $L$ algebraisch abgeschlossen ist.
  \end{itemize}
\end{definition}

\begin{example}
  \begin{enumerate}
    \item
      Es ist $\Complex$ ein algebraischer Abschluss von $\Real$.
    \item
      Es ist $\Complex$ kein algebraischer Abschluss von $\Rational$, da $\Complex/\Rational$ nicht algebraisch ist.
  \end{enumerate}
\end{example}



\subsection{Existenz des Algebraischen Abschlusses}

Zur Konstruktion eines algebraischen Abschlusses von $K$ adjungiert man für alle nicht-konstanten Polynome $f \in K[t]$ \enquote{gleichzeitig} eine Nullstelle hinzu, und iteriert diesen Prozess anschließend:

Es sei $\mathcal{F} \subseteq K[t]$ die Menge aller nicht-konstanten Polynome.
Für den Polynomring $R \defined K[X_f \suchthat f \in \mathcal{F}]$ sei $I \defined \genideal{f(X_f) \suchthat f \in \mathcal{F}}$.
Dann ist $I$ ein echtes Ideal in $R$, und somit in einem maximalen Ideal $M \ideal R$ enthalten.
Dann ist $L_1 \defined R/M$ ein Körper, und jedes $f \in \mathcal{F}$ has eine Nullstelle in $L_1$ (nämlich $\class{X_f}$).
Induktiv ergibt sich, ausgehend von $L_0 \defined K$, eine aufsteigende Folge von Körpern
\[
            K
  =         L_0
  \subseteq L_1
  \subseteq L_2
  \subseteq L_3
  \subseteq \dotsb \,,
\]
so dass $L_{i+1}/L_i$ für alle $i$ algebraisch ist, und alle nicht-konstanten Polynome aus $L_i[t]$ in $L_{i+1}$ eine Nullstelle haben.
Dann ist $L \defined \bigcup_{i \geq 0} L_i$ eine Körper, so dass $L/K$ algebraisch ist, und jedes nicht-konstante Polynom aus $L[t]$ eine Nullstelle in $L$ hat.
Also ist $L$ ein algebraischer Abschluss von $K$.

\begin{theorem}
  Jeder Körper $K$ besitzt einen algebraischen Abschluss.
\end{theorem}



\subsection{Eindeutigkeit des Algebraischen Abschlusses}

\begin{lemma}[Forsetzungssätze für algebraische Abschlüsse]
  \leavevmode
  \begin{enumerate}
    \item
      Ist $L/K$ eine algebraische Körpererweiterung und $L'$ ein algebraisch abgeschlossener Körper, so setzt sich jeder Körperhomomorphismus $\varphi \colon K \to L'$ zu einem Körperhomomorphismus $\psi \colon L \to L'$ fort, d.h.\ es gilt $\restrict{\psi}{K} = \varphi$.
      \[
        \begin{tikzcd}
            L
            \arrow[dashed]{r}[above]{\psi}
          & L'
          \\
            K
            \arrow[hook]{u}
            \arrow{ru}[below right]{\varphi}
          & {}
        \end{tikzcd}
      \]
    \item
      Sind $K$, $K'$ zwei Körper mit zugehörigen algebraischen Abschlüssen $L$, $L'$, so setzt sich jeder Körperisomorphismus $\varphi \colon K \to K'$ zu einem Körperisomorphismus $\psi \colon L \to L'$ fort, d.h.\ es gilt $\restrict{\psi}{K} = \varphi$.
      \[
        \begin{tikzcd}
            L
            \arrow[dashed]{r}[above]{\psi}
          & L'
          \\
            K
            \arrow[hook]{u}
            \arrow{r}[above]{\varphi}
          & K'
            \arrow[hook]{u}
        \end{tikzcd}
      \]
  \end{enumerate}
\end{lemma}


\begin{corollary}
  Je zwei algebraische Abschlüsse $\closure{K}, \closure{K}'$ von $K$ sind $K$-isomorph.
\end{corollary}

Wegen dieser Eindeutigkeit bis auf $K$-Isomorphismus spricht man auch von \emph{dem} algebraischen Abschluss von $K$, und notiert diesen mit $\closure{K}$.





\section{Zerfällungskörper}

Es sei $L/K$ eine Körpererweiterung.

\begin{definition}
  Es ist $L$ ein \emph{Zerfällungskörper} einer Familie $(f_i)_{i \in I}$ von Polynomen $f_i \in K[t]$, wenn jedes $f_i$ über $L$ in Linearfaktoren zerfällt, und die Erweiterung $L/K$ von den Nullstellen der $f_i$ erzeugt wird.
  Für jeden Zwischenkörper $L/L'/K$, so dass alle $f_i$ über $L'$ in Linearfaktoren zerfallen, gilt also bereits $L' = L$.
\end{definition}

\begin{lemma}
  Für jede Familie $(f_i)_{i \in I}$ nicht-konstanter Polynome $f_i \in K[t]$ existiert ein Zerfällungskörper.
\end{lemma}

\begin{lemma}
  Es sei $(f_i)_{i \in I}$ eine Familie von Polynomen $f_i \in K[t]$.
  Es sei $\varphi \colon K \to K'$ ein Körperisomorphismus und $\varphi_* \colon K[t] \to K'[t]$, $\sum_i a_i t^i \mapsto \sum_i \varphi(a_i) t^i$ der induzierte Ringhomomorphismus.
  Ist $L$ ein Zerfällungkörper der Familie $(f_i)_{i \in I}$ und $L'$ ein Zerfällungskörper der Familie $(\varphi_*(f_i))_{i \in I}$, so setzt sich $\varphi$ zu einem Körperisomorphismus $\psi \colon L \to L'$ fort, d.h.\ es gilt $\restrict{\psi}{K} = \varphi$.
  \[
    \begin{tikzcd}
      L
        \arrow{r}[above]{\psi}
      & L'
      \\
        K
        \arrow[hook]{u}
        \arrow{r}[above]{\varphi}
      & K'
        \arrow[hook]{u}
    \end{tikzcd}
  \]
\end{lemma}

\begin{corollary}
  Je zwei Zerfällungskörper $L, L'$ einer Familie $(f_i)_{i \in I}$ von Polynomen $f_i \in K[t]$ sind $K$-isomorph.
\end{corollary}

Ist $L$ ein Zerfällungskörper einer Familie von Polynom $(f_i)_{i \in I}$, so wird die Erweiterung $L/K$ von den Nullstellen der $f_i$ erzeugt.
Ist $L'/K$ eine weitere Körpererweiterung und $\varphi \colon L \to L'$ ein $K$-Homomorphismus, so ist für jede Nullstelle $a \in L$ von $f_i$, auch $\varphi(a) \in L'$ eine Nullstelle von $f_i$.
Für die Menge $N = \{ a_1, \dotsc, a_n \}$ aller Nullstellen von $f_i$ in $L$ ist deshalb das Bild $\varphi(N)$ die Menge aller Nullstellen von $f_i$ in $L'$.
Deshalb ist $\varphi(N)$ unabhängig von $\varphi$.
Da $L$ von den Nullstellen der $f_i$ erzeugt wird, ist somit $\varphi(L)$ unabhängig von der Wahl von $f_i$.
Dies führt zu folgendem Ergebnis:

\begin{proposition}
  Die folgenden Bedingungen sind äquivalent:
  \begin{enumerate}
    \item
      Es ist $L$ der Zerfällungskörper einer Familie $(f_i)_{i \in I}$ von Polynomen $f_i \in K[t]$.
    \item
      Für jede Körpererweiterung $L'/K$ haben alle Körperhomomorphismen $L \to L'$ das gleiche Bild.
    \item
      Für jede algebraische Körpererweiterung $L'/K$ haben alle Körperhomomorphismen $L \to L'$ das gleiche Bild.
    \item
      Ist $\closure{K}$ ein algebraischer Abschluss von $K$, so haben alle $K$-Homomorphismen $L \to \closure{K}$ das gleiche Bild.
    \item
      Ist $\closure{K}$ ein algebraischer Abschluss von $K$, so schränkt sich jeder $K$-Homomorphismus $L \to \closure{K}$ zu einem $K$-Automorphismus $L \to L$ ein.
  \end{enumerate}
\end{proposition}

\begin{definition}
  Erfüllt die Erweiterung $L/K$ eine \textup(und damit alle\textup) der obigen Bedingungen, so ist die Erweiterung $L/K$ \emph{normal}.
\end{definition}

\begin{lemma}
  Sind $M/L/K$ Körpererweiterungen, so dass $M/K$ normal ist, so ist auch $M/L$ normal.
\end{lemma}

\begin{warning}
  \begin{enumerate}
    \item
      Ist $M/K$ normal, so ist $L/K$ nicht notwendigerweise normal.
    \item
      Sind $M/L$ und $L/K$ normal, so ist $M/K$ nicht notwendigerweise normal.
  \end{enumerate}
\end{warning}





\section{Separablität}









\section{Klassifikation endlicher Körper}

Es sei $p$ eine Primzahl.
Im Folgenden seien alle Körper und Ringe von Charakteristik~$p$.

\begin{lemma}
  Ist $K$ ein endlicher Körper, so gilt $\card{\Finite_p} = p^n$ für $n = \fieldindex{K}{\Finite_p}$.
\end{lemma}

% \begin{lemma}
%   Ist $R$ ein Ring, so ist die Abbildung $\sigma \colon R \to R$, $x \mapsto x^p$ ein Ringhomomorphismus.
% \end{lemma}
% 
% \begin{definition}
%   Der Ringhomomorphismus $\sigma$ wie oben ist der \emph{Frobenius-Homo\-mor\-phis\-mus} von $R$.
% \end{definition}
% 
% Ist $K$ ein endlicher Körper, so ist der Frobenius-Homomorphismus $\sigma \colon K \to K$ bereits ein Körperautomorphismus.














