\chapter{Körpererweiterungen}

\begin{definition}
  Es seien $K$, $L$ zwei Körper.
  Ist $K$ ein Unterring von $L$, so ist $K$ ein \emph{Unterkörper} von $L$.
  Dann ist $L$ eine \emph{Körpererweiterung} von $K$, notiert als $L/K$.
\end{definition}

\begin{lemma}
  Ist $L$ ein Körper, so ist eine Teilmenge $K \subseteq L$ genau dann ein Unterkörper, wenn für alle $x, y, z \in K$, $z \neq 0$ auch
  \[
    1 \in K \,,
    \qquad
    x + y \in K \,,
    \qquad
    xy \in K \,,
    \qquad
    z^{-1} \in K \,.
  \]

\end{lemma}

\begin{definition}
  Ein Ringhomomorphismus $K \to L$ zwischen Körpern $K$, $L$ ist ein \emph{Körperhomomorphismus}.
\end{definition}

\begin{lemma}
  Jeder Körperhomomorphismus ist injektiv.
\end{lemma}

Ist $\varphi \colon K \to L$ ein Körperhomomorphismus, so induziert $\varphi$ einen Körperisomorphismus $\induced{\varphi} \colon  K \to \im(\varphi)$.
Indem man $K$ mit dem Unterkörper $\im(\varphi)$ von $L$ identifiziert, lässt sich $K$ als ein Unterkörper von $L$ auffassen.

\begin{definition}
  Ist $K$ ein Körper und $S \subseteq K$ eine Teilmenge, so ist
  \[
              \generated{S}_{\text{Körper}}
    \defined  \bigcap_{\substack{\text{Unterkörper} \\ K' \subseteq K \\S \subseteq K'}} K'
  \]
  der \emph{von $S$ erzeugte Unterkörper}.
  Für jeden Unterkörper $K' \subseteq K$ mit $S \subseteq K'$ gilt also $\generated{S}_{\text{Körper}} \subseteq K'$.

  Ist $L/K$ eine Körpererweiterung und $S \subseteq L$ eine Teilmenge, so ist
  \[
              K(S)
    \defined  \generated{K \cup S}_{\text{Körper}}
    =         \bigcap_{\substack{\text{Zwischenkörper} \\ K \subseteq L' \subseteq L \\ S \subseteq L'}} L'
  \]
  die \emph{von $S$ erzeugt} Zwischenerweiterung.
  Für jeden Zwischenkörper $K \subseteq L' \subseteq L$ mit $S \subseteq L'$ gilt also $K(S) \subseteq L$.
\end{definition}





\section{Der Primkörper}

Es sei $K$ ein Körper.
Dann ist
\[
            P
  \defined  \bigcap_{\substack{\text{Unterkörper} \\ K' \subseteq K}} K'
  =         \generated{1}_{\text{Körper}}
  =         \generated{\emptyset}_{\text{Körper}}
\]
ein Unterkörper von $K$.
Es handelt sich um den kleinsten Unterkörper, der in $K$ enthalten ist, d.h.\ für jeden anderen Unterkörper $K' \subseteq L$ gilt $P \subseteq K'$.

\begin{definition}
  Der Körper $P$ wie oben ist der \emph{Primkörper} von $K$.
\end{definition}

Zur näheren Bestimmung des Primkörpers betrachtet man den Ringhomomorphismus
\[
          \varphi
  \colon  \Integer
  \to     K \,,
  \quad   n
  \mapsto n \cdot 1_K \,.
\]
Gilt $\ringchar{K} = p > 0$, so gilt $\ker(\varphi) = \genideal{p}$, weshalb $\varphi$ einen Körperhomomorphismus
\[
          \induced{\varphi}
  \colon  \Finite_p
  \to     K \,,
  \quad   \class{n}
  \mapsto n \cdot 1_K
\]
induziert.
Gilt $\ringchar{K} = 0$, so induziert $\varphi$ hingegen einen Körperhomomorphismus
\[
          \induced{\varphi}
  \colon  \Rational
  \to     K \,,
  \quad   \frac{n}{m}
  \mapsto \frac{n \cdot 1_K}{m \cdot 1_K} \,.
\]
In beiden Fällen ist $\im(\induced{\varphi})$ der von $1_K$ erzeugte Unterkörper von $K$, also der Primkörper $P$.
Es gilt also
\[
          P
  \cong \begin{cases}
          \Finite_p & \text{falls $\ringchar{K} = p > 0$} \,, \\
          \Rational & \text{falls $\ringchar{K} = 0$} \,.
        \end{cases}
\]
Mithilfe des obigen Isomorphismus $\induced{\varphi}$ wird $P$ im Folgenden mit $\Finite_p$, bzw.\ $\Rational$ identifiziert.
Ist dabei $K' \subseteq K$ ein Unterkörper, so haben $K$ und $K'$ den gleichen Primkörper.

\begin{corollary}
  Sind $K$ und $L$ zwei Körper mit $\ringchar{K} \neq \ringchar{L}$, so gibt es keinen Körperhomomorphismus $K \to L$.
\end{corollary}





\section{Der Grad einer Körpererweiterung}

Es seien $M/L/K$ Körpererweiterungen.

\begin{definition}
  Der \emph{Grad} einer Körpererweiterung $L/K$ ist $\fieldindex{L}{K} \defined \dim_K(L)$.
\end{definition}

\begin{lemma}
  Es sei $V$ ein $L$-Vektorraum mit $L$-Basis $(v_i)_{i \in I}$ und $(b_j)_{j \in J}$ eine $K$-Basis von $L$.
  Dann ist $(b_j v_i)_{i \in I, j \in J}$ eine $K$-Basis von $V$, also $\dim_K(V) = \dim_K(L) \dim_L(V)$.
\end{lemma}

\begin{example}
  Für jeden komplexen Vektorraum $V$ gilt $\dim_\Real(V) = 2 \dim_\Complex(V)$.
\end{example}

\begin{corollary}[Multiplikativität des Grades]
  Es gilt $\fieldindex{M}{K} = \fieldindex{M}{L} \fieldindex{L}{K}$.
\end{corollary}

\begin{definition}
  Die Körpererweiterung $L/K$ ist \emph{endlich}, wenn $\fieldindex{L}{K}$ endlich ist.
\end{definition}





\section{Algebraizität}

Es seien $M/L/K$ eine Körpererweiterung.

\begin{lemma}
  Für $a \in L$ sind die folgenden Bedingungen äquivalent:
  \begin{enumerate}
    \item
      Die Körpererweiterung $K(a)/K$ ist endlich.
    \item
      Es gibt ein Polynom $p \in K[t]$ mit $p \neq 0$ und $p(a) = 0$.
    \item
      Es gilt $K[a] = K(a)$.
  \end{enumerate}
\end{lemma}

\begin{definition}
  Ein Element $a \in L$ ist \emph{algebraisch \textup(über $K$\textup)}, wenn es eine \textup(und damit alle\textup) der obigen Bedingungen erfüllt;
  andernfalls ist $a$ \emph{transzendent \textup(über $K$\textup)}.
  
  Die Körpererweiterung $L/K$ ist \emph{algebraisch}, wenn jedes $a \in L$ algebraisch über $K$ ist;
  andernfalls ist $L/K$ \emph{transzendent}.
\end{definition}

\begin{lemma}
  Ist $M/K$ algebraisch, so sind auch $M/L$ und $L/K$ algebraisch.
\end{lemma}

\begin{lemma}
  Ist die Erweiterung $L/K$ endlich, so ist $L/K$ auch algebraisch.
\end{lemma}

\begin{corollary}
  Sind $a, b \in L$ algebraisch, so sind auch $a + b$ und $a \cdot b$ algebraisch.
  Gilt $a \neq 0$, so ist auch $1/a$ algebraisch.
\end{corollary}

\begin{corollary}
  Es ist $(L/K)^\alg \defined \{ x \in L \suchthat \text{$x$ ist algebraisch über $K$} \}$ ein Zwischenkörper der Erweiterung $L/K$.
\end{corollary}

\begin{corollary}
  Für die Erweiterung $L/K$ sind die folgenden Bedingungen äquivalent:
  \begin{enumerate}
    \item
      $L/K$ ist endlich.
    \item
      $L/K$ wird von endlich vielen algebraischen Elementen $a_1, \dotsc, a_n \in L$ erzeugt.
  \end{enumerate}
\end{corollary}

\begin{corollary}[Transitivität von Algebraizität]
  Sind die Erweiterungen $M/L$ und $L/K$ beide algebraisch, so ist auch $M/K$ algebraisch.
\end{corollary}

\begin{corollary}
  Die folgenden beiden Bedingungen sind äquivalent:
  \begin{enumerate}
    \item
      Die Erweiterung $L/K$ ist algebraisch.
    \item
      Die Erweiterung $L/K$ wird von algebraischen Elementen erzeugt, d.h.\ es gibt algebraische Elemente $a_i \in L$, $i \in I$ mit $L = K(a_i \suchthat i \in I)$.
  \end{enumerate}
\end{corollary}






\section{Einfache Körpererweiterungen}

Es sei $L/K$ eine Körpererweiterung.

\begin{definition}
  $L/K$ ist \emph{einfach}, wenn es ein $a \in L$ mit $L = K(a)$ gibt.
\end{definition}

Es sei zunächst $a \in L$ algebraisch über $K$.
Dann gilt $K(a) = K[a]$, weshalb $K(a)$ das Bild des Ringhomomorphismus
\[
          \varphi
  \colon  K[t]
  \to     L \,,
  \quad   p
  \mapsto p(a)
\]
ist.
Dann ist $\ker(\varphi)$ ein Ideal in $K[t]$, weshalb es ein eindeutiges normiertes Polynom $m_a \in K[t]$ mit $\ker(\varphi) = \genideal{m_a}$ gibt.

\begin{definition}
  Ist $a \in L$ algebraisch über $K$, so ist $m_a \in K[t]$ wie oben das \emph{Minimalpolynom von $a$ \textup(über $K$\textup)}.
\end{definition}

Der Ringhomomorphismus $\varphi$ induziert einen Ringisomorphismus
\[
                          K[t]/\genideal{m_a}
  \xlongrightarrow{\sim}  K(a) \,,
  \quad                   \class{p}
  \mapsto                 p(a) \,.
\]
Insbesondere gilt $K[t]/\genideal{m_a} \cong K(a)$, weshalb $K[t]/\genideal{m_a}$ ein Körper ist.
Das Ideal $\genideal{m_a}$ ist also maximal, und das Polynom $m_a$ somit irreduzibel.
Ist $p \in K[t]$ ein weiteres irreduzibles normiertes Polynom mit $p(a) = 0$, so gilt $m_a \divides p$ und somit bereits $p = m_a$.

\begin{corollary}
  Ist $a \in L$ algebraisch über $K$, so ist $m_a \in K[t]$ das eindeutige normierte irreduzible Polynom, das $a$ als Nullstelle hat.
\end{corollary}

\begin{example}
  \label{example: minimal polynomials}
  \begin{enumerate}
    \item
      Für die Erweiterung $\Complex/\Real$ ist $f \defined t^2 + 1 \in \Real[t]$ das Minimalpolynom von $i \in \Complex$:
      Das Polynom $f$ ist normiert mit $f(i) = 0$.
      Es ist irreduzibel in $\Real[t]$, da es quadratisch ist, und keine Nullstelle in $\Real$ besitzt.
    \item
      Für die Erweiterung $\Rational(\sqrt{2})/\Rational$ ist $f \defined t^2 - 2 \in \Rational[t]$ das Minimalpolynom von $\sqrt{2} \in \Rational(\sqrt{2})$:
      Das Polynom $f$ ist normiert mit $f(\sqrt{2}) = 0$, und irreduzibel nach dem Eisenstein-Kriterium.
  \end{enumerate}
\end{example}

Der Körperisomorphismus $\induced{\varphi}$ ist auch $K$-linear, und somit ein Isomorphismus von $K$-Vektorräumen.
Dabei ist für $\deg(m_a) = d$ eine $K$-Basis von $K[t]/(m_a)$ durch $\class{1}, \class{t}, \dotsc, \class{t^{d-1}}$ gegeben.

\begin{corollary}
  Ist $a \in L$ algebraisch über $K$, so gilt $\fieldindex{K(a)}{K} = \deg(m_a)$.
\end{corollary}

Ist hingegen $a \in L$ transzendent über $K$, so ist der Ringhomomorphismus $\varphi$ injektiv, und induziert deshalb einen Körperhomomorphismus
\[
          \induced{\varphi}
  \colon  K(t)
  \to     L \,,
  \quad   \frac{p}{q}
  \mapsto \frac{p(a)}{q(a)} \,,
\]
dessen Bild gerade $K(a)$ ist.
Also gilt dann $K(a) \cong K(t)$.





\section{Das Kompositum}

Es sei $L/K$ eine Körpererweiterung, und es seien $L_1, L_2$ zwei Zwischenkörper dieser Erweiterung, d.h.\ es seien $L_1, L_2 \subseteq L$ Unterkörper mit $K \subseteq L_1, L_2$.


\begin{definition}
  Das \emph{Kompositum} der Zwischenkörper $L_1, L_2$ ist der Zwischenkörper
  \[
              L_1 L_2
    \defined  K(L_1 \cup L_2)
    =         L_1(L_2)
    =         L_2(L_1) \,,
  \]
  also der kleinste Zwischenkörper der Erweiterung $L/K$, der $L_1$ und $L_2$ enthält.
\end{definition}

Dies lässt sich wie folgt darstellen:
\[
  \begin{tikzcd}
      {}
    & L_1 L_2
    & {}
    \\
      L_1
      \arrow[-]{ru}
    & {}
    & L_2
      \arrow[-]{lu}
    \\
      {}
    & K
      \arrow[-]{lu}
      \arrow[-]{ru}
    & {}
  \end{tikzcd}
\]
Dabei lässt sich auch noch die Zwischenerweiterung $L_1 \cap L_2$ einzeichnen:
\[
  \begin{tikzcd}
      {}
    & L_1 L_2
    & {}
    \\
      L_1
      \arrow[-]{ru}
    & {}
    & L_2
      \arrow[-]{lu}
    \\
      {}
    & L_1 \cap L_2
      \arrow[-]{lu}
      \arrow[-]{ru}
    & {}
    \\
      {}
    & K
      \arrow[-]{u}
    & {}
  \end{tikzcd}
\]

\begin{proposition}
  Es seien $L_1/K$, $L_2/K$ algebraisch.
  \begin{enumerate}
    \item
      Es gelten $\fieldindex{L_1}{K}, \fieldindex{L_2}{K} \divides \fieldindex{L_1 L_2}{K}$.
    \item
      Ist $(a_i)_{i \in I}$ eine $K$-Basis von $L_1$ und $(b_j)_{j \in J}$ eine $K$-Basis von $L_2$, so ist $(a_i b_j)_{i \in I, j \in J}$ ein $K$-Erzeugendensystem von $L_1 L_2$.
    \item
      Es gilt $\fieldindex{L_1 L_2}{K} \leq \fieldindex{L_1}{K} \fieldindex{L_2}{K}$.
  \end{enumerate}
\end{proposition}

\begin{corollary}
  Sind $L_1/K$, $L_2/K$ endlich und $\fieldindex{L_1}{K}$, $\fieldindex{L_2}{K}$ teilerfremd, so gilt
  \[
      \fieldindex{L_1 L_2}{K}
    = \fieldindex{L_1}{K} \fieldindex{L_2}{K} \,.
  \]
\end{corollary}






\section{\texorpdfstring{$K$}{K}-Homomorphismen}

Es seien $L/K$ und $L'/K$ zwei Körpererweiterungen eines Körpers $K$.

\begin{definition}
  Ein $K$-linearer Körperhomomorphismus $\varphi \colon L \to L'$ ist ein \emph{$K$-Homomorphismus}.
  Ist $\varphi$ zudem ein Isomorphismus, bzw.\ Automorphismus, so ist $\varphi$ ein \emph{$K$-Isomorphismus}, bzw.\ \emph{$K$-Automorphismus}.
\end{definition}


\begin{remark}
  Die $K$-Linearität von $\varphi$ ist äquivalent dazu, dass $\restrict{\varphi}{K} = \id_K$ gilt, sowie äquivalent zur Kommutativität des folgenden Diagramms:
  \[
    \begin{tikzcd}
        L
        \arrow{rr}[above]{\varphi}
      & {}
      & L'
      \\
        {}
      & K
        \arrow[right hook->]{ru}
        \arrow[left hook->]{lu}
      & {}
    \end{tikzcd}
  \]
\end{remark}

\begin{example}
  Fasst man zwei Körper $L, L'$ gleicher Charakteristik als Körpererweiterungen $L/P$, $L'/P$ des gemeinsamen Primkörpers $P$ auf, so ist jeder Körperhomomorphismus $\varphi \colon L \to L'$ bereits $P$-linear, und somit ein $P$-Homomorphismus:
  Es gibt nämlich nur genau einen Körperhomomorphismus $P \to L'$, weshalb das Diagramm
  \[
    \begin{tikzcd}
        L_1
        \arrow{rr}[above]{\varphi}
      & {}
      & L_2
      \\
        {}
      & P
        \arrow[right hook->]{ru}
        \arrow[left hook->]{lu}
      & {}
    \end{tikzcd}
  \]
  notwendigerweise kommutiert.
\end{example}

\begin{definition}
  Es ist $\Aut{L/K} \defined \{\text{$K$-Automorphismen $L \to L$}\}$ die \emph{Automorphismengruppe} der Erweiterung $L/K$.
\end{definition}

Wie der Name vermuten lässt, bildet $\Aut{L/K}$ zusammen mit der Komposition von Abbildungen eine Gruppe.

\begin{lemma}
  \label{lemma: homomorphisms are automorphisms for finite extensions}
  Ist $L/K$ endlich, so ist jeder $K$-Homomorphismus $L \to L$ bereits ein $K$-Automorphismus.
\end{lemma}

\begin{remark}
  Lemma~\ref{lemma: homomorphisms are automorphisms for finite extensions} gilt allgemeiner für algebraische Erweiterungen $L/K$.
\end{remark}

\begin{lemma}
  Es sei $a \in L$.
  \begin{enumerate}
    \item
      Ist $\varphi \colon L \to L'$ ein $K$-Homomorphismus, so gilt für jedes Polynom $f \in K[t]$, dass $f(\varphi(a)) = \varphi(f(a))$.
    \item
      Es ist $a$ genau dann eine Nullstelle von $f$, wenn $\varphi(a)$ eine Nullstelle von $f$ ist.
    \item
      Es ist $a$ genau dann algebraisch über $K$, wenn $\varphi(a)$ algebraisch über $K$ ist, und es gilt dann $m_a = m_{\varphi(a)}$.
  \end{enumerate}
\end{lemma}

\begin{remark}
  Es seien allgemeiner $L/K$ und $L'/K'$ Körpererweiterungen und $\psi \colon K \to K'$, $\varphi  \colon L \to L'$ Körperhomomorphismen mit $\restrict{\varphi }{K} = \psi$, d.h.\ so dass das folgende Diagramm kommutiert:
  \[
    \begin{tikzcd}
        L
        \arrow{r}[above]{\varphi}
      & L'
      \\
        K
        \arrow[left hook ->]{u}
        \arrow{r}[above]{\psi}
      & K'
        \arrow[left hook ->]{u}
    \end{tikzcd}
  \]
  Dann ist $a \in L$ genau dann eine Nullstelle von $f \in K[t]$, wenn $\varphi(a)$ eine Nullstelle von $\psi_*(f) \in K'[t]$ ist, wobei $\psi_*$ der von $\psi$ induzierte Ringhomomorphismus
  \[
            \psi_*
    \colon  K[t]
    \to     K'[t] \,,
    \quad   \sum_{i=0}^n a_i t^i
    \mapsto \sum_{i=0}^n \psi(a_i) t^i
  \]
  ist.
\end{remark}

\begin{theorem}
  \label{theorem: zeroes to zeroes}
  Es sei $a \in L$.
  Für jede Nullstelle $a' \in L'$ des Minimalpolynoms $m_a \in K[t]$ gibt es dann einen eindeutigen $K$-Homomorphismus $\varphi \colon K(a) \to L'$ mit $\varphi(a) = a'$.
  In anderen Worten:
  Es ergibt sich eine Bijektion
  \begin{align*}
                            \{ \text{$K$-Homomorphismen $K(a) \to L'$} \}
    &\xlongrightarrow{\sim} \{ \text{Nullstellen von $m_a$ in $L'$} \} \,,  \\
                            \varphi
    &\longmapsto            \varphi(a) \,.
  \end{align*}
\end{theorem}

\begin{example}
  Die Erweiterung $\Complex/\Real$ ist endlich mit $\Complex = \Real(i)$, wobei $t^2 + 1 \in \Real[t]$ das Minimalpolynom von $i$ ist.
  Es gilt somit
  \[
  \arraycolsep=1pt
  \begin{array}{rcl}
      \Aut{\Complex/\Real}
    & \xlongequal{\ref{lemma: homomorphisms are automorphisms for finite extensions}}
    & \{ \text{$\Real$-Homomorphismen $\Real(i) \to \Complex$} \}
    \\
      {}
    & \xleftrightarrow{\ref{theorem: zeroes to zeroes}}
    & \{ \text{Nullstellen von $t^2 + 1 \in \Complex$} \}
    \\
      {}
    & \xlongequal{\phantom{\ref{lemma: homomorphisms are automorphisms for finite extensions}}}
    & \{i, -i\} \,.
  \end{array}
  \]
\end{example}






\section{Der Algebraische Abschluss}

Es sei $K$ ein Körper.



\subsection{Hinzuadjungieren von Nullstellen}

Ist $f \in K[t]$ irreduzibel, so ist $L \defined K[t]/\genideal{f}$ ein Körper, und durch den Körperhomomorphismus $K \to L$, $x \mapsto \class{x}$ ergibt sich eine Körpererweiterung $L/K$.
Dabei gilt für das Element $a \defined \class{t} \in L$, dass $f(a) = 0$.
Es lässt sich also eine Körperweiterung $L/K$ konstruieren, in der $f$ eine Nullstelle hat, und diese Körpererweiterung wird von dieser Nullstelle erzeugt.
Man kann also Nullstellen von Polynomen zu $K$ \enquote{hinzuadjungieren}.



\subsection{Definition des algebraischen Abschluss}

\begin{lemma}
  Für $K$ sind die folgenden Bedingungen äquivalent:
  \begin{enumerate}
    \item
      Jedes nicht-konstante Polynom $p \in K[t]$ besitzt eine Nullstelle in $K$.
    \item
      Jedes Polynom $p \in K[t]$ zerfällt in Linearfaktoren.
    \item
      Die normierten irreduziblen Polynome in $K[t]$ sind genau die Linearfaktoren.
    \item
      Für jede algebraische Körpererweiterung $L/K$ gilt bereits $L = K$.
  \end{enumerate}
\end{lemma}

\begin{definition}
  Erfüllt $K$ eine \textup(und damit alle\textup) der obigen Bedingungen, so ist $K$ \emph{algebraisch abgeschlossen}.
\end{definition}

\begin{example}
  Der \emph{Fundamentalsatz der Algebra} besagt, dass der Körper der komplexen Zahlen $\Complex$ algebraisch abgeschlossen ist.
\end{example}

\begin{definition}
  Ist $L/K$ eine Körpererweiterung, so ist $L$ ein \emph{algebraischer Abschluss von $K$}, wenn
  \begin{itemize}
    \item
      die Erweiterung $L/K$ algebraisch ist, und
    \item
      der Körper $L$ algebraisch abgeschlossen ist.
  \end{itemize}
\end{definition}

\begin{example}
  \begin{enumerate}
    \item
      Nach dem Fundamentalsatz der Algebra ist $\Complex$ ein algebraischer Abschluss von $\Real$.
      (Die Algebraizität der Erweiterung $\Complex/\Real$ folgt aus ihrer Endlichkeit.)
    \item
      Es ist $\Complex$ kein algebraischer Abschluss von $\Rational$, da $\Complex/\Rational$ nicht algebraisch ist.
  \end{enumerate}
\end{example}



\subsection{Existenz des Algebraischen Abschlusses}

Zur Konstruktion eines algebraischen Abschlusses von $K$ adjungiert man für alle nicht-konstanten Polynome $f \in K[t]$ \enquote{gleichzeitig} eine Nullstelle hinzu, und iteriert diesen Prozess anschließend:

Es sei $\mathcal{F} \subseteq K[t]$ die Menge aller nicht-konstanten Polynome.
Für den Polynomring $R \defined K[X_f \suchthat f \in \mathcal{F}]$ sei $I \defined \genideal{f(X_f) \suchthat f \in \mathcal{F}}$.
Dann ist $I$ ein echtes Ideal in $R$, und somit in einem maximalen Ideal $M \ideal R$ enthalten.
Dann ist $L_1 \defined R/M$ ein Körper, und jedes $f \in \mathcal{F}$ has eine Nullstelle in $L_1$ (nämlich $\class{X_f}$).
Induktiv ergibt sich, ausgehend von $L_0 \defined K$, eine aufsteigende Folge von Körpern
\[
            K
  =         L_0
  \subseteq L_1
  \subseteq L_2
  \subseteq L_3
  \subseteq \dotsb \,,
\]
so dass $L_{i+1}/L_i$ für alle $i$ algebraisch ist, und alle nicht-konstanten Polynome aus $L_i[t]$ in $L_{i+1}$ eine Nullstelle haben.
Dann ist $L \defined \bigcup_{i \geq 0} L_i$ eine Körper, so dass $L/K$ algebraisch ist, und jedes nicht-konstante Polynom aus $L[t]$ eine Nullstelle in $L$ hat.
Also ist $L$ ein algebraischer Abschluss von $K$.

\begin{theorem}
  Jeder Körper $K$ besitzt einen algebraischen Abschluss.
\end{theorem}



\subsection{Eindeutigkeit des Algebraischen Abschlusses}

\begin{lemma}[Forsetzungssätze für algebraische Abschlüsse]
  \leavevmode
  \begin{enumerate}
    \item
      Ist $L/K$ eine algebraische Körpererweiterung und $L'$ ein algebraisch abgeschlossener Körper, so setzt sich jeder Körperhomomorphismus $\varphi \colon K \to L'$, also jede Einbettung $K \hookrightarrow L'$ zu einem Körperhomomorphismus $\psi \colon L \to L'$, also zu einer Einbettung $L \hookrightarrow{L'}$ fort, d.h.\ es gilt $\restrict{\psi}{K} = \varphi$.
      \[
        \begin{tikzcd}
            L
            \arrow[dashed]{r}[above]{\psi}
          & L'
          \\
            K
            \arrow[left hook ->]{u}
            \arrow{ru}[below right]{\varphi}
          & {}
        \end{tikzcd}
      \]
    \item
      Sind $K$, $K'$ zwei Körper mit zugehörigen algebraischen Abschlüssen $L$, $L'$, so setzt sich jeder Körperisomorphismus $\varphi \colon K \to K'$ zu einem Körperisomorphismus $\psi \colon L \to L'$ fort, d.h.\ es gilt $\restrict{\psi}{K} = \varphi$.
      \[
        \begin{tikzcd}
            L
            \arrow[dashed]{r}[below]{\psi}[above]{\sim}
          & L'
          \\
            K
            \arrow[left hook ->]{u}
            \arrow{r}[below]{\varphi}[above]{\sim}
          & K'
            \arrow[left hook ->]{u}
        \end{tikzcd}
      \]
  \end{enumerate}
\end{lemma}

\begin{corollary}
  Je zwei algebraische Abschlüsse $\closure{K}, \closure{K}'$ von $K$ sind $K$-isomorph.
\end{corollary}

Wegen dieser Eindeutigkeit bis auf $K$-Isomorphismus spricht man auch von \emph{dem} algebraischen Abschluss von $K$, und notiert diesen mit $\closure{K}$.





\section{Zerfällungskörper}

Es sei $L/K$ eine Körpererweiterung.

\begin{definition}
  Es ist $L$ ein \emph{Zerfällungskörper} einer Familie $(f_i)_{i \in I}$ von Polynomen $f_i \in K[t]$, wenn jedes $f_i$ über $L$ in Linearfaktoren zerfällt, und die Erweiterung $L/K$ von den Nullstellen der $f_i$ erzeugt wird.
\end{definition}

Ist $L/K$ ein Zerfällungskörper, so wird die Erweiterung $L/K$ von algebraischen Elementen (nämlich den Nullstellen der $f_i$) erzeugt, und ist somit algebraisch.

\begin{lemma}[Existenz von Zerfällungskörpern]
  Für jede Familie $(f_i)_{i \in I}$ nicht-konstanter Polynome $f_i \in K[t]$ existiert ein Zerfällungskörper.
\end{lemma}

\begin{lemma}
  Es sei $(f_i)_{i \in I}$ eine Familie von Polynomen $f_i \in K[t]$.
  Es sei $\varphi \colon K \to K'$ ein Körperisomorphismus und $\varphi_* \colon K[t] \to K'[t]$, $\sum_{i=0}^n a_i t^i \mapsto \sum_{i=0}^n \varphi(a_i) t^i$ der induzierte Ringisomorphismus.
  Ist $L$ ein Zerfällungkörper der Familie $(f_i)_{i \in I}$ und $L'$ ein Zerfällungskörper der entsprechenden Familie $(\varphi_*(f_i))_{i \in I}$, so setzt sich $\varphi$ zu einem Körperisomorphismus $\psi \colon L \to L'$ fort, d.h.\ es gilt $\restrict{\psi}{K} = \varphi$.
  \[
    \begin{tikzcd}
      L
        \arrow[dashed]{r}[above]{\psi}
      & L'
      \\
        K
        \arrow[left hook ->,]{u}
        \arrow{r}[above]{\varphi}
      & K'
        \arrow[left hook ->]{u}
    \end{tikzcd}
  \]
\end{lemma}

\begin{corollary}[Eindeutigkeit von Zerfällungskörpern]
  Je zwei Zerfällungskörper $L, L'$ einer Familie $(f_i)_{i \in I}$ von Polynomen $f_i \in K[t]$ sind $K$-isomorph.
\end{corollary}

Ist $L$ ein Zerfällungskörper einer Familie von Polynom $(f_i)_{i \in I}$, so wird die Erweiterung $L/K$ von den Nullstellen der $f_i$ erzeugt.
Ist $L'/K$ eine weitere Körpererweiterung und $\varphi \colon L \to L'$ ein $K$-Homomorphismus, so ist für jede Nullstelle $a \in L$ von $f_i$ auch $\varphi(a) \in L'$ eine Nullstelle von $f_i$.
Für die Menge $N = \{ a_1, \dotsc, a_n \}$ aller Nullstellen von $f_i$ in $L$ ist deshalb das Bild $\varphi(N)$ die Menge aller Nullstellen von $f_i$ in $L'$.
Deshalb ist $\varphi(N)$ unabhängig von der Wahl von $\varphi$, d.h.\ für jeden weiteren $K$-Homomorphismus $\psi \colon L \to L'$ gilt $\varphi(N) = \psi(N)$.
Da $L$ von den Nullstellen der $f_i$ erzeugt wird, ist somit das gesamte Bild $\varphi(L)$ unabhängig von der Wahl von $\varphi$.

\begin{example}
  Für jeden $\Real$-Homomorphismus $\varphi \colon \Complex \to \Complex$ gilt $\varphi(\{i, -i\}) = \{i, -i\}$.
\end{example}

\begin{proposition}
  Ist $L/K$ algebraisch, so sind die folgenden Bedingungen äquivalent:
  \begin{enumerate}
    \item
      Es ist $L$ der Zerfällungskörper einer Familie $(f_i)_{i \in I}$ von Polynomen $f_i \in K[t]$.
    \item
      Für jede Körpererweiterung $L'/K$ haben alle $K$-Homomorphismen $L \to L'$ das gleiche Bild.
    \item
      Für jede algebraische Körpererweiterung $L'/K$ haben alle $K$-Homomorphismen $L \to L'$ das gleiche Bild.
    \item
      Ist $\closure{K}$ ein algebraischer Abschluss von $K$, so haben alle $K$-Homomorphismen $L \to \closure{K}$ das gleiche Bild.
    \item
      Ist $\closure{K}$ ein algebraischer Abschluss von $K$ mit $L \subseteq \closure{K}$, so hat jeder $K$-Homo\-mor\-phis\-mus $L \to \closure{K}$ das Bild $\varphi(L) = L$.
    \item
      Ist $\closure{K}$ ein algebraischer Abschluss von $K$ mit $L \subseteq \closure{K}$, so schränkt sich jeder $K$-Homomorphismus $L \to \closure{K}$ zu einem $K$-Automorphismus $L \to L$ ein.
    \item
      Jedes irreduzible Polynom $f \in K[t]$, das in $L$ eine Nullstelle hat, zerfällt in $L$ bereits in Linearfaktoren.
  \end{enumerate}
\end{proposition}

\begin{definition}
  Erfüllt die Erweiterung $L/K$ eine \textup(und damit alle\textup) der obigen Bedingungen, so ist die Erweiterung $L/K$ \emph{normal}.
\end{definition}

\begin{example}
  Jede Körpererweiterung $L/K$ vom Grad $\fieldindex{L}{K} = 2$ ist normal.
\end{example}

\begin{lemma}
  Sind $M/L/K$ Körpererweiterungen, so dass $M/K$ normal ist, so ist auch $M/L$ normal.
\end{lemma}

\begin{warning}
  \begin{enumerate}
    \item
      Ist $M/K$ normal, so ist $L/K$ nicht notwendigerweise normal.
    \item
      Sind $M/L$ und $L/K$ normal, so ist $M/K$ nicht notwendigerweise normal.
  \end{enumerate}
\end{warning}










\pagebreak










\section{Separablität}

Es sei $K$ ein Körper.

\begin{definition}
  Die \emph{\textup(formale\textup) Ableitung} eines Polynoms $f = \sum_{i=0}^n a_i t^i \in K[t]$ ist das Polynom
  \[
              f'
    \defined  \sum_{i=1}^n i a_i t^{i-1}
    =         \sum_{i=0}^{n-1} (i+1) a_{i+1} t^i \,.
  \]
\end{definition}

\begin{lemma}
  Für alle $f, g \in K[t]$, $\lambda \in K$ gelten die Gleichheiten.
  \[
    (f + g)' = f' + g' \,,
    \qquad
    (\lambda f)' = \lambda f' \,,
    \qquad
    (fg)' = fg' + f'g \,.
  \]

\end{lemma}



\subsection{Mehrfache Nullstellen}

\begin{lemma}
  Ein Element $a \in K$ ist genau dann eine mehrfache Nullstelle eines Polynoms $f \in K[t]$, wenn $a$ eine gemeinsame Nullstelle von $f$ und $f'$ ist.
\end{lemma}

Ist $K$ algebraisch abgeschlossen, so besitzen je zwei Polynome $f, g \in K[t]$ genau dann eine gemeinsame Nullstelle, wenn sie einen gemeinsamen Linearfaktoren besitzten.
Da die Linearfaktoren ein Repräsentantensystem der irreduziblen Elemente von $K[t]$ bilden, ist dies äquivalent dazu, dass $f$ und $g$ nicht teilerfremd sind, dass also $\ggT(f,g) \neq 1$ gilt.
Da sich der $\ggT$ zweier Polynome mithilfe des euklidischen Algorithmus und Polynomdivison berechen lässt, hängt dieser $\ggT$ dabei nicht von dem Körper $K$ ab:

\begin{lemma}
  Ist $L/K$ eine Körperweiterung, so haben $f, g \in K[t]$ den gleichen $\ggT$ in $K[t]$ und $L[t]$, d.h.\ ein Polynom $h \in K[t]$ ist genau dann ein $\ggT$ von $f$ und $g$ in $K[t]$, wenn $h$ in $L[t]$ ein $\ggT$ von $f$ und $g$ ist.
  Inbesondere sind $f$ und $g$ genau dann teilerfremd in $K[t]$, wenn sie es in $L[t]$ sind.
\end{lemma}

Dies führt zu der folgenden Charakterisierung gemeinsamer Nullstellen:

\begin{corollary}
  Für $f, g \in K[t]$ sind die folgenden Bedingungen äquivalent:
  \begin{enumerate}
    \item
      Ist $\closure{K}$ ein algebraischer Abschluss von $K$, so haben $f$ und $g$ eine gemeinsame Nullstelle in $K$.
    \item
      Es gibt eine Körpererweiterung $L/K$, so dass $f$ und $g$ eine gemeinsame Nullstelle in $L$ haben.
    \item
      Es gibt eine algebraische Körpererweiterung $L/K$, so dass $f$ und $g$ eine gemeinsame Nullstelle in $L$ haben.
    \item
      Die Polynome $f$ und $g$ sind nicht teilerfremd.
  \end{enumerate}
\end{corollary}

\begin{corollary}
  Für ein Polynom $f \in K[t]$ sind die folgenden Bedingungen äquivalent:
  \begin{enumerate}
    \item
      Für keine Körpererweiterung $L/K$ hat $f$ eine mehrfache Nullstelle in $L$.
    \item
      Das Polynom $f$ hat in einem algebraischen Abschluss $\closure{K}$ keine mehrfache Nullstelle.
    \item
      Das Polynom $f$ hat in einem Zerfällungkörper von $f$ keine mehrfache Nullstelle.
    \item
      Die Polynome $f$ und $f'$ sind teilerfremd.
  \end{enumerate}
\end{corollary}

\begin{definition}
  Ein Polynom $f \in K[t]$, das eine \textup(und damit alle\textup) der obigen Bedingungen erfüllt, ist \emph{separabel}.
\end{definition}

\begin{example}
  \begin{enumerate}
    \item
      Gilt $\ringchar{K} = 0$, so ist jedes irreduzible Polynom $f \in K[t]$ separabel:
      Es gilt $\deg(f') = \deg(f)-1$, weshalb $f$ kein Teiler von $f'$ ist, und somit $f$ und $f'$ teilerfremd sind.
    \item
      Das Polynom $f \defined t^p - u \in \Finite_p(u)[t]$ ist nicht separabel, denn es gilt $f' = 0$, und somit $\ggT(f,f') = f \neq 1$.
    \item
      Gilt allgemeiner $\ringchar{K} = p > 0$ und ist $f \in K[t]$ irreduzibel, so gilt
      \[
              \text{$f$ ist nicht separabel}
        \iff  f' = 0
        \iff  \exists g \in K[t] : f(t) = g(t^p) \,.
      \]

  \end{enumerate}
\end{example}



\subsection{Separable Elemente und Erweiterungen}

Es sei $L/K$ ein Körpererweiterung.

\begin{lemma}
  Für jedes Element $a \in L$ sind die folgenden Bedingungen äquivalent:
  \begin{enumerate}
    \item
      Es gibt ein separables Polynom $f \in K[t]$ mit $f(a) = 0$.
    \item
      Das Element $a$ ist algebraisch, und das Minimalpolynom $m_a \in K[t]$ ist separabel.
  \end{enumerate}
\end{lemma}

\begin{definition}
  Ein Element $a \in L$ ist \emph{separabel \textup(über $K$\textup)}, wenn es eine \textup(und damit alle\textup) der obigen Bedingungen erfüllt;
  ansonsten ist $a$ \emph{inseparabel}.
  
  Die Erweiterung $L/K$ ist \emph{separabel}, wenn jedes Element $a \in L$ separabel über $K$ ist;
  ansonsten ist die Erweiterung \emph{inseparabel}.
\end{definition}

\begin{example}
  Gilt $\ringchar{K} = 0$, so ist jedes algebraische Element $a \in L$ separabel, und somit jede algebraische Körpererweiterung $L/K$ separabel.
\end{example}

\begin{lemma}
  Sind $M/L/K$ Erweiterungen, so dass $M/K$ separabel ist, so sind auch $M/L$ und $L/K$ separabel.
\end{lemma}



\subsection{Der Separablitätsgrad}

\begin{definition}
  Der \emph{Separablitätsgrad} der algebraischen Erweiterung $L/K$ ist die Anzahl der $K$-Homomorphismen $L \to \closure{K}$, und wird mit $\sepindex{L}{K}$ notiert.
\end{definition}

\begin{lemma}[Multiplikativität des Separablitätsgrades]
  Für alle algebraischen Körpererweiterungen $M/L/K$ gilt $\sepindex{M}{K} = \sepindex{M}{L} \sepindex{L}{K}$.
\end{lemma}

\noindent
\begin{minipage}[t]{\textwidth}
\begin{lemma}
  Es sei $a \in L$ algebraisch.
  \begin{enumerate}
    \item
      Es gilt $\sepindex{K(a)}{K} \leq \fieldindex{K(a)}{K}$.
    \item
      Es gilt genau dann Gleichheit, wenn $K(a)/K$ separabel ist.
  \end{enumerate}
\end{lemma}
\end{minipage}

\begin{corollary}
  Die Erweiterung $L/K$ sei endlich.
  \begin{enumerate}
    \item
      Es gilt $\sepindex{L}{K} \leq \fieldindex{L}{K}$.
    \item
      Es gilt genau dann Gleichheit, wenn $L/K$ separabel ist.
  \end{enumerate}
\end{corollary}

\begin{corollary}
  Sind $M/L$ und $L/K$ endliche separable Körpererweiterungen, so ist auch $M/K$ separabel.
\end{corollary}

\begin{corollary}
  Sind $a, b \in L$ separabel, so sind auch $a + b$ und $a \cdot b$ separabel;
  gilt $a \neq 0$, so ist auch $1/a$ separabel.
\end{corollary}

\begin{corollary}
  Es ist $L' = \{x \in L \suchthat \text{$x$ ist separabel}\}$ ein Zwischenkörper von $L/K$.
\end{corollary}

\begin{corollary}
  Die Erweiterung $L/K$ ist genau dann separabel, wenn sie von separablen Elementen erzeugt wird.
\end{corollary}

\begin{theorem}[Satz vom primitiven Element]
  Ist $L/K$ endlich und separabel, so gibt es ein $a \in L$ mit $L = K(a)$, d.h.\ die Erweiterung $L/K$ ist einfach.
\end{theorem}



\subsection{Perfekte Körper}

\begin{definition}
  Ein Körper $K$ ist \emph{perfekt} oder \emph{vollkommen}, wenn jede algebraische Körpererweiterung $L/K$ separabel ist.
\end{definition}

\begin{example}
  Jeder Körper von Charakteristik $0$ ist perfekt.
\end{example}

Ob ein Körper $K$ der Charakteristik $\ringchar{K} = p > 0$ perfekt ist, hängt vom Verhalten des \emph{Frobenius-Homomorphismus} ab:

\begin{lemma}
  Ist $R$ ein Ring mit $\ringchar{R} = p > 0$, so ist die Abbildung $\sigma \colon R \to R$, $x \mapsto x^p$ ein Ringhomomorphismus.
\end{lemma}

\begin{definition}
  Der Ringhomomorphismus $\sigma$ wie oben ist der \emph{Frobenius-Homo\-mor\-phis\-mus} von $R$.
\end{definition}

\begin{proposition}
  Ein Körper $K$ von Charakteristik $\ringchar{K} = p > 0$ ist genau dann perfekt, wenn der Frobenius-Homomorphismus $\sigma \colon K \to K$, $x \mapsto x^p$ surjektiv ist.
\end{proposition}

\begin{example}
  \begin{enumerate}
    \item
      Endliche Körper sind perfekt.
    \item
      Algebraisch abgeschlossene Körper sind perfekt.
    \item
      Der Körper $\Finite_p(u)$ ist nicht perfekt, denn es gibt kein $a \in \Finite_p(u)$ mit $a^p = u$.
  \end{enumerate}
\end{example}




\section{Klassifikation endlicher Körper}

Es sei $p$ eine Primzahl.
Im Folgenden seien alle Körper von Charakteristik~$p$;
der zugehörige Primkörper ist also stets $\Finite_p$.

\begin{lemma}
  Ist $K$ ein endlicher Körper, so gilt $\card{\Finite_p} = p^n$ für $n = \fieldindex{K}{\Finite_p}$.
\end{lemma}

Es sei $K$ ein Körper mit $\card{K} = q$.
Dann gilt $\card{\unitgroup{K}} = q-1$, und somit $x^{q-1} = 1$ für alle $x \in \unitgroup{K}$.
Für alle $x \in K$ gilt somit $x^q = x$.

\begin{lemma}
  Ist $K$ ein endlicher Körper mit $\card{K} = p^n = q$, so gilt besteht $K$ aus den $q$ verschiedenen Nullstellen des Polynomes $t^q - t \in \Finite_p[t]$.
  Inbesondere ist $K$ ein Zerfällungskörper des Polynoms $t^q - t$.
\end{lemma}

\begin{corollary}
  Je zwei endliche Körper $K$, $K'$ mit $\card{K} = p^n = \card{K'}$ sind isomorph.
\end{corollary}

Es sei andererseits $n \geq 1$, $q \defined p^n$ und $K$ ein Zerfällungskörper des Polynoms $f \defined t^q - t \in \Finite_p[t]$.
Dann ist die Menge der Nullstellen $K' \defined \{x \in K \suchthat x^q = x \}$ ein Unterkörper von $K$, denn es ist $K' = \{x \in K \suchthat \sigma^n(x) = \id(x)\}$ die Übereinstimmungsmenge zweier Körperhomomorphismen.
Es gilt somit $K = K'$.
Es gilt $f' = -1$, weshalb $f$ und $f'$ teilerfremd sind.
Es ist also $f$ separabel, und somit $\card{K} = \deg(f) = q = p^n$.

\begin{theorem}[Klassifikation endlicher Körper]
  Es sei $p$ eine Primzahl.
  Dann gibt es für alle $n \geq 0$ einen Körper $\Finite_{p^n}$ mit $p^n$ Elementen, dieser ist eindeutig bis auf Isomorphie, und jeder endliche Körper von Charakteristik $p$ ist von dieser Form.
  Der Körper $\Finite_{p^n}$ besteht aus den Nullstellen des Polynoms $tü^{p^n} - t \in \Finite_p[t]$.
\end{theorem}

\begin{lemma}
  Für alle $n, m \geq 1$ gibt es genau dann eine Einbettung \textup(d.h.\ einen Körperhomomorphismus\textup) $\Finite_{p^n} \to \Finite_{p^m}$, wenn $n \divides m$ gilt.
\end{lemma}

\begin{example}
  Für alle $n \leq m$ lässt sich $\Finite_{p^{n!}}$ als Unterkörper von $\Finite_{p^{m!}}$ auffassen.
  Dann ist $\Finite_{p^\infty} \defined \bigcup_{n \geq 0} \Finite_{p^{n!}}$ ein algebraischer Abschluss von $\Finite_p$.
\end{example}






