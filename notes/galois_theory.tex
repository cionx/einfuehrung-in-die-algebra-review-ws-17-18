\chapter{Galois-Theorie}





\section{Galois-Erweiterungen}

Es sei $L/K$ eine endliche Körpererweiterung.

\begin{definition}
  Der \emph{Fixkörper} einer Untergruppe $H \subgroup \Aut{L/K}$ ist
  \[
              L^H
    \defined  \{
                x \in L
              \suchthat
                \text{$\sigma(x) = x$ für alle $\sigma \in H$}
              \} \,.
  \]

\end{definition}

\begin{lemma}
  \begin{enumerate}
    \item
      Es gilt $\card{\Aut{L/K}} \leq \sepindex{L}{K}$.
    \item
      Ist $L/K$ normal, so gilt bereits Gleichheit.
  \end{enumerate}
\end{lemma}


\begin{proposition}
  Für $L/K$ sind die folgenden Bedingungen äquivalent:
  \begin{enumerate}
    \item
      Es gilt $\card{\Aut{L/K}} = \fieldindex{L}{K}$.
    \item
      Die Erweiterung $L/K$ ist normal und separabel.
    \item
      Für jedes $a \in L$ gilt $m_a \defined \prod_{\sigma \in \Aut{L/K}} (t - \sigma(a)) \in K[t]$.
    \item
      Es gilt $K = L^{\Aut{L/K}}$.
  \end{enumerate}
\end{proposition}

\begin{definition}
  Erfüllt die Erweiterung $L/K$ eine \textup(und damit alle\textup) der obigen Bedingungen, so ist $L/K$ \emph{galoissch}.
  Es ist dann $\Gal{L/K} \defined \Aut{L/K}$ die \emph{Galois-Gruppe} der Erweiterung $L/K$.
\end{definition}

\begin{theorem}[Hauptsatz der Galoistheorie]
  Die Erweiterung $L/K$ sei galoissch.
  \begin{enumerate}
    \item
      Es gibt es eine inklusionsumkehrende Bijektion
      \begin{align*}
                                \{ \text{Zwischenkörper $K \subseteq M \subseteq L$} \}
        &\xlongrightarrow{\sim} \{ \text{Untergruppen $H \subgroup \Gal{L/K}$} \} \,, \\
                                M
        &\longmapsto            \Aut{L/M} \,, \\
                                L^H
        &\longmapsfrom          H \,.
      \end{align*}
    \item
      Dabei gelten $\fieldindex{L}{M} = \card{\Aut{L/M}}$ und $\fieldindex{M}{K} = \groupindex{Aut{L/M}}{\Gal{L/K}}$.
    \item
      Die Erweiterung $M/K$ ist genau dann normal, wenn die zugehörige Untergruppe $\Aut{L/M} \subgroup \Gal{L/K}$ normal ist.
      Dann ist auch $M/K$ galoissch, und es gilt $\Gal{M/K} \cong \Gal{L/K} / {\Aut{L/M}}$.
  \end{enumerate}
\end{theorem}
