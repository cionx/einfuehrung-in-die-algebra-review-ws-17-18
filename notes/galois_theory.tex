\chapter{Galois-Theorie}





\section{Galois-Erweiterungen}

Es sei $L/K$ eine endliche Körpererweiterung.

\begin{definition}
  Der \emph{Fixkörper} einer Untergruppe $H \subgroup \Aut{L/K}$ ist
  \[
              L^H
    \defined  \{
                x \in L
              \suchthat
                \text{$\sigma(x) = x$ für alle $\sigma \in H$}
              \} \,.
  \]

\end{definition}

\begin{lemma}
  \begin{enumerate}
    \item
      Es gilt $\card{\Aut{L/K}} \leq \sepindex{L}{K} \leq \fieldindex{L}{K}$.
    \item
      Die Eweiterung $L/K$ ist genau dann normal, wenn $\card{\Aut{L/K}} = \sepindex{L}{K}$ gilt.
  \end{enumerate}
\end{lemma}


\begin{proposition}
  Für $L/K$ sind die folgenden Bedingungen äquivalent:
  \begin{enumerate}
    \item
      Es gilt $\card{\Aut{L/K}} = \fieldindex{L}{K}$.
    \item
      Die Erweiterung $L/K$ ist normal und separabel.
    \item
      Für jedes $a \in L$ zerfällt das Minimalpolynom $m_a$ über $L$ in die Linearfaktoren $t - \sigma(a)$ mit $\sigma \in \Gal{L/K}$, jeweils mit Vielfachheit $1$.
    \item
      Es gilt $K = L^{\Aut{L/K}}$.
  \end{enumerate}
\end{proposition}

\begin{definition}
  Erfüllt die Erweiterung $L/K$ eine \textup(und damit alle\textup) der obigen Bedingungen, so ist $L/K$ \emph{galoissch}.
  Es ist dann $\Gal{L/K} \defined \Aut{L/K}$ die \emph{Galoisgruppe} der Erweiterung $L/K$.
\end{definition}

\begin{theorem}[Hauptsatz der Galoistheorie]
  Die Erweiterung $L/K$ sei galoissch mit Galoisgruppe $G \defined \Gal{L/K}$.
  \begin{enumerate}
    \item
      Es gibt es eine inklusionsumkehrende Bijektion
      \begin{align*}
                                \{ \text{Zwischenkörper $K \subseteq M \subseteq L$} \}
        &\xlongrightarrow{\sim} \{ \text{Untergruppen $H \subgroup G$} \} \,, \\
                                M
        &\longmapsto            \Aut{L/M} \,, \\
                                L^H
        &\longmapsfrom          H \,.
      \end{align*}
      Dies lässt sich wie folgt veranschaulichen:
      \[
        \begin{tikzcd}
            L
          \\
            M
            \arrow{u}
          \\
            K
            \arrow{u}
        \end{tikzcd}
        \qquad
        \leftrightsquigarrow
        \qquad
        \begin{tikzcd}
            1
            \arrow{d}
          \\
            H
            \arrow{d}
          \\
            G
        \end{tikzcd}
      \]
    \item
      Es sei $K \subseteq M \subseteq L$ ein Zwischenkörper mit zugehöriger Untergruppe $H \subgroup G$.
      \begin{enumerate}
        \item
          Es ist $L/M$ galoissch mit Galoisgrouppe $H$.
          Insbesondere gilt $\fieldindex{L}{M} = \card{H}$.
        \item
          Es gilt $\fieldindex{M}{K} = \groupindex{G}{H}$.
      \end{enumerate}
      Dies lässt sich wie folgt veranschaulichen:
      \[
        \begin{tikzcd}[row sep = large]
            L
          \\
            M
            \arrow{u}[left]{\fieldindex{L}{M}}
          \\
            K
            \arrow{u}[left]{\fieldindex{M}{K}}
        \end{tikzcd}
        \qquad
        \leftrightsquigarrow
        \qquad
        \begin{tikzcd}[row sep = large]
            1
            \arrow{d}[right]{\card{H} = \groupindex{H}{1}}
          \\
            H
            \arrow{d}[right]{\groupindex{G}{H}}
          \\
            G
        \end{tikzcd}
      \]
    \item
      Die Erweiterung $M/K$ ist genau dann normal, wenn die zugehörige Untergruppe $N \subgroup G$ normal ist.
      Dann ist $M/K$ ebenfalls galoissch, und es gilt
      \[
              \Gal{M/K}
        \cong G/N \,.
      \]
  \end{enumerate}
\end{theorem}

\begin{definition}
  Die obige inklusionsumkehrende Korrespondenz zwischen Unterkörpern und Untergruppen ist die \emph{Galoiskorrespondenz}.
\end{definition}





\section{Beispiel: \texorpdfstring{$\Complex/\Real$}{C/R}}

Die Körpererweiterung $\Complex/\Real$ ist normal, da $\Complex = \Real(i)$ ein Zerfällungskörper des Polynoms $f \defined t^2 + 1 \in \Real[t]$ ist.
Die Erweiterung ist separabel, da $\Real$ perfekt ist, da $\ringchar{\Real} = 0$ gilt.
Somit ist die Erweiterung $\Complex/\Real$ galoissch.

Aus $\card{\Gal{\Complex/\Real}} = \fieldindex{\Complex}{\Real} = 2$ ergibt sich bereits, dass $\Gal{\Complex/\Real} \cong \Integer/2$ gilt.
Ein nicht-trivialer $\Real$-Automorphismus $\Complex \to \Complex$ ist durch die Konjugationsabbildung $c \colon \Complex \to \Complex$, $z \mapsto \conjugate{z}$ gegeben.
Somit gilt $\Gal{\Complex/\Real} = \{\id, c\}$.

\[
  \begin{tikzcd}
      \{\id, c\}
    \\
      1
      \arrow{u}[left]{2}
  \end{tikzcd}
  \qquad
  \leftrightsquigarrow
  \qquad
  \begin{tikzcd}
      \Real
      \arrow{d}[left]{2}
    \\
      \Complex
  \end{tikzcd}
\]





\section{Beispiel: \texorpdfstring{$\Finite_{p^n}/\Finite_p$}{F\_(pn)/F\_p}}

Die Erweiterung $\Finite_{p^n}/\Finite_p$ ist normal, denn nach der Klassifikation endlicher Körper ist $\Finite_{p^n}$ ein Zerfällungskörper des Polynoms $f \defined t^{p^n} - t \in \Finite_p[t]$.
Die Erweiterung ist separabel, da $\Finite_p$ endlich, und somit perfekt ist.
Also ist $\Finite_{p^n}/\Finite_p$ galoissch.

Der Frobenius-Homomorphismus $\sigma \colon \Finite_{p^n} \to \Finite_{p^n}$, $x \mapsto x^p$ ist ein Körperautomorphismus von $\Finite_{p^n}$;
er ist $\Finite_p$-linear, da $\Finite_p$ der Primkörper von $\Finite_{p^n}$ ist.
Somit gilt $\sigma \in \Gal{\Finite_{p^n}/\Finite_p}$.

Gilt dabei $\sigma^m = \id$ für ein $m \geq 1$, so gilt $x^{p^m} - x = 0$ für alle $x \in \Finite_{p^n}$, d.h.\ alle Elemente von $\Finite_{p^n}$ sind Nullstellen des Polynoms $g \defined t^{p^m} - t \in \Finite_p[t]$.
Dann gilt allerdings $p^n = \card{\Finite_{p^n}} \leq \deg(g) = p^m$, und somit $m \geq n$.
Es gilt also $\ord{\sigma} \geq n$.
Andererseits gilt
\[
            \ord{\sigma}
  \divides  \ord{ \Gal{\Finite_{p^n}/\Finite_p} }
  =         \fieldindex{\Finite_{p^n}}{\Finite_p}
  =         n \,.
\]

Ingesamt gilt also $\ord{\sigma} = n = \ord{ \Gal{\Finite_{p^n}/\Finite_p} }$.
Somit ist die Galoisgruppe $\Gal{\Finite_{p^n}/\Finite_p}$ zyklisch, und wird vom Frobenius-Homomorphismus $\sigma$ erzeugt.
Es gilt insbesondere $\Gal{\Finite_{p^n}/\Finite_p} \cong \Integer/n$.

Für $n = 6$ ergibt sich die folgende Galoiskorrespondenz:

\[
  \begin{tikzcd}[column sep = tiny]
      {}
    & \{1, \sigma, \dotsc, \sigma^5\}
    & {}
    \\
      \{1, \sigma^2, \sigma^4\}
      \arrow{ur}[above left]{2}
    & {}
    & \{1, \sigma^3\}
      \arrow{ul}[above right]{3}
    \\
      {}
    & 1
      \arrow{lu}[below left]{3}
      \arrow{ru}[below right]{2}
    & {}
  \end{tikzcd}
  \qquad\leftrightsquigarrow\qquad
  \begin{tikzcd}[column sep = tiny]
      {}
    & \Finite_p
      \arrow{dl}[above left]{2}
      \arrow{dr}[above right]{3}
    & {}
    \\
      \Finite_{p^2}
      \arrow{dr}[below left]{3}
    & {}
    & \Finite_{p^3}
      \arrow{dl}[below right]{2}
    \\
      {}
    & \Finite_{p^6}
    & {}
  \end{tikzcd}
\]





\section{Beispiel: \texorpdfstring{$\Rational(\sqrt{2}, \sqrt{3})/\Rational$}{Q(sqrt(2),sqrt(3))/Q}}

\subsection{Die Erweiterung ist galoissch}

Es sei $L \defined \Rational(\sqrt{2}, \sqrt{3})$.
Dann ist $L = \Rational(\sqrt{2}, -\sqrt{2}, \sqrt{3}, -\sqrt{3})$ ein Zerfällungskörper des Polynoms $f \defined (t^2 - 2)(t^2 - 3) \in \Rational[t]$, und die Erweiterung $L/\Rational$ somit normal.
Diese Erweiterung ist auch separabel, da $\Rational$ perfekt ist, da $\ringchar{\Rational} = 0$ gilt.

\subsection{Bestimmung des Grades}

Es gilt $\fieldindex{L}{\Rational} = \fieldindex{\Rational(\sqrt{2})(\sqrt{3})}{\Rational(\sqrt{2})} \fieldindex{\Rational(\sqrt{2})}{\Rational}$.
Dabei gilt $\fieldindex{\Rational(\sqrt{2})}{\Rational} = 2$, denn das Minimalpolynom von $\sqrt{2}$ ist $t^2 - 2 \in \Rational[t]$ (siehe Beispiel~\ref{example: minimal polynomials}).
Da $\sqrt{3}$ Nullstelle des Polynoms $t^2 - 3 \in \Rational(\sqrt{2})[t]$ ist, gilt $\fieldindex{\Rational(\sqrt{2})(\sqrt{3})}{\Rational(\sqrt{2})} \leq 2$.
Im Fall $\fieldindex{\Rational(\sqrt{2})(\sqrt{3})}{\Rational(\sqrt{2})} = 1$ wäre $\sqrt{3} \in \Rational(\sqrt{2})$, was aber nicht gilt:

\begin{lemma}
  Es gilt $\sqrt{3} \notin \Rational(\sqrt{2})$.
\end{lemma}

Es gilt also $\fieldindex{\Rational(\sqrt{2})(\sqrt{3})}{\Rational(\sqrt{2})} = 2$.
Insgesamt gilt somit $\fieldindex{L}{\Rational} = 4$.

\subsection{Bestimmung einer Basis}

Es ist $1, \sqrt{2}$ eine $\Rational$-Basis von $\Rational(\sqrt{2})$, und $1, \sqrt{3}$ eine $\Rational(\sqrt{2})$-Basis von $\Rational(\sqrt{2})(\sqrt{3}) = L$.
Eine $\Rational$-Basis von $L$ ist deshalb durch
\[
  1         \cdot 1         = 1 \,,
  \quad
  \sqrt{2}  \cdot 1         = \sqrt{2}  \,,
  \quad
  1         \cdot \sqrt{3}  = \sqrt{3}  \,,
  \quad
  \sqrt{2}  \cdot \sqrt{3}  = \sqrt{6}
\]
gegeben.

\subsection{Bestimmung der Galoisgruppe}

Jeder $\Rational$-Automorphismus $\varphi \colon L \to L$ muss die Nullstellen der beiden rationalen Polynome $t^2 - 2, t^2 - 3 \in \Rational[t]$ jeweils permutieren;
es muss also $\varphi(\sqrt{2}) = \pm \sqrt{2}$ und $\varphi(\sqrt{3}) = \pm \sqrt{3}$ gelten.
Dabei ist $\varphi$ durch die Werte $\varphi(\sqrt{2})$ und $\varphi(\sqrt{3})$ bereits eindeutig bestimmt.
Es gibt also höchstens $4$ $\Rational$-Automorphismen $L \to L$, und diese sind durch die folgenden Zuordnungen festgelegt:
\[
  \arraycolsep = 2em
  \renewcommand*{\arraystretch}{3}
  \begin{array}{cc}
  \id
  \colon
  \left\{
    \begin{aligned}
      \sqrt{2}  &\mapsto  \phantom{-}\sqrt{2}  \\
      \sqrt{3}  &\mapsto  \phantom{-}\sqrt{3}
    \end{aligned}
  \right.
  &
  \tau_1
  \colon
  \left\{
    \begin{aligned}
      \sqrt{2}  &\mapsto            -\sqrt{2}  \\
      \sqrt{3}  &\mapsto  \phantom{-}\sqrt{3}
    \end{aligned}
  \right.
  \\
  \tau_2
  \colon
  \left\{
    \begin{aligned}
      \sqrt{2}  &\mapsto  \phantom{-}\sqrt{2} \\
      \sqrt{3}  &\mapsto            -\sqrt{3}
    \end{aligned}
  \right.
  &
  \tau_1 \tau_2
  \colon
  \left\{
    \begin{aligned}
      \sqrt{2}  &\mapsto  -\sqrt{2} \\
      \sqrt{3}  &\mapsto  -\sqrt{3}
    \end{aligned}
  \right.
  \end{array}
\]
Da die Erweiterung $L/\Rational$ galoissch ist, gilt dabei $\card{\Gal{L/K}} = \fieldindex{L}{\Rational} = 4$.
Somit muss jede der obigen $4$ Zuordnungen tätsächlich schon einen $\Rational$-Automorphismus definieren.

Insbesondere gilt $\Gal{L/\Rational} \cong \Integer/2 \times \Integer/2$;
dabei entspricht etwa $\tau_1$ dem Element $(1,0)$, und $\tau_2$ dem Element $(0,1)$.

\subsection{Bestimmung der Untergruppen der Galoisgruppe}

Die Untergruppen von $\Integer/2 \times \Integer/2$ sind
\begin{itemize}
  \item
    die triviale Gruppe $0$,
  \item
    die zyklischen Untergruppen $\generated{(1,0)}$, $\generated{(1,1)}$, $\generated{(0,1)}$,
  \item
    die gesamte Gruppe $\Integer/2 \times \Integer/2$.
\end{itemize}
Die Untergruppen von $G \defined \Gal{L/\Rational}$ sind also
\begin{itemize}
  \item
    die triviale Gruppe $1$,
  \item
    die zyklischen Untergruppen $\generated{\tau_1}$, $\generated{\tau_2}$, $\generated{\tau_1 \tau_2}$,
  \item
    die gesamte Gruppe $G$.
\end{itemize}

\subsection{Bestimmung der Fixkörper}

Für $x \in L$,
\[
  x = a + b\sqrt{2} + c\sqrt{3} + d\sqrt{6}
\]
mit $a, b, c, d \in \Rational$ gilt etwa
\[
  \tau_1(x) = a - b\sqrt{2} + c \sqrt{2} - d\sqrt{6} \,,
\]
und deshalb genau dann $\tau_1(x) = x$, wenn $b = d = 0$.
Also gilt
\[
    L^{\generated{\tau_1}}
  = \{
      a + c\sqrt{3}
    \suchthat
      a,c \in \Rational
    \}
  = \Rational(\sqrt{3}) \,.
\]
Analog ergibt sich, dass
\[
    L^{\generated{\tau_2}}
  = \Rational(\sqrt{2}) \,,
  \qquad
    L^{\generated{\tau_1 \tau_2}}
  = \Rational(\sqrt{6}) \,.
\]

\subsection{Die Galoiskorrespondenz}

Die gesamte Galoiskorrespondenz sieht somit wie folgt aus:

\[
  \begin{tikzcd}[column sep = small]
      {}
    & G
    & {}
    \\
      \generated{\tau_1}
      \arrow{ru}[above left]{2}
    & \generated{\tau_1 \tau_2}
      \arrow{u}{2}
    & \generated{\tau_2}
      \arrow{lu}[above right]{2}
    \\
      {}
    & 1
      \arrow{lu}[below left]{2}
      \arrow{u}{2}
      \arrow{ru}[below right]{2}
    & {}
  \end{tikzcd}
  \qquad\leftrightsquigarrow\qquad
  \begin{tikzcd}[column sep = small]
      {}
    & \Rational
      \arrow{dl}[above left]{2}
      \arrow{d}[left]{2}
      \arrow{dr}[above right]{2}
    & {}
    \\
      \Rational(\sqrt{3})
      \arrow{dr}[below left]{2}
    & \Rational(\sqrt{6})
      \arrow{d}[left]{2}
    & \Rational(\sqrt{2})
      \arrow{dl}[below right]{2}
    \\
      {}
    & \Rational(\sqrt{2},\sqrt{3})
    & {}
  \end{tikzcd}
\]





\section{Beispiel: Kreisteilungskörper \texorpdfstring{$\Rational(\zeta_n)/\Rational$}{Q(zeta\_n)/Q}}

\begin{definition}
  Für alle $n \geq 1$ ist $W_n \defined \{z \in \Complex \suchthat z^n = 1\}$ die Gruppen der $n$-ten Einheitswurzeln.
  Eine $n$-te Einheitswurzel $\zeta \in W_n$ ist \emph{primitiv}, wenn $W_n = \generated{\zeta}$ gilt.
\end{definition}

Ist $\zeta$ eine primitive $n$-te Einheitswurzeln, so ist die Abbildung
\[
          \Integer/n
  \to     W_n \,,
  \quad   \class{k}
  \mapsto \zeta^k
\]
ein Gruppenisomorphismus.
Dabei gilt
\begin{align*}
      &\, \text{$\zeta_k$ ist primitiv} \\
  \iff&\, \text{$\class{k} \in \Integer/n$ ist ein zyklischer Erzeuger} \\
  \iff&\, \text{$k$ und $n$ sind teilerfremd} \\
  \iff&\, \text{$\class{k}$ ist eine Einheit in $\Integer/n$} \,.
\end{align*}

Also ist $\zeta^k$ genau dann primitiv, wenn $\class{k}$ ein Erzeuger von $\Integer/n$ ist, wenn also $k$ und $n$ teilerfremd sind, wenn also $\class{k}$ eine Einheit in $\Integer/n$ ist.

\begin{definition}
  Für alle $n \geq 1$ ist $\Phi_n \defined \prod_{\text{$\zeta$ primitive $n$-te Einheitswurzel}} (t - \zeta)$ das \emph{$n$-te Kreisteilungspolynom}.
\end{definition}

\begin{theorem}
  Es sei $n \geq 1$.
  \begin{enumerate}
    \item
      Das Polynom $\Phi_n$ ist normiert.
    \item
      Es gilt $t^n - 1 = \prod_{d \divides n} \Phi_d$.
    \item
      Für $p$ prim gilt $\Phi_p = t^{p-1} + \dotsb + t + 1 = (t^p-1)/(t-1)$.
    \item
      Es gilt $\Phi_n \in \Integer[t]$.
    \item
      Das Polynom $\Phi_n$ ist irreduzibel.
  \end{enumerate}
\end{theorem}

\begin{corollary}
  $\Phi_n$ ist das Minimalpolynom jeder primitiven $n$-ten Einheitswurzel
\end{corollary}

\begin{remark}
  Für $p$ prim lässt sich die Irreduziblität von $\Phi_p$ dadurch zeigen, dass man auf
  \[
      \Phi_p(t+1)
    = \frac{(t+1)^p - 1}{t}
    = \sum_{k=1}^p \binom{p}{k} t^{k-1}
    = \sum_{k=0}^{p-1} \binom{p}{k+1} t^k
  \]
  das Eisenstein-Kriterium bezüglich der Primzahl $p$ anwendet.
\end{remark}

Es sei im Folgenden $\zeta_n$ eine primitive $n$-te Einheitswurzel.

\subsection{Die Erweiterung ist galoissch}

Es ist $\Rational(\zeta_n) = \Rational(\zeta_n^k \suchthat k \in \Integer)$ ein Zerfällungskörper von $t^n - 1$, und $\Rational(\zeta_n)/\Rational$ somit normal.
Die Erweiterung ist separabel, da $\Rational$ perfekt ist, da $\ringchar{\Rational} = 0$ gilt.

\subsection{Bestimmung der Galoisgruppe}

Es gilt
  \[
  \arraycolsep=1pt
  \begin{array}{rcl}
      \Gal{\Rational(\zeta_n)/\Rational}
    & =
    & \{ \text{$\Rational$-Automorphismen $\Rational(\zeta_n) \to \Rational(\zeta_n)$} \}
    \\
      {}
    & =
    & \{ \text{$\Rational$-Homomorphismen $\Rational(\zeta_n) \to \Rational(\zeta_n)$} \}
    \\
      {}
    & \leftrightarrow
    & \{ \text{Nullstellen von $\Phi_n$ in $\Rational(\zeta_n)$} \}
    \\
      {}
    & =
    & \{\text{primitive $n$-te Einheitswurzeln}\}
    \\
      {}
    & \leftrightarrow
    & \unitgroup{(\Integer/n)} \,,
    \\
  \end{array}
\]
wobei für jedes $\class{k} \in \unitgroup{(\Integer/n)}$ das zugehörige Gruppenelement $\psi_{\,\class{k}} \in \Gal{\Rational(\zeta_n)/\Rational}$ durch
\[
    \psi_{\,\class{k}}(\zeta_n)
  = \zeta_n^k
\]
eindeutig bestimmt ist.
Dabei gilt
\[
    \psi_{\,\class{k}} \psi_{\,\class{\ell}}
  = \psi_{\,\class{k} \cdot \class{\ell}}
\]
weshalb es sich bereits um einen Gruppenisomorphismus handelt.
Es gilt also
\[
        \Gal{\Rational(\zeta_n)/\Rational}
  \cong \unitgroup{(\Integer/n)} \,.
\]









