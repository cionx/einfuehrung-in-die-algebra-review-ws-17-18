\chapter{Galois-Theorie}





\section{Galois-Erweiterungen}

Es sei $L/K$ eine endliche Körpererweiterung.

\begin{definition}
  Der \emph{Fixkörper} einer Untergruppe $H \subgroup \Aut{L/K}$ ist
  \[
              L^H
    \defined  \{
                x \in L
              \suchthat
                \text{$\sigma(x) = x$ für alle $\sigma \in H$}
              \} \,.
  \]

\end{definition}

\begin{lemma}
  \begin{enumerate}
    \item
      Es gilt $\card{\Aut{L/K}} \leq \sepindex{L}{K} \leq \fieldindex{L}{K}$.
    \item
      Die Eweiterung $L/K$ ist genau dann normal, wenn $\card{\Aut{L/K}} = \sepindex{L}{K}$ gilt.
  \end{enumerate}
\end{lemma}


\begin{proposition}
  Für $L/K$ sind die folgenden Bedingungen äquivalent:
  \begin{enumerate}
    \item
      Es gilt $\card{\Aut{L/K}} = \fieldindex{L}{K}$.
    \item
      Die Erweiterung $L/K$ ist normal und separabel.
    \item
      Für jedes $a \in L$ gilt $m_a \defined \prod_{\sigma \in \Aut{L/K}} (t - \sigma(a)) \in K[t]$.
    \item
      Es gilt $K = L^{\Aut{L/K}}$.
  \end{enumerate}
\end{proposition}

\begin{definition}
  Erfüllt die Erweiterung $L/K$ eine \textup(und damit alle\textup) der obigen Bedingungen, so ist $L/K$ \emph{galoissch}.
  Es ist dann $\Gal{L/K} \defined \Aut{L/K}$ die \emph{Galoisgruppe} der Erweiterung $L/K$.
\end{definition}

\begin{theorem}[Hauptsatz der Galoistheorie]
  Die Erweiterung $L/K$ sei galoissch.
  \begin{enumerate}
    \item
      Es gibt es eine inklusionsumkehrende Bijektion
      \begin{align*}
                                \{ \text{Zwischenkörper $K \subseteq M \subseteq L$} \}
        &\xlongrightarrow{\sim} \{ \text{Untergruppen $H \subgroup \Gal{L/K}$} \} \,, \\
                                M
        &\longmapsto            \Aut{L/M} \,, \\
                                L^H
        &\longmapsfrom          H \,.
      \end{align*}
    \item
      Dabei gelten $\fieldindex{L}{M} = \card{\Aut{L/M}}$ und $\fieldindex{M}{K} = \groupindex{\Aut{L/M}}{\Gal{L/K}}$.
    \item
      Die Erweiterung $M/K$ ist genau dann normal, wenn die zugehörige Untergruppe $\Aut{L/M} \subgroup \Gal{L/K}$ normal ist.
      Dann ist auch $M/K$ galoissch, und es gilt $\Gal{M/K} \cong \Gal{L/K} / {\Aut{L/M}}$.
  \end{enumerate}
\end{theorem}





\section{Beispiel: \texorpdfstring{$\Complex/\Real$}{C/R}}

Die Körpererweiterung $\Complex/\Real$ ist normal, da $\Complex = \Real(i)$ ein Zerfällungskörper des Polynoms $f \defined t^2 + 1 \in \Real[t]$ ist.
Die Erweiterung ist separabel, da $\Real$ perfekt ist, da $\ringchar{\Real} = 0$ gilt.
Somit ist die Erweiterung $\Complex/\Real$ galoissch.

Aus $\card{\Gal{\Complex/\Real}} = \fieldindex{\Complex}{\Real} = 2$ ergibt sich bereits, dass $\Gal{\Complex/\Real} \cong \Integer/2$ gilt.
Ein nicht-trivialer $\Real$-Automorphismus $\Complex \to \Complex$ ist durch die Konjugationsabbildung $c \colon \Complex \to \Complex$, $z \mapsto \conjugate{z}$ gegeben.
Somit gilt $\Gal{\Complex/\Real} = \{\id, c\}$.

Ersetzt man $\Real$ durch $\Rational$, und $\Complex$ durch $\Rational(i)$, so ergibt sich, dass $\Rational(i)/\Rational$ ebenfalls galoissch ist, und dass die Galoisgruppe $\Gal{\Rational(i)/\Rational}$ ebenfalls $\id$ und der Konjugationsabbildung besteht.
Zur Bestimmung des Grades $\fieldindex{\Rational(i)}{\Rational} = 2$ bemerke man dabei, dass $f \in \Rational[t]$ bereits das Minimalpolynom von $i \in \Rational(i)$ ist:
Das Polynom $f$ ist normiert mit $f(i) = 0$, und es ist irreduzibel in $\Rational[t]$, da es vom Grad $\leq 3$ ist und keine Nullstelle in $\Rational$ besitzt.

\[
  \begin{tikzcd}
      \{\id, c\}
    \\
      1
      \arrow{u}[left]{2}
  \end{tikzcd}
  \qquad
  \leftrightsquigarrow
  \qquad
  \begin{tikzcd}
      \Real
      \arrow{d}[left]{2}
    \\
      \Complex
  \end{tikzcd}
\]





\section{Beispiel: \texorpdfstring{$\Finite_{p^n}/\Finite_p$}{F\_(pn)/F\_p}}

Die Erweiterung $\Finite_{p^n}/\Finite_p$ ist normal, da $\Finite_{p^n}/\Finite_p$ ein Zerfällungskörper des Polynoms $f \defined t^{p^n} - t \in \Finite_p[t]$ ist.
Da der endliche Körper $\Finite_p$ perfekt ist, ist die Erweiterung auch seperabel.
Somit ist $\Finite_{p^n}/\Finite_p$ galoissch.

Der Frobenius-Homomorphismus $\sigma \colon \Finite_{p^n} \to \Finite_{p^n}$, $x \mapsto x^p$ ist ein Körperautomorphismus von $\Finite_{p^n}$, und $\Finite_p$-linear, da $\Finite_p$ der Primkörper von $\Finite_{p^n}$ ist.
Somit gilt $\sigma \in \Finite_{p^n}$.

Gilt dabei $\sigma^m = \id$ für $m \geq 1$, so gilt $x^{p^m} - x = 0$ für alle $x \in \Finite{p^n}$, d.h.\ $\Finite{p^n}$ besteht aus Nullstellen des Polynoms $g \defined t^{p^m} - t \in \Finite_p[t]$.
Dann gilt allerdings $p^n = \card{\Finite_{p^n}} \leq \deg(g) = p^m$, und somit $n \geq m$.
Es gilt also $\ord{\sigma} \geq n$.

Andererseits gilt $\ord{\sigma} \divides \ord{ \Gal{\Finite_{p^n}/\Finite_p} } = \fieldindex{\Finite_{p^n}}{\Finite_p} = n$.

Ingesamt gilt also $\ord{\sigma} = n = \ord{ \Gal{\Finite_{p^n}/\Finite_p} }$.
Somit ist die Galoisgruppe $\Gal{\Finite_{p^n}/\Finite_p}$ zyklisch, und wird vom Frobenius-Homomorphismus $\sigma$ erzeugt.

Für $n = 6$ ergibt sich etwa das folgende Bild:

\[
  \begin{tikzcd}[column sep = tiny]
      {}
    & \{1, \sigma, \dotsc, \sigma^5\}
    & {}
    \\
      \{1, \sigma^2, \sigma^4\}
      \arrow{ur}[above left]{2}
    & {}
    & \{1, \sigma^3\}
      \arrow{ul}[above right]{3}
    \\
      {}
    & 1
      \arrow{lu}[below left]{3}
      \arrow{ru}[below right]{2}
    & {}
  \end{tikzcd}
  \qquad\leftrightsquigarrow\qquad
    \begin{tikzcd}[column sep = tiny]
      {}
    & \Finite_p
      \arrow{dl}[above left]{2}
      \arrow{dr}[above right]{3}
    & {}
    \\
      \Finite_{p^2}
      \arrow{dr}[below left]{3}
    & {}
    & \Finite_{p^3}
      \arrow{dl}[below right]{2}
    \\
      {}
    & \Finite_{p^6}
    & {}
  \end{tikzcd}
\]






\section{Beispiel: \texorpdfstring{$\Rational(\sqrt{2}, \sqrt{3})/\Rational$}{Q(sqrt(2),sqrt(3))/Q}}

Es sei $L \defined \Rational(\sqrt{2}, \sqrt{3})$.
Dann ist $L = \Rational(\sqrt{2}, -\sqrt{2}, \sqrt{3}, -\sqrt{3})$ ein Zerfällungskörper des Polynoms $f \defined (t^2 - 2)(t^2 - 3) \in \Rational[t]$, und die Erweiterung $L/\Rational$ somit normal.
Diese Erweiterung ist auch separabel, da $\Rational$ perfekt ist, da $\ringchar{\Rational} = 0$ gilt.
Die Erweiterung $L/\Rational$ ist also galoissch.

Es gilt $\fieldindex{L}{\Rational} = \fieldindex{\Rational(\sqrt{2})(\sqrt{3})}{\Rational(\sqrt{2})} \fieldindex{\Rational(\sqrt{2})}{\Rational}$.
Dabei gilt $\fieldindex{\Rational(\sqrt{2})}{\Rational} = 2$, denn das Minimalpolynom von $\sqrt{2}$ ist $t^2 - 2 \in \Rational[t]$ (siehe Beispiel~\ref{example: minimal polynomials}).
Da $\sqrt{3}$ Nullstelle des Polynoms $t^2 - 3 \in \Rational(\sqrt{2})[t]$ ist, gilt $\fieldindex{\Rational(\sqrt{2})(\sqrt{3})}{\Rational(\sqrt{2})} \leq 2$.
Im Fall $\fieldindex{\Rational(\sqrt{2})(\sqrt{3})}{\Rational(\sqrt{2})} = 1$ wäre $\sqrt{3} \in \Rational(\sqrt{2})$, was aber nicht gilt:

\begin{lemma}
  Es gilt $\sqrt{3} \notin \Rational(\sqrt{2})$.
\end{lemma}

Es gilt also $\fieldindex{\Rational(\sqrt{2})(\sqrt{3})}{\Rational(\sqrt{2})} = 2$.
Insgesamt gilt somit $\fieldindex{L}{\Rational} = 2$.
Dabei ist $1, \sqrt{2}$ eine $\Rational$-Basis von $\Rational(\sqrt{2})$, und $1, \sqrt{3}$ eine $\Rational(\sqrt{2})$-Basis von $\Rational(\sqrt{2})(\sqrt{3}) = L$.
Eine $\Rational$-Basis von $L$ ist deshalb durch
\[
  1         \cdot 1         = 1 \,,
  \quad
  \sqrt{2}  \cdot 1         = \sqrt{2}  \,,
  \quad
  1         \cdot \sqrt{3}  = \sqrt{3}  \,,
  \quad
  \sqrt{2}  \cdot \sqrt{3}  = \sqrt{6}
\]
gegeben.














\section{Beispiel: Kreisteilungskörper \texorpdfstring{$\Rational(\zeta_n)/\Rational$}{Q(zeta\_n)/Q}}

