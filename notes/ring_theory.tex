\chapter{Ringtheorie}

Es seien $R, S$ zwei Ringe.

\begin{definition}
  Eine Abbildung $f \colon R \to S$ ist ein Ringhomomorphismus, wenn
  \[
      f(r_1 + r_2)
    = f(r_1) + f(r_2),
    \quad
      f(r_1 r_2)
    = f(r_1) f(r_2),
    \quad
      f(1) = 1
  \]
  für alle $r_1, r_2 \in R$ gilt.
\end{definition}

\begin{definition}
  Der \emph{Kern} eines Ringhomomorphismus $f \colon R \to S$ ist
  \[
              \ker(f)
    \defined  \{ x \in R \suchthat f(x) = 0 \} \,.
  \]
\end{definition}

\begin{definition}
  Eine Teilmenge $S \subseteq R$ ist ein \emph{Unterring}, wenn
  \[
    s_1 + s_2 \in S,
    \quad
    s_1 s_2 \in S,
    \quad
    1 \in S
  \]
  für alle $s_1, s_2 \in S$ gilt.
\end{definition}

Ist $X \subseteq R$ eine Teilmenge, so sind für jeden Unterring $S \subseteq R$ die folgenden Bedingungen äquivalent:

\begin{enumerate}
  \item
    Es ist $S$ der kleinste Unterring, der $X$ enthält, d.h.\ es gilt $X \subseteq S$, und für jeden Unterring $S' \subseteq R$ mit $X \subseteq S'$ gilt $S \subseteq S'$.
  \item
    Es gilt $S = \bigcap_{\text{Unterring $S' \subseteq S$, $X \subseteq S'$}} S'$.
  \item
    Es gilt $S = \{ \sum_{i=1}^n x_{i,1} \dotsm x_{i,m_i} \suchthat n, m_i \in \Natural, x_{i,j} \in X \}$.
\end{enumerate}

\begin{definition}
  Der Unterring $S \subseteq R$ der eine \textup(und damit alle\textup) der obigen Bedingungen erfüllt, ist der \emph{von $X$ erzeugte} Unterring von $R$.
  Es ist $X$ ein \emph{Erzeugendensystem} von $R$.
\end{definition}


\begin{definition}
  Für $x, y \in R$ ist $x$ ein \emph{Teiler} von $y$ wenn es ein $z \in R$ mit $y = xz$ gibt.
  Dies wird mit $x \divides y$ notiert.
\end{definition}





\section{Einheiten und Nullteiler}

Es sei $R$ ein Ring.

\begin{definition}
  Ein Element $u \in R$ ist eine \emph{Einheit}, wenn es ein $r \in R$ mit $ur = 1$ und $ru = 1$ gibt.
  Es ist
  $
              \unitgroup{R}
    \defined  \{
                u \in R
              \suchthat 
                \text{$u$ ist eine Einheit}
              \}
  $
  die \emph{Einheitengruppe} von $R$.
\end{definition}

Es bildet $\unitgroup{R}$ bezüglich der Multiplikation von $R$ eine Gruppe, was den Begriff der Einheiten\emph{gruppe} rechtfertigt.

\begin{example}
  \begin{enumerate}
    \item
      Ist $K$ ein Körper, so gilt $\unitgroup{\matrices{n}{K}} = \GL{n}{K}$.
    \item
      Ein kommutativer Ring $R$ ist genau dann ein Körper, wenn $\unitgroup{R} = R \smallsetminus \{0\}$ gilt.
    \item
      Es gilt $\unitgroup{\Integer} = \{1, -1\}$.
    \item
      Für $n \geq 1$ ist ein Element $\class{k} \in \Integer/n$ ist genau dann eine Einheit, wenn $n$ und $k$ teilerfremd sind.
  \end{enumerate}
\end{example}

\begin{definition}
  Ein Element $r \in R$ ist ein \emph{Linksnullteiler} bzw.\ \emph{Rechtsnullteiler}, wenn es ein $x \in R$ mit $x \neq 0$ und $rx = 0$, bzw.\ $xr = 0$ gibt.
  Ist $r$ ein Links- und Rechtsnullteiler, so ist $r$ ein \emph{\textup(beidseitiger\textup) Nullteiler}.
\end{definition}

\begin{example}
  \begin{enumerate}
    \item
      Für $n \geq 1$ ist $\class{k} \in \Integer/n$ genau dann ein Nullteiler, wenn $k$ und $n$ nicht teilerfremd sind.
    \item
      Ist $V$ ein Vektorraum, so ist $f \in \End{V}$ genau dann ein
      \begin{itemize}
        \item
          Linksnullteiler, wenn $f$ nicht injektiv ist,
        \item
          Rechtsnullteiler, wenn $f$ nicht surjektiv ist.
      \end{itemize}
      Ist $V$ endlichdimensional, so sind beide Bedingungen äquivalent.
      Ist $V$ unendlichdimensional, so gibt es ein $f \in \End{V}$, das surjektiv aber nicht injektiv ist, und somit ein Linksnullteiler aber kein Rechtsnullteiler ist;
      analog gibt es in $\End{V}$ auch Rechtsnullteiler, die keine Linkssnullteiler sind.
  \end{enumerate}
\end{example}


\begin{definition}
  Ein \emph{Integritätsbereich} ist ein kommutativer Ring $R$ mit $R \neq \{0\}$, so dass $0$ der einziger Nullteiler in $R$ ist.
\end{definition}





\section{Polynomringe}

Es sei $R$ ein Ring.



\subsection{Polynomring in einer Variables}

\begin{definition}
  Der \emph{Grad} eines Polynoms $f = \sum_{i=0}^n a_i t^i \in R[t]$ ist für $f \neq 0$ durch
  \[
              \deg(f)
    \defined  \max \{i \suchthat a_i \neq 0\}
  \]
  definiert, und für $f = 0$ durch $\deg(0) \defined -\infty$.
\end{definition}


\begin{proposition}
  \begin{enumerate}
    \item
      $R[t]$ ist genau dann kommutativ, wenn $R$ kommutativ ist.
    \item
      Für alle $f, g \in R[t]$ gilt $\deg(fg) \leq \deg(f) + \deg(g)$.
    \item
      Dabei gilt genau dann Gleichheit für alle $f, g \in R[t]$, wenn $R$ keinen von $0$ verschiedenen Linksnullteiler \textup(und somit auch keinen von $0$ verschiedenen Rechtsnullteiler\textup) besitzt.
      Dies gilt insbesondere, wenn $R$ ein Integritätsbereich ist.
    \item
      $R[t]$ ist genau dann ein Integritätsbereich, wenn $R$ ein Integritätsbereich ist.
    \item
      Ist $R$ ein Integritätsbereich, so gilt $\unitgroup{R[t]} = \unitgroup{R}$.
  \end{enumerate}
\end{proposition}

\begin{proposition}[Polynomdivision]
  Es sei $K$ ein Körper.
  Für alle $f, g \in K[t]$ mit $g \neq 0$ gibt es eindeutige Polynome $q, r \in K[t]$ mit
  \[
    f = qg + r \,,
    \qquad
    \text{wobei $\deg(r) < \deg(g)$} \,.
  \]
\end{proposition}

Es sei nun $R$ kommutativ.
Ist $S$ ein kommutativer Ring, so dass $R \subseteq S$ ein Unterring ist, so lassen sich Elemente $s \in S$ in Polynome $f = \sum_{i=0}^n a_i t^i \in R[t]$ einsetzen:
\[
            f(s)
  \defined  \sum_{i=0}^n a_i s^i \,.
\]
Die Abbildung $R[t] \to S$, $f \mapsto f(s)$ ist dabei ein Ringhomomorphismus.
Ist allgemeiner $\varphi \colon R \to S$ ein Ringhomomorphismus, so ergibt sich für jedes $s \in S$ ein Ringhomomorphismus ,
\[
          R[t]
  \to     S \,,
  \quad   \sum_{i=0}^n a_i t^i
  \mapsto \sum_{i=0}^n \varphi(a_i) s^i \,.
\]

\begin{theorem}[Universelle Eigenschaft des Polynomrings]
  Sind $R, S$ kommutative Ringe, so gibt es für jeden Ringhomomorphismus $\varphi \colon R \to S$ und jedes Element $s \in S$ einen eindeutigen Ringhomomorphismus $\hat{\varphi} \colon R[t] \to S$ mit $\restrict{\hat{\varphi}}{R} = \varphi$ und $\hat{\varphi}(t) = s$;
  dabei gilt
  \[
      \hat{\varphi}\left( \sum_{i=0}^n a_i t^i \right)
    = \sum_{i=0}^n \varphi(a_i) s^i \,.
  \]
  In anderen Worten:
  Es ergibt sich eine Bijektion
  \begin{align*}
                            \{ \text{Ringhomo.\ $\hat{\varphi} \colon R[t] \to S$} \}
    &\xlongrightarrow{\sim} \{
                              (\varphi, s)
                            \suchthat
                                \text{Ringhomo.\ $\varphi \colon R \to S$, $s \in S$}
                            \} \,,  \\
                            \hat{\varphi}
    &\longmapsto            \left( \restrict{\hat{\varphi}}{R}, \hat{\varphi}(t) \right) \,.
  \end{align*}
\end{theorem}



\subsection{Polynomringe in endlich vielen Variablen}

Ist $R$ ein Ring, so lässt sich der Polynomring $R[t_1, \dotsc, t_n]$ in endlich vielen Variablen induktiv als $R[t_1, \dotsc, t_n] \defined R[t_1, \dotsc, t_{n-1}][t_n]$ definieren.

\begin{example}
  Es gilt $3tu^2 + 2u^2 + t^2 u - 4u + t + 6 \in \Integer[t, u]$.
  Unter der Identifikation $\Integer[t,u] \cong \Integer[t][u]$ entspricht dies dem Polynom $(3t+2) u^2 + (t^2 - 4)u + (t + 6)$, und unter der Identifikation $\Integer[t,u] \cong \Integer[u][t]$ dem Polynom $ut^2 + (3u^2 + 1)t + (2u^2 - 4u + 6)$.
\end{example}


\begin{theorem}[Universelle Eigenschaft des Polynomrings]
  Sind $R, S$ kommutative Ringe, so gibt es für jeden Ringhomomorphismus $\varphi \colon R \to S$ und alle Elemente $s_1, \dotsc, s_n \in S$ einen eindeutigen Ringhomomorphismus $\hat{\varphi} \colon R[t_1, \dotsc, t_n] \to S$ mit $\restrict{\hat{\varphi}}{R} = \varphi$ und $\hat{\varphi}(t_i) = s_i$ für alle $i = 1, \dotsc, n$;
  dabei gilt
  \[
      \hat{\varphi}\left( \sum_{\alpha \in \Natural^n} a_\alpha t_1^{\alpha_1} \dotsm t_n^{\alpha_n} \right)
    = \sum_{\alpha \in \Natural^n} a_\alpha s_1^{\alpha_1} \dotsm ts_n^{\alpha_n} \,.
  \]
  In anderen Worten:
  Es ergibt sich eine Bijektion
  \begin{align*}
                          &\, \{\text{Ringhomo.\ $\hat{\varphi} \colon R[t_1, \dotsc, t_n] \to S$}\}  \\
    \xlongrightarrow{\sim}&\, \{
                                (\varphi, s_1, \dotsc, s_n)
                              \suchthat
                                  \text{Ringhomo.\ $\varphi \colon R \to S$, $s_i \in S$}
                              \}  \\
                              \hat{\varphi}
    \longmapsto&\,            ( \restrict{\hat{\varphi}}{R}, \hat{\varphi}(t_1), \dotsc, \hat{\varphi}(t_n) ) \,.
  \end{align*}
\end{theorem}

\begin{remark}
  Es lassen sich auch allgemeiner Polynomringe $R[t_i \suchthat i \in I]$ für eine beliebige Indexmenge $I$ konstruieren.
\end{remark}

\begin{lemma}
  \label{lemma: constructed of generated subring}
  Es seien $R, S$ kommutative Ringe, so dass $R \subseteq S$ ein Unterring ist.
  Für $s_1, \dotsc, s_n \in S$ sei
  \[
            \eval_{s_1, \dotsc, s_n}
    \colon  R[t_1, \dotsc, t_n]
    \to     S
  \]
  der eindeutige Ringhomomorphismu mit
  \[
      \restrict{\eval_{s_1, \dotsc, s_n}}{R}
    = \id_R
    \qquad\text{und}\qquad
      \eval_{s_1, \dotsc, s_n}(t_i)
    = s_i
  \]
  für alle $i$.
  Dann ist $\im(\eval_{s_1, \dotsc, s_n})$ der von $R$ und $s_1, \dotsc, s_n$ erzeugte Unterring von $S$.
\end{lemma}

\begin{definition}
  In der obigen Situation ist $\eval_{s_1, \dotsc, s_n}$ der \emph{Einsetzhomomorphismus} und $R[s_1, \dotsc, s_n] \defined \im(\eval_{s_1, \dotsc, s_n})$ der von $R$ und den $s_i$ erzeugte Unterring von $S$.
\end{definition}



\section{Ideale und Quotientenringe}

Es sei $R$ ein Ring.

\begin{definition}
  Eine Teilmenge $I \subseteq R$ ist ein \emph{Linksideal}, wenn $I$ eine additive Untergruppe ist, und $RI \subseteq I$ gilt, d.h.\ für alle $r \in R$, $x \in I$ gilt $rx \in I$.
  Gilt $IR \subseteq I$, so ist $I$ ein \emph{Rechtsideal}.
  Ist $I$ ein Links-\ und Rechtsideal, so ist $I$ ein \emph{\textup(beidseitiges\textup) Ideal}.
  Dies wird dann mit $I \ideal R$ notiert.
\end{definition}

\begin{example}
  \begin{enumerate}
    \item
      Die Ideale in $\Integer$ sind genau $n\Integer$ mit $n \in \Integer$.
    \item
      Für jeden Ring $R$ sind $\{0\}$ und $R$ Ideale in $R$.
    \item
      Ein kommutativer Ring $K$ ist genau dann ein Körper, wenn $\{0\}$ und $K$ die einzigen beiden Ideale in $K$ sind.
    \item
      Für jeden Ringhomomorphismus $f \colon R \to S$ ist $\ker(f)$ ein Ideal in $R$.
  \end{enumerate}
\end{example}

\noindent
\begin{minipage}[t]{\textwidth}
Ein \emph{Ideal} in $R$ ist das analog zu einem Normalteiler einer Gruppe:
\begin{center}
  \renewcommand{\arraystretch}{1.3}
  \begin{tabular}{ccc}
      Gruppe $G$
    & $\leadsto$
    & Ring $R$
    \\
      Untergruppe $H \subgroup G$
    & $\leadsto$
    & Unterring $S \subseteq R$
    \\
      Normalteiler $N \normalgroup G$
    & $\leadsto$
    & Ideal $I \ideal G$
  \end{tabular}
\end{center}
Man beachte jedoch, dass $I \ideal R$ für $I \neq R$ kein Unterring ist, da dann $1 \notin I$ gilt.
\end{minipage}

So wie sich Quotientegruppen $G/N$ konstruieren lassen, gibt es nun auch Quotientenringe $R/I$:
Es sei $I \ideal R$ ein Ideal.
Dann wird auf der Quotientengruppe $R/I$ durch
\[
    \class{x} \cdot \class{y}
  = \class{xy}
  \quad
  \text{für alle $x, y \in R$}
\]
eine Ringstruktur definiert.
Dies ist die eindeutige Ringstruktur auf $R/I$, welche die kanonische Projektion $p \colon R \to R/I$, $r \mapsto \class{r}$ zu einem Ringhomomorphismus macht, und es gilt $\ker(p) = I$.

\begin{theorem}[Homomorphiesatz für Ringe]
  Ist $f \colon R \to S$ ein Ringhomomorphismus mit $I \subseteq \ker f$, so induziert $f$ einen eindeutigen Ringhomomorphismus $\induced{f} \colon R/I \to S$ mit $f = \induced{f} \circ p$, d.h.\ so dass das folgende Diagramm kommutiert:
  \[
    \begin{tikzcd}
        R
        \arrow{rr}[above]{f}
        \arrow{dr}[below left]{p}
      & {}
      & S
      \\
        {}
      & R/I
        \arrow[dashed]{ru}[below right]{\induced{f}}
      & {}
    \end{tikzcd}
  \]
  In anderen Worten:
  Es ergibt sich eine Bijektion
  \begin{align*}
                            \{ \text{Ringhomo.\ $\induced{f} \colon R/I \to S$} \}
    &\xlongrightarrow{\sim} \{ \text{Ringomo.\ $f \colon R \to S$ mit $I \subseteq \ker(f)$} \} \,,  \\
                            \induced{f}
    &\mapsto                \induced{f} \circ p \,.
  \end{align*}
\end{theorem}

\begin{corollary}[1.\ Isomorphiesatz]
  Jeder Ringhomomorphismus $f \colon R \to S$ induziert einen Ringisomorphismus
  \[
            \induced{f}
    \colon  R/{\ker(f)}
    \to     \im(f) \,,
    \quad   \class{r}
    \mapsto f(r) \,.
  \]
\end{corollary}

\begin{corollary}[2.\ Isomorphiesatz]
  Sind $I, J \ideal R$ Ideale mit $J \subseteq I$, so ist $I/J$ ein Ideal in $R/J$ und es gibt einen wohldefinierten Ringisomorphismus
  \[
                            (R/J)/(I/J)
    \xlongrightarrow{\sim}  R/I \,,
    \quad                   \class{\class{r}}
    \mapsto                 \class{r} \,.
  \]

\end{corollary}

\begin{corollary}[3.\ Isomorphiesatz]
  Es sei $S \subseteq R$ ein Unterring und $I \ideal R$ ein Ideal.
  Dann ist $S + I$ ein Unterring von $R$, $I$ ein Ideal in $S + I$, $S \cap I$ ein Ideal in $S$, und es gibt einen wohldefinierten Ringisomorphismus
  \[
                            R/(R \cap I)
    \xlongrightarrow{\sim}  (R + I)/I \,,
    \quad                   \class{r}
    \mapsto                 \class{r} \,.
  \]
\end{corollary}

\begin{lemma}
  \label{lemma: correspondence between ideals}
  Es sei $I \ideal R$ ein Ideal und $p \colon R \to R/I$ die kanonische Projektion.
  \begin{enumerate}
    \item
      Es gibt es eine 1:1-Korrespondenz
      \begin{align*}
        \{ \text{Unterringe $S \subseteq R$ mit $I \subseteq S$} \}
        &\xleftrightarrow{1:1}
        \{ \text{Unterringe $S' \subseteq R/I$} \} \,,
        \\
        S
        &\longmapsto
        p(S)
        =
        S/I \,,
        \\
        p^{-1}(S')
        &\longmapsfrom
        S' \,.
      \end{align*}
    \item
      Es gibt es eine 1:1-Korrespondenz
      \begin{align*}
        \{ \text{Ideale $J \ideal R$ mit $I \subseteq J$} \}
        &\xleftrightarrow{1:1}
        \{ \text{Ideale $J' \ideal R/I$} \} \,,
        \\
        J
        &\longmapsto
        p(J)
        =
        J/I \,,
        \\
        p^{-1}(J')
        &\longmapsfrom
        J' \,.
      \end{align*}
    \item
      Für jedes Ideal $J \ideal R$ mit $I \subseteq J$ gilt dabei für das zugehörige Ideal $J' = p(J) = J/I$ nach dem 2.\ Isomorphiesatz, dass
      \[
              (R/I)/J'
        \cong (R/I)/(J/I)
        \cong  R/J \,.
      \]
      Auf beiden Seiten der obigen 1:1-Korrespondenz erhält man also \textup(bis auf Isomorphie\textup) die gleichen Quotientenringe.
  \end{enumerate}

  
\end{lemma}

Es sei $X \subseteq R$ eine Teilmenge.
Für jedes Ideal $I \subseteq R$ sind dann die folgenden Bedingungen äquivalent:

\begin{enumerate}
  \item
    Es ist $I$ das kleinste Ideal, das $X$ enthält, d.h.\ es gilt $X \subseteq I$, und für jedes Ideal $J \ideal R$ mit $X \subseteq J$ gilt $I \subseteq J$.
  \item
    Es gilt $I = \bigcap_{J \ideal R, X \subseteq J} J$.
  \item
    Es gilt
    $
        I
      = \{
          r_1 x_1 r'_1 + \dotsb + r_n x_n r'_n
        \suchthat
          n \in \Natural,
          r_i, r'_i \in R,
          x_i \in X
        \}
    $
\end{enumerate}

\begin{definition}
  Das Ideal $I \ideal R$, dass eine \textup(und damit alle\textup) der obigen Bedingungen erfüllt, ist das \emph{von $X$ erzeugte} Ideal, und wird mit $\genideal{X}$ notiert.
  Es ist dann $X$ ist ein \emph{Erzeugendensystem} von $I$.
\end{definition}

\begin{definition}
  Gibt es für $I \ideal R$ ein $x \in R$ mit $I = \genideal{x}$, so ist $I$ ein \emph{Hauptideal}.
\end{definition}





\section{Prim- und maximale Ideale}

Es sei $R$ ein kommutativer Ring.

\begin{definition}
  \begin{enumerate}
    \item
      Ein Ideal $P \ideal R$ ist ein \emph{Primideal}, bzw.\ \emph{prim}, wenn $P \neq R$ gilt, und für alle $x,y \in P$ mit $xy \in P$ bereits $x \in P$ oder $y \in P$ gilt.
    \item
      Ein Ideal $M \ideal R$ ist ein \emph{maximales Ideal}, wenn $M \neq R$ gilt, und für jedes Ideal $I \ideal R$ mit $M \subsetneq I$ bereits $I = R$ gilt;
      es gibt also bzgl.\ $\subseteq$ kein größeres echtes Ideal.
  \end{enumerate}
\end{definition}

\begin{lemma}
  \begin{enumerate}
    \item
      $P \ideal R$ ist genau dann prim, wenn $R/P$ ein Integritätsbereich ist.
    \item
      $M \ideal R$ ist genau dann maximal, wenn $R/M$ ein Körper ist.
  \end{enumerate}
\end{lemma}

\begin{corollary}
  Maximale Ideale sind prim.
\end{corollary}


\begin{corollary}
  Ist $I \ideal R$ ein Ideal und $p \colon R \to R/I$ die kanonische Projektion, so schränkt sich die 1:1-Korrespondenz
  \begin{align*}
    \{ \text{Ideale $J \ideal R$ mit $I \subseteq J$} \}
    &\xleftrightarrow{1:1}
    \{ \text{Ideale $J' \ideal R/I$} \} \,,
    \\
    J
    &\longmapsto
    p(J)
    =
    J/I \,,
    \\
    p^{-1}(J')
    &\longmapsfrom
    J'
  \intertext{aus Lemma~\ref{lemma: correspondence between ideals} zu 1:1-Korrespondenzen}
    \{ \text{Primideale $P \ideal R$ mit $I \subseteq P$} \}
    &\xleftrightarrow{1:1}
    \{ \text{Primideale $P' \ideal R/I$} \}
  \intertext{und}
    \{ \text{maximale Ideale $M \ideal R$ mit $I \subseteq M$} \}
    &\xleftrightarrow{1:1}
    \{ \text{maximale Ideale $M' \ideal R/I$} \}
  \end{align*}
  ein.
\end{corollary}


\begin{lemma}[Existenz maximaler Ideale]
  Für jedes echte Ideal $I \ideal R$, $I \neq R$ gibt es ein maximales Ideal $M \ideal R$ mit $I \subseteq M$.
\end{lemma}

\begin{corollary}
  Ist $R \neq 0$, so gibt es ein maximales Ideal $M \ideal R$.
\end{corollary}





\pagebreak





\section{Lokalisierung}



\subsection{Für allgemeinen kommutative Ringe}

Es sei $R$ ein kommutativer Ring.

\begin{definition}
  Eine Teilmenge $S \subseteq R$ ist \emph{multiplikativ \textup(abgeschlossen\textup)}, wenn $1 \in S$ gilt, und für alle $s, t \in S$ auch $st \in S$ gilt.
\end{definition}

Ist $S \subseteq R$ eine multiplikative Teilmenge, so wird auf $R \times S$ durch
\begin{equation}
\label{equation: formula for localization}
        (r,s) \sim (r', s')
  \iff  \exists t \in S:
        rs't = r'st
\end{equation}
eine Äquivalenzrelation definiert.
Für alle $r \in R$, $s \in S$ ist
\[
            \frac{r}{s}
  \defined  [(r,s)]_{\sim} \,,
\]
und es ist
\[
            S^{-1} R
  \defined  (R \times S)/{\sim}
  =         \left\{
              \frac{r}{s}
            \suchthat*
              r \in R,
              s \in S
            \right\}.
\]
Auf $S^{-1} R$ wird durch
\[
              \frac{r}{s}
            + \frac{r'}{s'}
  \defined  \frac{rs' + r's}{ss'}
  \qquad\text{und}\qquad
                  \frac{r}{s}
            \cdot \frac{r'}{s'}
  \defined  \frac{rr'}{ss'}
\]
die Struktur eines kommutativen Rings definiert.
Das Nullelement ist durch $0/1$ gegeben, und das Einselement durch $1/1$.

\begin{definition}
  Der Ring $S^{-1} R$ ist die \emph{Lokalisierung} von $R$ an $S$.
\end{definition}


Für alle $r/s \in S^{-1} R$ und $t \in S$ gilt dabei
\[
    \frac{rt}{st}
  = \frac{r}{s} \,.
\]
Für jedes $s \in S$ ist deshalb $s/1$ eine Einheit mit
\[
    \left( \frac{s}{1} \right)^{-1}
  = \frac{1}{s} \,.
\]
Dies spiegelt sich in der universellen Eigenschaft des kanonischen Ringhomomorphismus $i \colon R \to S^{-1} R$, $r \mapsto r/1$ wieder:

\begin{theorem}[Universelle Eigenschaft der Lokalisierung]
  Ist $f \colon R \to T$ ein Ringhomomorphismus, so dass $f(s)$ für jedes $s \in S$ eine Einheit ist, so induziert $f$ einen eindeutigen Ringhomomorphismus $\induced{f} \colon S^{-1} R \to T$ mit $f = \induced{f} \circ i$, d.h.\ so dass das folgende Diagramm kommutiert:
  \[
    \begin{tikzcd}
        R
        \arrow{rr}[above]{f}
        \arrow{dr}[below left]{i}
      & {}
      & T
      \\
        {}
      & S^{-1} R
        \arrow[dashed]{ur}[below right]{\induced{f}}
      & {}
    \end{tikzcd}
  \]
  In anderen Worten:
  Es ergibt sich eine Bijektion
  \begin{align*}
                            \{ \text{Ringhomo.\ $\induced{f} \colon S^{-1} R \to T$} \}
    &\xlongrightarrow{\sim} \{ \text{Ringhomo.\ $f \colon R \to T$ mit $f(S) \subseteq \unitgroup{T}$} \} \,,  \\
                            \induced{f}
    &\mapsto                \induced{f} \circ i \,.
  \end{align*}
\end{theorem}

\begin{warning}
  Der kanonische Ringhomomorphismus $i \colon R \to S^{-1} R$ ist im Allgemeinen nicht injektiv.
  Es ist $i$ genau dann injektiv, wenn $S$ keinen Nullteiler enthält.
\end{warning}



\subsection{Für Integritätsbereiche}

Es sei $R$ ein Integritätsbereich.

Gilt $0 \in S$, so ist $S^{-1} R = 0$ der Nullring.
Gilt hingegen $0 \notin S$, so enthält $S$ keinen Nullteiler, weshalb der kanonische Ringhomomorphismus $i \colon R \to S^{-1} R$ dann injektiv ist.
Außerdem lässt sich die rechte Seite von \eqref{equation: formula for localization} dann zu $rs' = r's$ vereinfachen.

Da $S$ ein Integritätsbereich ist, lässt sich $S = R \smallsetminus \{0\}$ wählen.
Die Lokalisierung $S^{-1} R$ ist dann bereits ein Körper, denn für jedes $r/s \in S^{-1} R$ mit $r/s \neq 0$ gilt $r \neq 0$, weshalb $s/r \in S^{-1} R$ ein wohldefinierter Bruch ist, für den
\[
        \frac{r}{s}
  \cdot \frac{s}{r}
  =     \frac{rs}{rs}
  =     \frac{1}{1}
  =     1_{S^{-1} R}
\]
gilt.

\begin{definition}
  Ist $R$ ein Integritätsbereich, so ist $\Quot{R} \defined S^{-1} R$ für $S = R \smallsetminus \{0\}$ der \emph{Quotientenkörper} von $R$.
\end{definition}

Der kanonische Ringhomomorphismus $i \colon R \to \Quot{R}$ injektiv, weshalb sich $R$ als ein Unterring des Körpers $\Quot{R}$ auffassen lässt.

\begin{corollary}
  Integritätsbereiche sind genau die Unterring von Körpern, d.h.\ ein Ring $R$ ist genau dann ein Integritätsbereich, wenn es einen Körper $K$ gibt, so dass $R \subseteq K$ ein Unterring ist.
\end{corollary}

Es ist $\Quot{R}$ der \enquote{kleinste} Körper, der $R$ enthält:
Ist $K$ ein Körper und $j \colon R \to K$ ein injektiver Ringhomomorphismus, so faktorisiert $j$ eindeutig über $i$, d.h.\ es gibt einen eindeutigen Körperhomomorphismus $\induced{j} \colon \Quot{R} \to K$, der das folgende Diagramm zum Kommutieren bringt:
\[
  \begin{tikzcd}
      R
      \arrow{rr}[above]{j}
      \arrow{dr}[below left]{i}
    & {}
    & K
    \\
      {}
    & \Quot{R}
      \arrow[dashed]{ur}[below right]{\induced{j}}
    & {}
  \end{tikzcd}
\]

\begin{example}
  \begin{enumerate}
    \item
      Es gilt $\Quot{\Integer} = \Rational$.
    \item
      Ist $K$ ein Körper, so ist der kanonische Ringhomomorphismus $i \colon K \to \Quot{K}$ ein Isomorphismus.
    \item
      Ist $K$ ein Körper, so ist $\Quot{K[t_1, \dotsc, t_n]} \eqqcolon K(t_1, \dotsc, t_n)$ der \emph{Funktionenkörper} oder \emph{Körper der rationalen Funktionen} in den Variablen $t_1, \dotsc, t_n$.
  \end{enumerate}
\end{example}





\section{Hauptideal- und euklidische Ringe}

\begin{definition}
  Ein \emph{Hauptidealring} ist ein Integritätsbereich $R$, so dass jedes Ideal $I \ideal R$ ein Hauptideal ist.
\end{definition}

\begin{definition}
  Ein \emph{euklidischer Ring} ist ein kommutativer Ring $R$ zusammen mit einer \emph{Gradabbildung} $\delta \colon R \smallsetminus \{0\} \to \Natural$, so dass es für alle $f, g \in R$ mit $g \neq 0$ Elemente $q, r \in R$ gibt, so dass
  \[
    f = qg + r \,,
    \qquad
    \text{wobei $r = 0$ oder $\delta(r) < \delta(g)$} \,.
  \]
\end{definition}

\begin{lemma}
  Jeder euklidische Ring $R$ ist ein Hauptidealring:
  Ist $I \ideal R$ mit $I \neq \{0\}$ und hat $x \in I$ minimalen Grad unter allen Elementen in $I \smallsetminus \{0\}$, so gilt $I = \genideal{x}$.
\end{lemma}

\begin{example}
  \begin{enumerate}
    \item
      Der Ring $\Integer$ ist ein euklidischer Ring bezüglich der Gradabbildung $\delta(n) \defined \abs{x}$.
      Die Folgerung, dass $\Integer$ ein Hauptidealring ist, sowie die obige explizite Beschreibung eines Erzeugers eines Ideal $I \ideal R$, entspricht genau Lemma~\ref{lemma: subgroups of Z}.
    \item
      Ist $K$ ein Körper, so ist der Polynomring $K[t]$ ein euklidischer Ring bezüglich der üblichen Gradabbildung $\delta \defined \deg$.
      Insbesondere ist $K[t]$ ein Hauptidealring.
  \end{enumerate}
\end{example}

\begin{remark}
  Ist $R$ ein Integritätsbereich Ring, so ist für jedes $a \in R$ ist das Ideal $\genideal{t, a} \ideal R[t]$ genau dann ein Hauptideal, wenn $a$ eine Einheit in $R$ ist.
  Deshalb ist $R[t]$ genau dann ein Hauptidealring, wenn $R$ ein Körper ist.
  
  So sind etwa $\Integer[t]$ und $K[t,u] \cong K[t][u]$ für einen Körper $K$ keine Hauptidealringe, denn $\genideal{t, 2} \ideal \Integer[t]$ und $\genideal{t,u} \ideal K[t][u]$ sind keine Hauptideale.
\end{remark}

\begin{example}
  Es ist $\Integer[\sqrt{-1}] \defined \Integer[i] = \{a + bi \suchthat a, b \in \Integer\}$ ein Unterring von $\Complex$, der Ring der \emph{Gaußschen Zahlen}.
  Für jedes $z = a+ib \in \Integer[i]$ ist $N(z) \defined \abs{z}^2 = a^2 + b^2 \in \Natural$ die \emph{Norm} von $z$.
  Dann ist $\Integer[i]$ zusammen mit der Norm $N$ ein euklidischer Ring:
  
  Für jedes $z \in \Complex$ gibt es ein Element $w \in \Integer[i]$ mit $\abs{z-w} \leq \sqrt{2}/2 = 1/\sqrt{2}$.
  Sind nun $f, g \in \Integer[i]$ mit $g \neq 0$, so lässt sich in $\Complex$ der Quotient $f/g$ bilden.
  Dann gibt es ein $q \in \Integer[i]$ mit $\abs{f/g - q} < 1/\sqrt{2}$.
  Für den Rest $r \defined f - qg$ gilt dann $f = qg + r$, sowie
  \[
          N(r)
    =     \abs{r}^2
    =     \abs{f - qg}^2
    =     \abs{f/g - q}^2 \abs{g}^2
    \leq  \frac{1}{2} \abs{g}^2
    <     \abs{g}^2
    =     N(g) \,.
  \]
\end{example}





\section{Faktorielle Ringe}

Es sei $R$ ein kommutativer Ring.

\begin{definition}
  Zwei Elemente $a, b \in R$ sind \emph{assoziiert \textup(zueinander\textup)}, wenn sie \enquote{gleich bis auf Einheit} sind, d.h.\ wenn es eine Einheit $\varepsilon \in \unitgroup{R}$ mit $b = \varepsilon a$ gibt.
\end{definition}

\begin{lemma}
  Assoziiertheit ist eine Äuquivalenzrelation auf $R$.
\end{lemma}

\begin{remark}
  Ist $R$ ein Integritätsbereich, so sind zwei Elemente $a, b \in R$ genau dann assoziiert, wenn $\genideal{a} = \genideal{b}$ gilt, wenn also $a \divides b$ und $b \divides a$ gelten.
  Wenn wir im Folgenden von \enquote{Eindeutigkeit bis auf Assoziiertheit} von Elementen sprechen, so geht es also eigentlich um eindeutige Hauptideale.
\end{remark}



\subsection{Definition faktorieller Ringe}

Es sei $R$ ein Integritätsbereich.

\begin{definition}
  Es sei $p \in R$ eine Nichteinheit mit $p \neq 0$.
  \begin{enumerate}
    \item
      $p$ ist \emph{irreduzibel}, wenn für jede Zerlegung $p = xy$ bereits $x \in \unitgroup{R}$ oder $y \in \unitgroup{R}$ gilt.
    \item
       $p$ ist prim, wenn für alle $x, y \in R$ mit $p \divides (xy)$ bereits $p \divides x$ oder $p \divides y$ gilt, d.h.\ wenn das Ideal $\genideal{p}$ prim ist.
  \end{enumerate}
\end{definition}


\begin{definition}
\label{definition: ufd}
  Der Ring $R$ ist \emph{faktoriell}, wenn jedes $r \in R$, $r \neq 0$ eine Zerlegung
  \begin{equation}
  \label{equation: decomposition into irreducibles}
    r = \varepsilon p_1 \dotsm p_n
  \end{equation}
  besitzt, wobei
  \begin{itemize}
    \item
      $\varepsilon \in \unitgroup{R}$ eine Einheit ist,
    \item
      $p_1, \dotsc, p_n$ irreduzibel sind, und
    \item
      diese Zerlegung eindeutig \enquote{bis auf Assoziiertheit und Permutation} ist:
      
      Falls $r = \varepsilon' p'_1 \dotsm p'_m$ eine weitere solche Zerlegung ist, so gilt $n = m$, und es gibt Einheiten $\gamma_1, \dotsc, \gamma_n \in \unitgroup{R}$ und eine Permutation $\pi \in S_n$ mit $p'_i = \gamma_i p_{\pi(i)}$ für alle $i = 1, \dotsc, n$.
  \end{itemize}
\end{definition}

\begin{lemma}
  Ist $R$ faktoriell, so ist $p \in R$ genau dann irreduzibel, wenn $p$ prim ist.
\end{lemma}

In faktoriellen Ringen muss also nicht zwischen Prim- und irreduziblen Elementen unterschieden werden.
Deshalb bezeichnet man für $r \in R$, $r \neq 0$ die Zerlegung \eqref{equation: decomposition into irreducibles} als \emph{Primfaktorzerlegung}.

\begin{remark}
  Ist $R$ faktoriell und $\mathcal{P} \subseteq R$ ein Repräsentantensystem der Assoziiertheitsklassen der Primelemente, so lässt sich jedes Element $r \in R$, $r \neq 0$ als
  \[
    r = \varepsilon \prod_{p \in \mathcal{P}} p^{\nu_p}
  \]
  schreiben, wobei $\varepsilon \in \unitgroup{R}$ eine Einheit ist, und $\nu_p = 0$ für fast alle $p \in \mathcal{P}$ gilt.
  Diese Zerlegung ist dann eindeutig (bis auf Permutation der $p^{\nu_p}$).
\end{remark}


\begin{proposition}
  Hauptidealringe sind faktoriell.
\end{proposition}

\begin{example}
  \begin{enumerate}
    \item
      Der Ring der ganzen Zahlen $\Integer$ ist faktoriell.
      Für $\mathcal{P}$ lässt sich die Menge der üblichen Primzahlen wählen.
    \item
      Ist $K$ ein Körper, so ist der Polynomring $K[t]$ faktoriell.
      Für $\mathcal{P}$ lässt sich die Menge der normierten irreduziblen Polynome wählen.
    \item
      Der Ring der gaußschen Zahlen $\Integer[i]$ ist faktoriell.
  \end{enumerate}
\end{example}

\begin{example}
  Es ist $\Integer[\sqrt{-5}] \defined \Integer[i\sqrt{5}] = \{a + ib\sqrt{5} \suchthat a, b \in \Integer \}$ ein Unterring von $\Complex$, der \emph{nicht} faktoriell ist:
  Es sei $R \defined \Integer[\sqrt{-5}]$ und für jedes $z = a +ib\sqrt{5} \in R$ sei $N(z) \defined \abs{z}^2 = a^2 + 5 b^2$ die \emph{Norm} von $z$.
  Dann gilt
  \begin{itemize}
    \item
      $N(z) \in \Natural$ für alle $z \in R$, und
    \item
      $N(zw) = N(z) N(w)$ für alle $z, w \in R$.
  \end{itemize}
  
  Mithilfe der Norm $N$ lassen sich die Einheiten von $R$ bestimmen:
  Aus $zw = 1$ in $R$ folgt $N(z)N(w) = 1$ in $\Natural$, also $N(z) = N(w) = 1$ in $\Natural$ und somit $z, w = \pm 1$.
  Somit gilt $\unitgroup{R} = \{1, -1\} = \{z \in R \suchthat N(z) = 1\}$.
  
  Hieraus ergibt sich, dass die Elemente $2, 3, 1 + \sqrt{-5}, 1 - \sqrt{-5}$ irreduzibel in $R$ sind:
  Gilt etwa $2 = zw$ für $z,w \in R$, so gilt $4 = N(2) = N(z)N(w)$.
  Es gilt somit
  \[
    N(z) = 1, N(w) = 4 \,,
    \quad\text{oder}\quad
    N(z) = 2, N(w) = 2 \,,
    \quad\text{oder}\quad
    N(z) = 4, N(w) = 1 \,.
  \]
  Da es aber kein $z' \in R$ mit $N(z') = 2$ gibt, muss $N(z) = 1$ oder $N(w) = 1$ gelten.
  Somit ist $z$ oder $w$ eine Einheit.
  Die Irreduziblität von $3, 1 + \sqrt{-5}, 1 - \sqrt{-5}$ ergibt sich analog.
  
  Nun gilt aber $6 = 2 \cdot 3 = (1 + \sqrt{-5})(1 - \sqrt{-5})$.
  Dabei sind die irreduziblen Elemente $2, 3, 1 + \sqrt{-5}, 1 - \sqrt{-5}$ paarweise nicht assoziiert zueinander (denn $z \in R$ ist nur zu $\pm z$ assoziiert).
  Somit besitzt $6 \in R$ zwei nicht-äquivalente Zerlegungen in irreduzible Elemente.
\end{example}




\subsection{Prim- und irreduzible Elemente im Allgemeinen}

Es sei $R$ ein Integritätsbereich, und $p \in R$.

\begin{lemma}
  Ist $p$ prim, so ist $p$ auch irreduzibel.
\end{lemma}

\begin{lemma}
  Ist $R$ ein Hauptidealring, so sind die folgenden Bedingungen äquivalent:
  \begin{enumerate}
    \item
      Das Ideal $\genideal{p}$ ist maximal.
    \item
      Das Ideal $\genideal{p}$ ist prim.
    \item
      Das Element $p$ ist prim.
    \item
      Das Element $p$ ist irreduzibel.
  \end{enumerate}
  Inbesondere ist jedes Primideal in $R$ schon maximal.
\end{lemma}



\subsection{\texorpdfstring{$\ggT$}{ggT} und \texorpdfstring{$\kgV$}{kgV}}

Es sei $R$ ein kommutativer Ring

\begin{definition}
  Es seien $a_i \in R$, $i \in I$.
  \begin{enumerate}
    \item
      Ein \emph{größter gemeinsamer Teiler} der $a_i$ ist ein Element $a \in R$ mit $a \divides a_i$ für alle $i \in I$, so dass für jedes andere $b \in R$ mit $b \divides a_i$ für alle $i \in I$ bereits $b \divides a$ gilt.
      Man schreibt $a = \ggT( a_i \suchthat i \in I )$.
    \item
      Ein \emph{kleinstes gemeinsames Vielfaches} der $a_i$ ist ein Element $a \in R$ mit $a_i \divides a$ für alle $i \in I$, so dass für jedes andere $b \in R$ mit $a_i \divides b$ für alle $i \in I$ bereits $a \divides b$ gilt.
      Man schreibt $a = \kgV( a_i \suchthat i \in I )$.
  \end{enumerate}
\end{definition}

\begin{lemma}
  Ist $R$ ein Integritätsbereich, so sind größte gemeinsame Teiler und kleinste gemeinsame Vielfache \textup(sofern sie existieren\textup) eindeutig bis auf Assoziiertheit.
\end{lemma}

\begin{lemma}[Berechnung von $\ggT$ und $\kgV$ in faktoriellen Ringen]
  Es sei $R$ ein faktorieller Ring und $\mathcal{P} \subseteq R$ ein Repräsentantensystem der Assoziiertheitsklassen der Primelement von $R$.
  Es seien $a_1, \dotsc, a_n \in R$, $a_i \neq 0$ mit Primfaktorzerlegungen
  \[
      a_i
    = \varepsilon_i \prod_{p \in \mathcal{P}} p^{\nu_{p,i}} \,.
  \]
  Dann existieren $\ggT(a_1, \dotsc, a_n)$ und $\kgV(a_1, \dotsc, a_n)$, und es gilt
  \begin{align*}
        \ggT(a_1, \dotsc, a_n)
    &=  \prod_{p \in \mathcal{P}} p^{\min \{ \nu_{p,1}, \dotsc, \nu_{p,n} \}}
  \shortintertext{und}
        \kgV(a_1, \dotsc, a_n)
    &=  \prod_{p \in \mathcal{P}} p^{\max \{ \nu_{p,1}, \dotsc, \nu_{p,n} \}} \,.
  \end{align*}
\end{lemma}

\begin{definition}
  Es sei $R$ ein faktorieller Ring und es seien $a_1, \dotsc, a_n \in R$.
  Die Elemente $a_1, \dotsc, a_n$ sind \emph{\textup(insgesamt\textup) teilerfremd}, wenn $\ggT(a_1, \dotsc, a_n) = 1$ gilt.
  In anderen Worten:
  Jedes Element $a \in R$ mit $a \divides a_i$ für alle $i$ ist bereits eine Einheit.
\end{definition}

\begin{lemma}[Charakterisierung von $\ggT$ und $\kgV$ in Hauptidealringen]
  Ist $R$ ein Hauptidealring, so gilt für alle $a_1, \dotsc, a_n \in R$, dass
  \begin{gather*}
      \genideal{ a_1 } + \dotsb + \genideal{ a_n }
    = \genideal{ a_1, \dotsc, a_n }
    = \genideal{ \ggT(a_1, \dotsc, a_n) }
  \shortintertext{sowie}
      \genideal{ a_1 } \cap \dotsb \cap \genideal{ a_n }
    = \genideal{ \kgV(a_1, \dotsc, a_n) } \,.
  \end{gather*}
  Inbesondere gibt es Koeffizienten $c_1, \dotsc, c_n \in R$ mit
  \[
      \ggT(a_1, \dotsc, a_n)
    = c_1 a_1 + \dotsb + c_n a_n \,.
  \]
\end{lemma}

\begin{corollary}
  Ist $R$ ein Hauptidealring, so sind $a, b \in R$ genau dann teilerfremd, wenn $\genideal{a} + \genideal{b} = R$ gilt.
\end{corollary}

Ist $R$ ein euklidischer Ring mit Gradabbildung $\delta$, so lässt sich der größte gemeinsame Teiler von $f, g \in R$, $g \neq 0$ mithilfe des \emph{euklidischen Algorithmus} berechnen:
Es gibt $q, r \in R$ mit $a = qb + r$, wobei $r = 0$ oder $\delta(r) < \delta(b)$ gilt.
Es gilt dann
\[
    \ggT(a,b)
  = \ggT(qb+r,b)
  = \ggT(r,b)
  = \ggT(b,r) \,.
\]
Iteriert man dieses Vorgehen, so ergibt sich aus $\delta(r) < \delta(b)$, dass nach endlich vielen Schritten der Fall $r = 0$ eintritt.
Dann lässt sich nutzen, dass
\[
    \ggT(a,0)
  = a
\]
gilt.
Es lassen sich dann auch Koeffizienten $c, d \in R$ mit $\ggT(a,b) = ca + db$ bestimmen.

\begin{example}
  Es gilt
  \begin{align*}
     &\,  \ggT(84, 30)
    =     \ggT(2 \cdot 30 + 24, 30) \\
    =&\,  \ggT(30, 24)
    =     \ggT(24 + 6, 24)  \\
    =&\,  \ggT(24, 6)
    =     \ggT(4 \cdot 6 + 0, 6)  \\
    =&\,  \ggT(6,0)
    =     6 \,.
  \end{align*}
  Wir erhalten außerdem, dass
  \[
      \ggT(84, 30)
    = 6
    = 30 - 24
    = 30 - (84 - 2 \cdot 30)
    = 3 \cdot 30 - 84 \,.
  \]
\end{example}





\subsection{Der Satz von Gauß}

Es sei $R$ ein faktorieller Ring.

\begin{definition}
  Ein Polynom $f = \sum_{i=0}^n a_i t^i \in R[t]$ ist \emph{primitiv}, wenn die Koeffizienten $a_0, \dotsc, a_n$ insgesamt teilerfremd sind.
\end{definition}

\begin{theorem}[Satz von Gauß]
  Der Polynomring $R[t]$ ist ebenfalls faktoriell.
  Dabei ist ein Polynom $p \in R[t]$ genau dann irreduzibel, wenn eine der folgenden beiden Bedingungen erfüllt ist:
  \begin{itemize}
    \item
      Es gilt $p \in R$, und $p$ ist irreduzibel in $R$.
    \item
      Das Polynom $p$ ist primitiv und irreduzibel in $\Quot{R}[t]$.
  \end{itemize}
  Ein primitives Polynom $p \in R[t]$ ist genau dann irreduzibel in $R[t]$, wenn $p$ irreduzibel in $\Quot{R}[t]$ ist.
\end{theorem}

\begin{corollary}
  Der Polynomring $R[t_1, \dotsc, t_n]$ ist faktoriell.
\end{corollary}

\begin{example}
  \begin{enumerate}
    \item
      Ist $K$ ein Körper, so ist $K[t_1, \dotsc, t_n]$ faktoriell.
    \item
      $\Integer[t_1, \dotsc, t_n]$ ist faktoriell.
  \end{enumerate}
\end{example}





\section{Irreduziblitätskriterien}

Es sei $R$ ein faktorieller Ring.

\begin{lemma}
  Es sei $K$ ein Körper und $f \in K[t]$ nicht-konstant.
  \begin{enumerate}
    \item
      Gilt $\deg(f) = 1$, so ist $f$ irreduzibel.
    \item
      Gilt $\deg(f) = 2$ oder $\deg(f) = 3$ so ist $f$ genau dann irreduzibel, wenn $f$ keine Nullstelle in $K$ besitzt.
    \item
      Gilt $\deg(f) \geq 2$ und besitzt $f$ eine Nullstelle in $K$, so ist $f$ reduzibel.
  \end{enumerate}
\end{lemma}

\begin{example}
  Jedes Polynom $f \in \Real[t]$ ungeraden Grades besitzt nach dem Zwischenwertsatz eine Nullstelle;
  für $\deg(f) \neq 1$ ist $f$ somit reduzibel.
\end{example}

\begin{proposition}[Eisenstein]
  Es sei $f = \sum_{i=0}^n a_i t^i \in R[t]$ primitiv, und es gebe $p \in R$ prim mit
  \[
    p \notdivides a_n \,,
    \qquad
    \text{$p \divides a_i$ für alle $i < n$} \,,
    \qquad
    p^2 \notdivides a_0 \,.
  \]
  Dann ist $f$ irreduzibel in $R[t]$, und somit auch in $\Quot{R}[t]$.
\end{proposition}

\begin{proposition}[Reduktionskriterium]
  Es sei $f \in R[t]$ primitiv und $p \in R$ prim, so dass der Leitkoeffizient von $f$ nicht von $p$ geteilt wird.
  Für jedes Polynom $g = \sum_{j=0}^m b_j t^j \in R[t]$ sei
  \[
              \class{g}
    \defined  \sum_{j=0}^m \class{b_j} t^j
    \in       (R/\genideal{p})[t]
  \]
  das Polynom, dass durch Reduzieren der Koeffizienten modulo $p$ entsteht.
  Ist $\class{f}$ irreduzibel in $(R/\genideal{p})[t]$, so ist $f$ irreduzibel in $R[t]$, und somit auch in $\Quot{R}[t]$.
\end{proposition}







\section{Chinesischer Restsatz}

Es sei $R$ ein Ring.
Für Ideale $I_1, \dotsc, I_n \ideal R$ induzieren die kanonischen Projektionen $p_j \colon R \to R/I_j$ einen Ringhomomorphismus
\[
            p
  \defined  (p_1, \dotsc, p_n)
  \colon    R
  \to       (R/I_1) \times \dotsb \times (R/I_n) \,,
  \quad     r
  \mapsto   (\class{r}, \dotsc, \class{r})
\]
mit $\ker(p) = \bigcap_{j=1}^n \ker(p_j) = \bigcap_{j=1}^n I_j$.
Gilt dabei $I_j + I_k = R$ für alle $j \neq k$, so ist $p$ auch surjektiv:

\begin{theorem}[Chinesischer Restsatz]
  Sind $I_1, \dotsc, I_n \ideal R$ Ideale mit $I_j + I_k = R$ für alle $j \neq k$, so gibt es einen wohldefinierten Isomorphismus
  \[
                            R / \bigcap_{j=1}^n I_j
    \xlongrightarrow{\sim}  (R/I_1) \times \dotsb \times (R/I_n) \,,
    \quad                   \class{r}
    \mapsto                 (\class{r}, \dotsc, \class{r}) \,.
  \]
\end{theorem}

\begin{corollary}
  Es sei $R$ ein Hauptidealring, und es seien $a_1, \dotsc, a_n \in R$ paarweise teilerfremd.
  Dann gibt es einen wohldefinierten Ringisomorphismus
  \[
                            R/\genideal{\kgV(a_1, \dotsc, a_n)}
    \xlongrightarrow{\sim}  R/\genideal{a_1} \times \dotsb \times R/\genideal{a_n} \,,
    \quad                   \class{r}
    \mapsto                 ( \class{r}, \dotsc, \class{r} ) \,.
  \]
\end{corollary}

\begin{corollary}
  Für jedes $n \geq 1$ mit Primfakorzerlegung $n = p_1^{n_1} \dotsm p_r^{n_r}$ gilt
  \[
          \Integer/n
    \cong \Integer/p_1^{n_1} \times \dotsb \times \Integer/p_r^{n_r} \,.
  \]
\end{corollary}

\begin{corollary}
  Sind $n_1, \dotsc, n_k \in \Integer$ und $b_1, \dotsc, b_k \in \Integer$, und ist $x_0 \in \Integer$ eine Lösung des Systems simultaner Kongruenzen
  \[
    \left\{
      \begin{array}{ccll}
        x &\equiv& b_1  & \pmod{n_1}  \,, \\
        x &\equiv& b_2  & \pmod{n_2}  \,, \\
          &\vdots&      &                 \\
        x &\equiv& b_k  & \pmod{n_k}  \,,
      \end{array}
    \right.
  \]
  so ist die Lösungsmenge dieses Systems von Kongruenzen durch
  \[
      x_0
    + \kgV(n_1, \dotsc, n_k) \Integer
  \]
  gegeben.
  Sind $n_1, \dotsc, n_k$ paarweise teilerfremd, so existert eine solche Lösung $x_0$.
\end{corollary}

\begin{example}
  Wir betrachten das folgende System simultaner Kongruenzen:
  \[
    \left\{
      \begin{array}{ccll}
        x &\equiv& 6  & \pmod{11} \,, \\
        x &\equiv& 7  & \pmod{13} \,,
      \end{array}
    \right.
  \]
  Da $11$ und $13$ teilerfremd sind, besitzt gibt es eine Lösung $x_0$.
  Die Lösungsmenge ist dann $x_0 + 143\Integer$.
  Die Lösungen der ersten Kongruenz sind
  \[
    6,
    17,
    28,
    39,
    50,
    61,
    72,
    \dotsc
  \]
  Dabei gilt $11 \equiv -2 \pmod{13}$, weshalb die obigen Gleichungen modulo $13$ zu
  \[
    \class{6},
    \class{4},
    \class{2},
    \class{0},
    \class{-2} = \class{11},
    \class{9},
    \class{7},
    \dotsc
  \]
  werden.
  Es lässt sich somit $x_0 = 72$ wählen.
\end{example}

\begin{example}
  Wir betrachten das folgende System simultaner Kongruenzen:
  \[
    \left\{
      \begin{array}{ccll}
        x &\equiv&  7 & \pmod{6}  \,, \\
        x &\equiv&  5 & \pmod{15} \,,
      \end{array}
    \right.
  \]
  Mithilfe des chinesischen Restklassensatzen können wir die einzelnen Kongruenzen auftrennen:
  \begin{align*}
          \left\{
            \begin{array}{ccll}
              x &\equiv&  7 & \pmod{2}  \,, \\
              x &\equiv&  7 & \pmod{3}  \,, \\
              x &\equiv&  5 & \pmod{3}  \,, \\
              x &\equiv&  5 & \pmod{5} \,,
            \end{array}
          \right.
    \iff  \left\{
            \begin{array}{ccll}
              x &\equiv&  1 & \pmod{2}  \,, \\
              x &\equiv&  1 & \pmod{3}  \,, \\
              x &\equiv&  2 & \pmod{3}  \,, \\
              x &\equiv&  0 & \pmod{5} \,.
            \end{array}
          \right.
  \end{align*}
  Die mittleren beiden Kongruenzen stehen im Widerspruch, weshalb es keine Lösungen gibt.
\end{example}

\begin{example}
  Wir betrachten das abgeänderte System
  \[
    \left\{
      \begin{array}{ccll}
        x &\equiv&  8 & \pmod{6}  \,, \\
        x &\equiv&  5 & \pmod{15} \,,
      \end{array}
    \right.
  \]
  Durch Aufteilen der einzelnen Kongruenzen erhalten wir nun
  \begin{align*}
          \left\{
            \begin{array}{ccll}
              x &\equiv&  8 & \pmod{2}  \,, \\
              x &\equiv&  8 & \pmod{3}  \,, \\
              x &\equiv&  5 & \pmod{3}  \,, \\
              x &\equiv&  5 & \pmod{5}  \,,
            \end{array}
          \right.
    \iff  \left\{
            \begin{array}{ccll}
              x &\equiv&  0 & \pmod{2}  \,, \\
              x &\equiv&  2 & \pmod{3}  \,, \\
              x &\equiv&  0 & \pmod{5}  \,.
            \end{array}
          \right.
  \end{align*}
  Die äußeren beiden Kongruenzen lassen sich zu $x \equiv 2 \pmod{10}$ zusammenfassen, und wir erhalten das kleinere System
  \[
    \left\{
      \begin{array}{ccll}
        x &\equiv&  2 & \pmod{3}  \,, \\
        x &\equiv&  0 & \pmod{10} \,.
      \end{array}
    \right.
  \]
  Die Lösungen der unteren Kongruenz sind
  \[
    0, 10, 20, \dotsc,
  \]
  und modulo $3$ wir dies zu
  \[
    \class{0},
    \class{1},
    \class{2},
    \dotsc.
  \]
  Eine Lösung ist also durch $x_0 = 20$ gegeben.
  Die Lösungsmenge ist insgesamt
  \[
      x_0 + \kgV(6,15)\Integer
    = 20 + 30\Integer \,.
  \]
\end{example}

\begin{remark}
  Simultane Kongruenzen lassen sich aus mithilfe des euklidischen Algorithmus lösen.
  Ein entsprechende Erklärung, sowie das Ausrechnen der Beispiele von Übungsblatt~6, findet sich unter \url{goo.gl/J2ML2r}\footnote{https://github.com/cionx/einfuehrung-in-die-algebra-tutorial-ws-17-18/raw/master
  /sheet\_06/sheet\_06.pdf}.
\end{remark}












