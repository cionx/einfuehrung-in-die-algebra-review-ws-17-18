\chapter{Ringtheorie}



\section{Grundlegendes}

Es seien $R, S$ zwei Ringe.

\begin{definition}
  Eine Abbildung $f \colon R \to S$ ist ein Ringhomomorphismus, wenn
  \[
      f(r_1 + r_2)
    = f(r_1) + f(r_2),
    \quad
      f(r_1 r_2)
    = f(r_1) f(r_2),
    \quad
      f(1) = 1
  \]
  für alle $r_1, r_2 \in R$ gilt.
\end{definition}

\begin{definition}
  Der \emph{Kern} eines Ringhomomorphismus $f \colon R \to S$ ist
  \[
              \ker(f)
    \defined  \{x \in R \suchthat f(x) = 0\}
  \]
\end{definition}

\begin{definition}
  Eine Teilmenge $S \subseteq R$ ist ein \emph{Unterring}, wenn
  \[
    s_1 + s_2 \in S,
    \quad
    s_1 s_2 \in S,
    \quad
    1 \in S
  \]
  für alle $s_1, s_2 \in S$ gilt.
\end{definition}

\begin{definition}
  Für $x, y \in R$ ist $x$ ein \emph{Teiler} von $y$, bzw.\ $y$ wird von $x$ \emph{geteilt}, wenn es ein $z \in R$ mit $y = xz$ gibt.
  Dies wird mit $x \divides y$ notiert.
\end{definition}




\section{Ideale und Quotientenringe}

Es sei $R$ ein Ring.

\begin{definition}
  Eine Teilmenge $I \subseteq R$ ist ein Ideal, wenn $I$ eine additive Untergruppe ist, und $RI, IR \subseteq I$ gelten, d.h.\ für alle $r \in R$, $x \in I$ gilt $rx \in I$ und $xr \in I$.
  Dies wird mit $I \ideal R$ notiert.
\end{definition}

\begin{example}
  \begin{enumerate}
    \item
      Die Ideale in $\Integer$ sind genau $n\Integer$ mit $n \in \Integer$.
    \item
      Für jeden Ring $R$ sind $\{0\}$ und $R$ Ideale in $R$.
    \item
      Ist $K$ ein Körper, so sind $\{0\}$ und $K$ die einzigen beiden Ideale in $K$.
    \item
      Für jeden Ringhomomorphismus $f \colon R \to S$ ist $\ker(f)$ ein Ideal in $R$.
  \end{enumerate}
\end{example}

Es gilt auch die Umkehrung des obigen Beispiels:
Es sei $I \ideal R$ ein Ideal.
Dann ist auf der Quotientengruppe $R/I$ durch
\[
    \class{x} \cdot \class{y}
  = \class{xy}
  \quad
  \text{für alle $x, y \in R$}
\]
eine Ringstruktur definiert.
Dies ist die eindeutige Ringstruktur auf $R/I$, welche die kanonische Projektion $p \colon R \to R/I$, $r \mapsto \class{r}$ zu einem Ringhomomorphismus macht, und es gilt $\ker p = I$.

\begin{theorem}[Homomorphiesatz]
  Ist $f \colon R \to S$ ein Ringhomomorphismus mit $I \subseteq \ker f$, so induziert $f$ einen eindeutigen Ringhomomorphismus $\induced{f} \colon R/I \to S$ mit $f = \induced{f} \circ p$, d.h.\ so dass das folgende Diagramm kommutiert:
  \[
    \begin{tikzcd}
        R
        \arrow{rr}[above]{f}
        \arrow{dr}[below left]{p}
      & {}
      & S
      \\
        {}
      & R/I
        \arrow{ru}[below right]{\induced{f}}
      & {}
    \end{tikzcd}
  \]
  In anderen Worten:
  Es ergibt sich eine Bijektion
  \begin{align*}
                            \{ \text{Ringhomo.\ $\induced{f} \colon R/I \to S$} \}
    &\xlongrightarrow{\sim} \{ \text{Ringomo.\ $f \colon R \to S$ mit $I \subseteq \ker(f)$} \} \,,  \\
                            \induced{f}
    &\mapsto                \induced{f} \circ p \,.
  \end{align*}
\end{theorem}

\begin{corollary}[1.\ Isomorphiesatz]
  Jeder Ringhomomorphismus $f \colon R \to S$ induziert einen Ringisomorphismus
  \[
            \induced{f}
    \colon  R/{\ker(f)}
    \to     \im(f) \,,
    \quad   \class{r}
    \mapsto f(r) \,.
  \]
\end{corollary}

\begin{corollary}[2.\ Isomorphiesatz]
  Sind $I, J \ideal R$ Ideale mit $J \subseteq I$, so ist $I/J$ ein Ideal in $R/J$ gibt es einen wohldefinierten Ringisomorphismus
  \[
                            (R/J)/(I/J)
    \xlongrightarrow{\sim}  R/I \,,
    \quad                   \class{\class{r}}
    \mapsto                 \class{r} \,.
  \]

\end{corollary}

\begin{corollary}[3.\ Isomorphiesatz]
  Es sei $S \subseteq R$ ein Unterring und $I \ideal R$ ein Ideal.
  Dann ist $S + I$ ein Unterring von $R$, $I$ ein Ideal in $S + I$, $S \cap I$ ein Ideal in $S$, und es gibt einen wohldefinierten Ringisomorphismus
  \[
                            R/(R \cap I)
    \xlongrightarrow{\sim}  (R + I)/I \,,
    \quad                   \class{r}
    \mapsto                 \class{r} \,.
  \]
\end{corollary}

\begin{lemma}
  \label{lemma: correspondence between ideals}
  Es sei $I \ideal R$ ein Ideal und $p \colon R \to R/I$ die kanonische Projektion.
  Dann gibt es eine 1:1-Korrespondenz
  \begin{align*}
    \{ \text{Ideale $J \ideal R$ mit $I \subseteq J$} \}
    &\xleftrightarrow{1:1}
    \{ \text{Ideale $J' \ideal R/I$} \} \,,
    \\
    J
    &\longmapsto
    p(J)
    =
    J/I \,,
    \\
    p^{-1}(J')
    &\longmapsfrom
    J' \,.
  \end{align*}
  Für jedes Ideal $J \ideal R$ mit $I \subseteq J$ gilt dabei für das zugehörige Ideal $J' = p(J) = J/I$ nach dem 2.\ Isomorphiesatz, dass
  \[
          R/J
    \cong (R/I)/(J/I)
    \cong (R/I)/J' \,.
  \]
  Auf beiden Seiten der obigen 1:1-Korrespondenz erhält man also \textup(bis auf Isomorphie\textup) die gleichen Quotientenringe.
\end{lemma}

Es sei $X \subseteq R$ eine Teilmenge.
Für jedes Ideal $I \subseteq R$ sind dann die folgenden Bedingungen äquivalent:

\begin{enumerate}
  \item
    Es ist $I$ das kleinste Ideal, das $X$ enthält, d.h.\ es gilt $X \subseteq I$, und für jedes Ideal $J \ideal R$ mit $X \subseteq J$ gilt $I \subseteq J$.
  \item
    Es gilt $I = \bigcap_{J \ideal R, X \subseteq J} J$.
  \item
    Es gilt
    $
        I
      = \{
          r_1 x_1 r'_1 + \dotsb + r_n x_n r'_n
        \suchthat
          n \in \Natural,
          r_i, r'_i \in R,
          x_i \in X
        \}
    $
\end{enumerate}

\begin{definition}
  Das Ideal $I \ideal R$, dass eine \textup(und damit alle\textup) der obigen Bedingungen erfüllt, ist das von $X$ \emph{erzeugte} Ideal, und wird mit $\genideal{X}$ notiert.
  Es ist $X$ ist ein Erzeugendensystem von $I$.
\end{definition}

\begin{definition}
  Gibt es für $I \ideal R$ ein $x \in R$ mit $I = \genideal{x}$, so ist $I$ ein \emph{Hauptideal}.
\end{definition}





\section{Einheiten und Nullteiler}

Es sei $R$ ein Ring.

\begin{definition}
  Ein Element $u \in R$ ist eine \emph{Einheit}, wenn es ein $r \in R$ mit $ur = 1$ und $ru = 1$ gibt.
  Es ist
  $
              \unitgroup{R}
    \defined  \{
                u \in R
              \suchthat 
                \text{$u$ ist eine Einheit}
              \}
  $
  die \emph{Einheitengruppe} von $R$.
\end{definition}

Es bildet $\unitgroup{R}$ bezüglich der Multiplikation von $R$ eine Gruppe, was den Begriff der Einheiten\emph{gruppe} rechtfertigt.

\begin{example}
  \begin{enumerate}
    \item
      Ist $K$ ein Körper, so gilt $\unitgroup{\matrices{n}{K}} = \GL{n}{K}$.
    \item
      Ein kommutativer Ring $R$ ist genau dann ein Körper, wenn $\unitgroup{R} = R \smallsetminus \{0\}$ gilt.
    \item
      Es gilt $\unitgroup{\Integer} = \{1, -1\}$.
    \item
      Für $n \geq 1$ ist ein Element $\class{k} \in \Integer/n$ ist genau dann eine Einheit, wenn $n$ und $k$ teilerfremd sind.
  \end{enumerate}
\end{example}

\begin{definition}
  Ein Element $r \in R$ ist ein \emph{Linksnullteiler} bzw.\ \emph{Rechtsnullteiler}, wenn es ein $x \in R$ mit $x \neq 0$ und $rx = 0$, bzw.\ $xr = 0$ gibt.
  Ist $r$ ein Links- und Rechtsnullteiler, so ist $r$ ein \emph{\textup(beidseitiger\textup) Nullteiler}.
\end{definition}

\begin{definition}
  Ein \emph{Integritätsbereich} ist ein kommutativer Ring $R$ mit $R \neq \{0\}$, so dass $0$ der einzige Nullteiler in $R$ ist.
\end{definition}






\section{Prim- und maximale Ideale}

Es sei $R$ ein kommutativer Ring.

\begin{definition}
  \begin{enumerate}
    \item
      Ein Ideal $P \ideal R$ ist ein \emph{Primideal}, wenn $P \neq R$ gilt, und für alle $x,y \in R$ mit $xy \in R$ bereits $x \in R$ oder $y \in R$ gilt.
    \item
      Ein Ideal $M \ideal R$ ist ein \emph{maximales Ideal}, wenn $M \neq R$ gilt, und für jedes Ideal $I \ideal R$ mit $M \subsetneq I$ bereits $I = R$ gilt;
      es gibt also bzgl.\ $\subseteq$ kein größeres echtes Ideal.
  \end{enumerate}
\end{definition}

\begin{lemma}
  \begin{enumerate}
    \item
      $P \ideal R$ ist genau dann prim, wenn $R/P$ ein Integritätsbereich ist.
    \item
      $M \ideal R$ ist genau dann maximal, wenn $R/M$ ein Körper ist.
  \end{enumerate}
\end{lemma}

\begin{corollary}
  Ist $M \ideal R$ maximal, so ist $M$ auch prim.
\end{corollary}


\begin{corollary}
  Ist $I \ideal R$ ein Ideal und $p \colon R \to R/I$ die kanonische Projektion, so schränkt sich die 1:1-Korrespondenz
  \begin{align*}
    \{ \text{Ideale $J \ideal R$ mit $I \subseteq J$} \}
    &\xleftrightarrow{1:1}
    \{ \text{Ideale $J' \ideal R/I$} \} \,,
    \\
    J
    &\longmapsto
    p(J)
    =
    J/I \,,
    \\
    p^{-1}(J')
    &\longmapsfrom
    J' \,.
  \intertext{aus Lemma~\ref{lemma: correspondence between ideals} zu 1:1-Korrespondenzen}
    \{ \text{Primideale $P \ideal R$ mit $I \subseteq P$} \}
    &\xleftrightarrow{1:1}
    \{ \text{Primideale $P' \ideal R/I$} \} \,,
  \intertext{und}
    \{ \text{maximale Ideale $M \ideal R$ mit $I \subseteq M$} \}
    &\xleftrightarrow{1:1}
    \{ \text{maximale Ideale $M' \ideal R/I$} \}
  \end{align*}
  ein.
\end{corollary}


\begin{lemma}[Existenz maximaler Ideale]
  Für jedes echte Ideal $I \ideal R$, $I \neq R$ gibt es ein maximales Ideal $M \ideal R$ mit $I \subseteq M$.
\end{lemma}

\begin{corollary}
  Ist $R \neq 0$, so gibt es ein maximales Ideal $M \ideal R$.
\end{corollary}





\section{Polynomringe}

Es sei $R$ ein Ring.



\subsection{Polynomring in einer Variables}

Der Polynomring $R[t]$ ein einer Variablen $t$ besteht aus \emph{Polynomen}, d.h.\ Ausdrücken der Form $\sum_{k=0}^\infty a_k t^k$ mit $a_k = 0$ für fast alle $k$.
Dabei gilt genau dann die Gleichheit $\sum_{k=0}^\infty a_k t^k = \sum_{k=0}^\infty b_k t^k$, wenn $a_k = b_k$ für alle $k$ gilt.
Die Addition und Multiplikation auf $R[t]$ sind durch
\begin{align*}
              \left( \sum_{k=0}^\infty a_k t^k \right)
            + \left( \sum_{k=0}^\infty b_k t^k \right)
  &\defined \sum_{k=0}^\infty (a_k + b_k) t^k
\shortintertext{und}
                  \left( \sum_{k=0}^\infty a_k t^k \right)
            \cdot \left( \sum_{k=0}^\infty b_k t^k \right)
  &\defined \sum_{k,\ell=0}^\infty a_k b_\ell t^{k+\ell}
\end{align*}
definiert.
Es gilt also $(\sum_{k=0}^\infty a_k t^k) \cdot (\sum_{k=0}^\infty b_k t^k) = \sum_{k=0}^\infty c_k t^k$ mit
\[
    c_k
  = \sum_{i + j = k} a_i b_j
  = \sum_{\ell=0}^k a_\ell b_{k - \ell}
\]
für alle $k$.
Der \emph{Grad} eines Polynom $f = \sum_{k=0}^\infty a_k t^k \in R[t]$ ist für $f \neq 0$ durch
\[
            \deg(f)
  \defined  \max \{k \suchthat a_k \neq 0\}
\]
definiert, und für $f = 0$ durch $\deg(0) = -\infty$.

\begin{proposition}
  \begin{enumerate}
    \item
      $R[t]$ ist genau dann kommutativ, wenn $R$ kommutativ ist.
    \item
      Für alle $f, g \in R[t]$ gilt $\deg(fg) \leq \deg(f) + \deg(g)$.
    \item
      Dabei gilt genau dann Gleichheit für alle $f, g \in R[t]$, wenn $R$ keinen von $0$ verschiedenen Linksnullteiler \textup(und somit auch keine von $0$ verschiedenen Rechtsnullteiler\textup) besitzt.
      Dies gilt insbesondere, wenn $R$ ein Integritätsbereich ist.
    \item
      $R[t]$ ist genau dann ein Integritätsbereich, wenn $R$ ein Integritätsbereich ist.
  \end{enumerate}
\end{proposition}

\begin{proposition}[Polynomdivision]
  Es sei $K$ ein Körper.
  Für alle $f, g \in K[t]$ mit $g \neq 0$ gibt es eindeutige Polynome $q, r \in K[t]$ mit
  \[
    f = qg + r \,,
    \qquad
    \text{wobei $\deg(r) < \deg(g)$} \,.
  \]
\end{proposition}



\subsection{Polynomringe in endlich vielen Variablen}










 





\section{Lokalisierung}

Es sei $R$ ein kommutativer Ring.

\begin{definition}
  Eine Teilmenge $S \subseteq R$ ist \emph{multiplikativ \textup(abgeschlossen\textup)}, wenn für alle $s, t \in S$ auch $st \in S$ gilt, und $1 \in S$ gilt.
\end{definition}

Ist $S \subseteq R$ eine multiplikative Teilmenge, so wird auf $R \times S$ durch
\begin{equation}
\label{equation: formula for localization}
        (r,s) \sim (r', s')
  \iff  \exists t \in S:
        rs't = r'st
\end{equation}
eine Äquivalenzrelation definiert.
Für alle $r \in R$, $s \in S$ ist
\[
            \frac{r}{s}
  \defined  [(r,s)]_{\sim} \,,
\]
und es ist
\[
            S^{-1} R
  \defined  (R \times S)/{\sim}
  =         \left\{
              \frac{r}{s}
            \suchthat*
              r \in R,
              s \in S
            \right\}.
\]
Auf $S^{-1} R$ wird durch
\[
              \frac{r}{s}
            + \frac{r'}{s'}
  \defined  \frac{rs' + r's}{ss'}
  \quad\text{und}\quad
                  \frac{r}{s}
            \cdot \frac{r'}{s'}
  \defined  \frac{rr'}{ss'}
\]
die Struktur eines kommutativen Rings definiert.
Das Nullelement ist durch $0/1$ gegeben, und das Einselement durch $1/1$.

\begin{definition}
  Der Ring $S^{-1} R$ wie oben ist die \emph{Lokalisierung} von $R$ an $S$.
\end{definition}


Für alle $r/s \in S^{-1} R$ und $t \in S$ gilt dabei
\[
    \frac{rt}{st}
  = \frac{r}{s} \,.
\]
Für jedes $s \in S$ ist deshalb $s/1$ eine Einheit mit
\[
    \left( \frac{s}{1} \right)^{-1}
  = \frac{1}{s} \,.
\]
Dies spiegelt sich in der universellen Eigenschaft des kanonischen Ringhomomorphismus $i \colon R \to S^{-1} R$, $r \mapsto r/1$ wieder:

\begin{theorem}[Universelle Eigenschaft der Lokalisierung]
  Ist $f \colon R \to T$ ein Ringhomomorphismus, so dass $f(s)$ für jedes $s \in S$ eine Einheit ist, so induziert $f$ einen eindeutigen Ringhomomorphismus $\induced{f} \colon S^{-1} R \to T$ mit $f = \induced{f} \circ i$, d.h.\ so dass das folgende Diagramm kommutiert:
  \[
    \begin{tikzcd}
        R
        \arrow{rr}[above]{f}
        \arrow{dr}[below left]{i}
      & {}
      & T
      \\
        {}
      & S^{-1} R
        \arrow{ur}[below right]{\induced{f}}
      & {}
    \end{tikzcd}
  \]
  In anderen Worten:
  Es gibt eine Bijektion
  \begin{align*}
                            \{ \text{Ringhomo.\ $\induced{f} \colon S^{-1} R \to T$} \}
    &\xlongrightarrow{\sim} \{ \text{Ringhomo.\ $f \colon R \to T$ mit $f(S) \subseteq \unitgroup{T}$} \} \,,  \\
                            \induced{f}
    &\mapsto                \induced{f} \circ i \,.
  \end{align*}
\end{theorem}

\begin{warning}
  Der kanonische Ringhomomorphismus $i \colon R \to S^{-1} R$ ist im Allgemeinen nicht injektiv.
  Es ist $i$ genau dann injektiv, wenn $S$ keinen Nullteiler enthält.
\end{warning}

Es sei $R$ ein Integritätsbereich.
Gilt $0 \in S$, so ist $S^{-1} R = 0$ der Nullring.
Gilt allerdings $0 \notin S$, so enthält $S$ keinen Nullteiler, weshalb der kanonische Ringhomomorphismus $i \colon R \to S^{-1} R$ dann injektiv ist.
Außerdem lässt sich die rechte Seite von \eqref{equation: formula for localization} dann zu $rs' = r's$ vereinfachen.

Da $S$ ein Integritätsbereich ist, lääst sich außerdem $S = R \smallsetminus \{0\}$ wählen.
Die Lokalisierung $S^{-1} R$ ist dann bereits ein Körper, denn für jedes $r/s \in S^{-1} R$ mit $r/s \neq 0$ gilt $r \neq 0$, weshalb $s/r \in S^{-1} R$ ein wohldefinierter Bruch ist, für den
\[
        \frac{r}{s}
  \cdot \frac{s}{r}
  =     \frac{rs}{rs}
  =     \frac{1}{1}
  =     1_{S^{-1} R}
\]
gilt.

\begin{definition}
  Ist $R$ ein Integritätsbereich, so ist $\Quot{R} \defined S^{-1} R$ für $S = R \smallsetminus \{0\}$ der \emph{Quotientenkörper} von $R$.
\end{definition}

Der kanonische Ringhomomorphismus $i \colon R \to \Quot{R}$ injektiv, weshalb sich $R$ als ein Unterring des Körpers $\Quot{R}$ auffassen lässt.

\begin{corollary}
  Integritätsbereiche sind genau die Unterring von Körpern, d.h.\ ein Ring $R$ ist genau dann ein Integritätsbereich, wenn es einen Körper $K$ gibt, so dass $R \subseteq K$ ein Unterring ist.
\end{corollary}

Es ist $\Quot{R}$ der \enquote{kleinste} Körper, der $R$ enthält:
Ist $K$ ein Körper und $j \colon R \to K$ ein injektiver Ringhomomorphismus, so faktorisiert $j$ eindeutig über $i$, d.h.\ es gibt einen eindeutigen Körperhomomorphismus $\induced{j} \colon \Quot{R} \to K$, der das folgende Diagramm zum Kommutieren bringt:
\[
  \begin{tikzcd}
      R
      \arrow{rr}[above]{j}
      \arrow{dr}[below left]{i}
    & {}
    & K
    \\
      {}
    & \Quot{R}
      \arrow{ur}[below right]{\induced{j}}
    & {}
  \end{tikzcd}
\]

\begin{example}
  \begin{enumerate}
    \item
      Es gilt $\Quot{\Integer} = \Rational$.
    \item
      Ist $K$ ein Körper, so ist der kanonische Ringhomomorphismus $i \colon K \to \Quot{K}$ ein Isomorphismus.
%   TODO: Function fields.
  \end{enumerate}
\end{example}









\section{Hauptideal- und euklidische Ringe}

\begin{definition}
  Ein \emph{Hauptidealring} ist ein Integritätsbereich $R$, so dass jedes Ideal $I \ideal R$ ein Hauptideal ist.
\end{definition}

\begin{definition}
  Ein \emph{euklidischer Ring} ist ein kommutativer Ring $R$ zusammen mit einer \emph{Gradabbildung} $\deg \colon R \smallsetminus \{0\} \to \Natural$, so dass es für alle $f, g \in R$ mit $g \neq 0$ Elemente $q, r \in R$ gibt, so dass
  \[
    f = qg + r \,,
    \quad
    \text{wobei $r = 0$ oder $\deg(r) < \deg(g)$} \,.
  \]
\end{definition}

\begin{lemma}
  Jeder euklidische Ring $R$ ist ein Hauptidealring:
  Ist $I \ideal R$ mit $I \neq \{0\}$ und hat $x \in I$ minimalen Grad unter allen Elementen in $I \smallsetminus \{0\}$, so gilt $I = \genideal{x}$.
\end{lemma}

\begin{example}
  \begin{enumerate}
    \item
      Der Ring $\Integer$ ist ein euklidischer Ring bezüglich der Gradabbildung $\deg(n) \defined \abs{x}$.
      Die Folgerung, dass $\Integer$ ein Hauptidealring ist, sowie die obige explizite Beschreibung eines Erzeugers eines Ideal $I \ideal R$ entspricht genau Lemma~\ref{lemma: subgroups of Z}.
    \item
      Ist $K$ ein Körper, so ist der Polynomring $K[t]$ ein euklidischer Ring bezüglich der üblichen Gradabbildung $\deg$.
      Insbesondere ist $K[t]$ ein Hauptidealring.
  \end{enumerate}
\end{example}

\begin{remark}
  Ist $R$ ein kommutativer Ring, so ist für jedes $a \in R$ ist das Ideal $\genideal{t, a} \ideal R[t]$ genau dann ein Hauptideal, wenn $a$ eine Einheit in $R$ ist.
  Deshalb ist $R[t]$ genau dann ein Hauptidealring, wenn $R$ ein Körper ist.
  
  So sind etwa $\Integer[t]$ und $K[t,u] \cong K[t][u]$ für einen Körper $K$ keine Hauptidealringe, denn $\genideal{t, 2} \ideal \Integer[t]$ und $\genideal{t, u} \in K[t][u]$ sind keine Hauptideale.
\end{remark}


% TODO: Example Z[i]







\section{Faktorielle Ringe}

Es sei $R$ ein Integritätsbereich.

\subsection{Definition faktorieller Ringe}

\begin{definition}
  Es sei $p \in R$ eine Nichteinheit mit $p \neq 0$.
  \begin{enumerate}
    \item
      $p$ ist \emph{irreduzibel}, wenn für jede Zerlegung $p = xy$ bereits $x \in \unitgroup{R}$ oder $y \in \unitgroup{R}$ gilt.
    \item
       $p$ ist prim, wenn für alle $x, y \in R$ mit $p \divides xy$ bereits $p \divides x$ oder $p \divides y$ gilt.
  \end{enumerate}
\end{definition}

\begin{definition}
\label{definition: ufd}
  Der Ring $R$ ist \emph{faktoriell}, wenn jedes $r \in R$, $r \neq 0$ eine Zerlegung
  \begin{equation}
  \label{equation: decomposition into irreducibles}
    r = \varepsilon p_1 \dotsm p_n
  \end{equation}
  besitzt, wobei
  \begin{itemize}
    \item
      $\varepsilon \in \unitgroup{R}$ eine Einheit ist, und $p_1, \dotsc, p_n$ irreduzibel sind, und
    \item
      diese Zerlegung eindeutig \enquote{bis auf Einheiten und Permutation} ist:
      
      Falls $r = \varepsilon' p'_1 \dotsm p'_m$ eine weiter solche Zerlegung ist, so gilt $n = m$, und es gibt Einheiten $\gamma_1, \dotsc, \gamma_n \in \unitgroup{R}$ und eine Permutation $\pi \in S_n$ mit $p'_i = \gamma_i p_{\pi(i)}$ für alle $i = 1, \dotsc, n$.
  \end{itemize}
\end{definition}

\begin{lemma}
  Ist $R$ faktoriell, so ist $p \in R$ genau dann irreduzibel, wenn $p$ prim ist.
\end{lemma}

In faktoriellen Ringen muss also nicht zwischen Prim- und irreduziblen Elementen unterschieden werden.
Deshalb bezeichnet man für $r \in R$, $r \neq 0$ die Zerlegung \eqref{equation: decomposition into irreducibles} als \emph{Primfaktorzerlegung}.

\begin{proposition}
  Hauptidealringe sind faktoriell.
\end{proposition}



\subsection{Prim- und irreduzible Elemente im Allgemeinen}

Es sei $p \in R$.

\begin{lemma}
  Ist $p$ prim, so ist $p$ auch irreduzibel.
\end{lemma}

\begin{lemma}
  Ist $R$ ein Hauptidealring, so sind die folgenden Bedingungen äquivalent:
  \begin{enumerate}
    \item
      Das Element $p$ ist prim.
    \item
      Das Element $p$ ist irreduzibel.
    \item
      Das Ideal $\genideal{p}$ ist maximal.
  \end{enumerate}
\end{lemma}

% TODO: Counterexample Z[sqrt(-5)]


\subsection{\texorpdfstring{$\ggT$}{ggT} und \texorpdfstring{$\kgV$}{kgV}}

% TODO: Write this.



\subsection{Der Satz von Gauß}

Es sei $R$ ein faktorieller Ring.

\begin{definition}
  Ein Polynom $f = \sum_{k=0}^n a_k t^k \in R[t]$ ist \emph{primitiv}, wenn die Koeffizienten $a_0, \dotsc, a_n$ insgesamt teilerfremd sind, d.h.\ für $r \in R$ mit $r \divides a_k$ für alle $k = 0, \dotsc, n$ gilt bereits $r \in \unitgroup{R}$.
\end{definition}

\begin{theorem}[Satz von Gauß]
  Der Polynomring $R[t]$ ist ebenfalls faktoriell.
  Dabei ist ein Polynom $p \in R[t]$ genau dann irreduzibel, wenn eine der folgenden beiden Bedingungen erfüllt ist:
  \begin{itemize}
    \item
      Es gilt $p \in R$, und $p$ ist irreduzibel in $R$.
    \item
      Das Polynom $p$ ist primitiv und irreduzibel in $\Quot{R}[t]$.
  \end{itemize}
  Insbesondere ist ein primitives Polynom $p \in R[t]$ genau dann irreduzibel in $R[t]$, wenn $p$ irreduzibel in $\Quot{R}[t]$ ist.
\end{theorem}

\begin{corollary}
  Der Polynomring $R[t_1, \dotsc, t_n]$ ist faktoriell.
\end{corollary}

\begin{example}
  \begin{enumerate}
    \item
      Ist $K$ ein Körper, so ist $K[t_1, \dotsc, t_n]$ faktoriell.
    \item
      $\Integer[t_1, \dotsc, t_n]$ ist faktoriell.
  \end{enumerate}
\end{example}





\section{Irreduziblitätskriterien}

Es sei $R$ ein faktorieller Ring.

\begin{lemma}
  Ist $K$ ein Körper, so ist ein nicht-konstantes Polynom $f \in K[t]$ vom Grad $\deg(f) \leq 3$ genau dann irreduzibel, wenn $f$ keine Nullstelle in $K$ besitzt.
\end{lemma}


\begin{proposition}[Eisenstein]
  Es sei $f = \sum_{k=0}^n a_k t^k \in R[t]$ primitiv, und es gebe $p \in R$ prim mit
  \[
    p \notdivides a_n \,,
    \qquad
    \text{$p \divides a_k$ für alle $k < n$} \,,
    \qquad
    p^2 \notdivides a_0 \,.
  \]
  Dann ist $f$ irreduzibel in $R[t]$, und somit auch in $\Quot{R}[t]$.
\end{proposition}

\begin{proposition}[Reduktionskriterium]
  Es sei $f = \sum_{k=0}^n a_k t^k \in R[t]$ primitiv.
  Es sei $p \in R$ prim, und für jedes Polynom $g = \sum_{k=0}^m b_k t^k \in R[t]$ sei
  \[
              \class{g}
    \defined  \sum_{k=0}^m \class{b_k} t^k
    \in       (R/\genideal{p})[t]
  \]
  das Polynom, dass durch Reduzieren der Koeffizienten modulo $p$ entsteht.
  Ist $\class{f}$ irreduzibel in $(R/\genideal{p})[t]$, so ist $f$ in $R[t]$ irreduzibel, und somit auch in $\Quot{R}[t]$.
\end{proposition}







\section{Chinesischer Restsatz}

Es sei $R$ ein Ring.
Für Ideale $I_1, \dotsc, I_n \ideal R$ induzieren die kanonischen Projektionen $p_i \colon R \to R/I_i$ einen Ringhomomorphismus
\[
            p
  \defined  (p_1, \dotsc, p_n)
  \colon    R
  \to       (R/I_1) \times \dotsb \times (R/I_n) \,,
  \quad     r
  \mapsto   (\class{r}, \dotsc, \class{r})
\]
mit $\ker(p) = \bigcap_{i=1}^n \ker(p_i) = \bigcap_{i=1}^n I_i$.
Gilt dabei $I_i + I_j = R$ für alle $i \neq j$, so ist $p$ auch surjektiv:

\begin{theorem}[Chinesischer Restsatz]
  Sind $I_1, \dotsc, I_n \ideal R$ Ideale mit $I_i + I_j = R$ für alle $i \neq j$, so gibt es einen wohldefinierten Isomorphismus
  \[
                            R / \bigcap_{i=1}^n I_i
    \xlongrightarrow{\sim}  \prod_{i=1}^n (R/I_i) \,,
    \quad                   \class{r}
    \mapsto                 (\class{r}, \dotsc, \class{r}) \,.
  \]
\end{theorem}

% TODO: Special Case PID

% TODO: Calculations for Z












