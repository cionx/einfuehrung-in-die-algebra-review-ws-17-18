\documentclass[a4paper, 10pt]{scrartcl}
\usepackage{../styles/generalstyle}
\usepackage{../styles/exercisestyle}

\subject{Repetitorium Einführung in die Algebra, WS 2017/18}
\title{Tag 1}
\author{}
\date{}

\begin{document}

\begin{question}
  Zeigen Sie, dass jede echte Untergruppe der symmetrischen Gruppe $S_3$ zyklisch ist.
  Bestimmen Sie hiermit alle Untergruppen.
\end{question}

\begin{question}
  Es seien $G, H$ zwei endliche Gruppen mit teilerfremden Ordnungen $\card{G}, \card{H}$.
  Zeigen Sie, dass jeder Gruppenhomomorphismus $G \to H$ bereits trivial ist.
\end{question}

\begin{question}
  Es sei $N \normalgroup G$ eine normale Untergruppe, so dass $G/N$ abelsch ist.
  Zeigen Sie, dass jede Untergruppe $H \subgroup G$ mit $N \subgroup H$ ebenfalls normal in $G$ ist.
\end{question}

% \begin{question}
%   Es sei $G$ eine Gruppe, $X$ eine $G$-Menge.
%   Zeigen Sie, dass $G_{g.x} = g G_x g^{-1}$ gilt.
% \end{question}

\begin{question}
  Es sei $G$ eine endliche Gruppe und $x_1, \dotsc, x_n$ ein Repräsentantensystem der Konjugationsklassen von $G$.
  Stellen Sie einen Zusammenhang zwischen den Zahlen $\card{G}$, $\card{\groupcenter{G}}$ und $\card{\centralizer{G}{x_i}}$ her.
\end{question}

\begin{question}
  Es sei $G$ eine Gruppe der Ordnung $77$, die auf einer $17$-elementigen Menge $X$ wirkt.
  Zeigen Sie, dass diese Wirkung mindestens $3$ Fixpunkte hat.
\end{question}

\begin{question}
  Es sei $p$ prim und $G$ eine endliche $p$-Gruppe, d.h.\ es gelte $\card{G} = p^n$ für passendes $n \geq 0$.
  \begin{enumerate}
    \item
      Die Gruppe $G$ wirke auf einer endlichen Menge $X$.
      Zeigen Sie, dass
      \[
                \card{X}
        \equiv  \card*{X^G}
        \pmod{p}
      \]
      gilt.
      Folgern Sie, dass diese Wirkung einen Fixpunkt besitzt, falls $p \notdivides \card{X}$ gilt.
    \item
      Folgern Sie, dass für $G \neq 1$ bereits $\groupcenter{G} \neq 1$ gilt.
  \end{enumerate}
\end{question}

\begin{question}
  Es sei $G$ eine endliche Gruppe mit genau zwei Konjugationsklassen.
  Zeigen Sie, dass bereits $G \cong \Integer/2$ gilt.
\end{question}

\begin{question}
  Es sei $G$ eine Gruppe und $H \subgroup G$ eine Untergruppe von endlichem Index.
  Zeigen Sie, dass es eine normale Untergruppe $N \normalgroup G$ mit $N \subgroup H$ gibt, so dass auch $N$ endlichen Index in $G$ hat.
  \newline
  (\emph{Hinweis}:
  Konstruieren Sie für $n \defined \groupindex{G}{H}$ einen Gruppenhomomorphismus $G \to S_n$.)
\end{question}

% \begin{question}
%   Es sei $K$ ein Körper.
%   Zeigen Sie, dass die Gruppen $(K,+)$ und $(K^\times, \cdot)$ nicht isomorph zueinander sind.
%   \newline
%   (\emph{Hinweis}:
%   Betrachten Sie Elemente der Ordnung $2$.)
% \end{question}






\end{document}
