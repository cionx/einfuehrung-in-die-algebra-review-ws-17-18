\documentclass[a4paper, 10pt]{scrartcl}
\usepackage{../styles/generalstyle}
\usepackage{../styles/exercisestyle}

\subject{Repetitorium Einführung in die Algebra, WS 2017/18}
\title{Tag 3}
\author{}
\date{}

\begin{document}

\begin{question}
  Zeigen Sie, dass die folgenden Polynome irreduzibel sind:
  \begin{enumerate}
    \item
      % Eisenstein
      $3t^6 + 4t^4 - 6t^2 - 10 \in \Integer[t]$
    \item
      % mod 2
      $3t^3 + 39t^2 - 4t + 8 \in \Rational[t]$
    \item
      %  divide by 2, then mod 2
      $2t^3 - 14t + 6 \in \Rational[t]$
    \item
      % t -> t+1
      $t^4 + 1 \in \Rational[t]$
    \item
      % Eisenstein
      $t^5 - u \in \Rational(u)[t]$
    \item
      % Eisenstein
      $t^2 u + tu^2 - t - u + 1 \in \Rational[t,u]$
  \end{enumerate}
\end{question}

\begin{question}
  Bestimmen Sie jeweils alle Lösungen $x \in \Integer$ der folgenden Systeme simultaner Kongruenzen:
  \begin{enumerate}
    \item
      $
        \left\{
          \begin{array}{ccrl}
            x &\equiv& 3 & \pmod{5}  \\
            x &\equiv& 2 & \pmod{7}
          \end{array}
        \right.
      $
    \item
      $
        \left\{
          \begin{array}{ccrl}
            5x &\equiv& 6 & \pmod{12}  \\
            3x &\equiv& 7 & \pmod{11}
          \end{array}
        \right.
      $
    \item
      $
        \left\{
          \begin{array}{ccrl}
            x &\equiv& 10 & \pmod{6}  \\
            x &\equiv& -6 & \pmod{14}
          \end{array}
        \right.
      $
    \item
      $
        \left\{
          \begin{array}{ccrl}
            x &\equiv&  2 & \pmod{6}  \\
            x &\equiv& -2 & \pmod{10} \\
            x &\equiv&  1 & \pmod{7}
          \end{array}
        \right.
      $
  \end{enumerate}
\end{question}

\begin{question}
  Bestimmen Sie mithilfe des euklidischen Algorithmus jeweils den größten gemeinsamen Teiler der folgenden Zahlen, und drücken Sie diesen als $\Integer$-Linearkombination dieser Zahlen aus.
  \begin{enumerate}
    \item
      $270$, $192$
    \item
      $30$, $42$, $70$
  \end{enumerate}
\end{question}

\begin{question}
  Es sei $R$ ein kommutativer Ring und $P \ideal R$ ein Primideal.
  Zeigen Sie, dass
  \[
              S
    \defined  R \smallsetminus P
    =         \{ r \in R \suchthat r \notin P \}
  \]
  ein multiplikative Teilmenge von $R$ ist.
\end{question}

\begin{question}
  Zeigen Sie, dass der Ring $\Integer[\sqrt{-2}] = \Integer[i\sqrt{2}]$ euklidisch ist.
\end{question}

\begin{question}
  Es seien $n, m \geq 1$.
  Zeigen Sie, dass
  \[
          \Integer/n \times \Integer/m
    \cong \Integer/{\ggT(n,m)} \times \Integer/{\kgV(n,m)} \,.
  \]
\end{question}

\begin{question}
  Für alle $n \geq 1$ sei $\varphi(n) = \card{\{1 \leq k \leq n \suchthat \text{$k$ und $n$ sind teilerfremd}\}}$.
  \begin{enumerate}
    \item
      Begründen Sie, dass $\varphi(n) = \card{\unitgroup{(\Integer/n)}}$ gilt.
    \item
      Zeigen Sie, dass $\varphi(nm) = \varphi(n)\varphi(m)$ gilt, wenn $n, m \geq 1$ teilerfremd sind.
    \item
      Zeige Sie für $p$ prim und $\ell \geq 1$, dass $\varphi(p) = p^{\ell-1} (p-1) $ gilt.
    \item
      Bestimmen Sie $\varphi(42)$, $\varphi(57)$ und $\varphi(144)$.
  \end{enumerate}
\end{question}








\end{document}
