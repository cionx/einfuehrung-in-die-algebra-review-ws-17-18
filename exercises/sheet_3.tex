\documentclass[a4paper, 10pt]{scrartcl}
\usepackage{../styles/generalstyle}
\usepackage{../styles/exercisestyle}

\subject{Repetitorium Einführung in die Algebra, WS 2017/18}
\title{Tag 3}
\author{}
\date{}

\begin{document}

\begin{question}
  Zeigen Sie, dass die folgenden Polynome irreduzibel sind:
  \begin{enumerate}
    \item
      $3t^6 + 4t^4 - 6t^2 - 10 \in \Integer[t]$.
  \end{enumerate}
\end{question}

\begin{question}
  Es sei $R$ ein kommutativer Ring und $P \ideal R$ ein Primideal.
  Zeigen Sie, dass
  \[
              S
    \defined  R \smallsetminus P
    =         \{ r \in R \suchthat r \notin P \}
  \]
  ein multiplikative Teilmenge von $R$ ist.
\end{question}

\begin{question}
  Es seien $n, m \geq 1$.
  Zeigen Sie, dass
  \[
          \Integer/n \times \Integer/m
    \cong \Integer/{\ggT(n,m)} \times \Integer/{\kgV(n,m)} \,.
  \]
\end{question}

\begin{question}
  Zeigen Sie, dass der Ring $\Integer[\sqrt{-2}] = \Integer[i\sqrt{2}]$ euklidisch ist.
\end{question}

\begin{question}
  Für alle $n \geq 1$ sei $\varphi(n) \defined \card{\unitgroup{(\Integer/n)}}$.,
  \begin{enumerate}
    \item
      Zeigen Sie, dass $\varphi(nm) = \varphi(n)\varphi(m)$ gilt, wenn $n, m \geq 1$ teilerfremd sind.
    \item
      Zeige Sie für $p$ prim und $\ell \geq 1$, dass $\varphi(p) = p^{\ell-1} (p-1) $ gilt.
    \item
      Bestimmen Sie $\varphi(42)$, $\varphi(57)$ und $\varphi(144)$.
  \end{enumerate}
\end{question}








\end{document}
