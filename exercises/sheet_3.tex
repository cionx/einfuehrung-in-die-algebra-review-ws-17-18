\documentclass[a4paper, 10pt]{scrartcl}
\usepackage{../styles/generalstyle}
\usepackage{../styles/exercisestyle}

\subject{Repetitorium Einführung in die Algebra, WS 2017/18}
\title{Tag 3}
\author{}
\date{}

\begin{document}

\begin{question}
  Zeigen Sie, dass die folgenden Polynome irreduzibel sind:
  \begin{enumerate}
    \item
      % Eisenstein
      $3t^6 + 4t^4 - 6t^2 - 10 \in \Integer[t]$
    \item
      % mod 2
      $t^3 + 39t^2 - 4t + 8 \in \Rational[t]$
    \item
      %  divide by 2, then mod 2
      $2t^3 - 14t + 6 \in \Rational[t]$
    \item
      % t -> t+1
      $t^4 + 1 \in \Rational[t]$
    \item
      % Eisenstein
      $t^5 - u \in \Rational(u)[t]$
    \item
      % Eisenstein
      $t^2 u + tu^2 - t - u + 1 \in \Rational[t,u]$
  \end{enumerate}
\end{question}

\begin{question}
  Bestimmen Sie jeweils alle Lösungen $x \in \Integer$ der folgenden Systeme simultaner Kongruenzen:
  \begin{enumerate}
    \item
      $
        \left\{
          \begin{array}{ccrl}
            x &\equiv& 3 & \pmod{5}  \\
            x &\equiv& 2 & \pmod{7}
          \end{array}
        \right.
      $
    \item
      $
        \left\{
          \begin{array}{ccrl}
            5x &\equiv& 6 & \pmod{12}  \\
            3x &\equiv& 7 & \pmod{11}
          \end{array}
        \right.
      $
    \item
      $
        \left\{
          \begin{array}{ccrl}
            x &\equiv& 10 & \pmod{6}  \\
            x &\equiv& -6 & \pmod{14}
          \end{array}
        \right.
      $
    \item
      $
        \left\{
          \begin{array}{ccrl}
            x &\equiv&  2 & \pmod{6}  \\
            x &\equiv& -2 & \pmod{10} \\
            x &\equiv&  1 & \pmod{7}
          \end{array}
        \right.
      $
  \end{enumerate}
\end{question}

\begin{question}
  Bestimmen Sie mithilfe des euklidischen Algorithmus jeweils den größten gemeinsamen Teiler der folgenden Zahlen, und drücken Sie diesen als $\Integer$-Linearkombination dieser Zahlen aus.
  \begin{enumerate}
    \item
      $270$, $192$
    \item
      $30$, $42$, $70$
  \end{enumerate}
\end{question}

\begin{solution}
  \begin{enumerate}
    \item
      Es gilt
      \begin{align*}
         &\,  \ggT(270, 192)
        =     \ggT(192 + 78, 192)
        \\
        =&\,  \ggT(192, 78)
        =     \ggT(2 \cdot 78 + 36, 78)
        \\
        =&\,  \ggT(78, 36)
        =     \ggT(2 \cdot 36 + 6, 36)
        \\
        =&\,  \ggT(36, 6)
        =     6 \,.
      \end{align*}
      Dabei gilt
      \begin{align*}
            6
        &=  78 - 2 \cdot 36
        =  78 - 2 \cdot (192 - 2 \cdot 78)
        \\
        &=  5 \cdot 78 - 2 \cdot 192
        =  5 \cdot (270 - 192) - 2 \cdot 192
        \\
        &=  5 \cdot 270 - 7 \cdot 192 \,.
      \end{align*}
      
    \item
      Es gilt $\ggT(30,42,70) = \ggT(\ggT(30, 42), 70)$, weshalb wir wiederholt den euklidischen Algorithmus anwenden können.
      \begin{enumerate}
        \item
          Es gilt
          \begin{align*}
             &\,  \ggT(42, 30)
            =     \ggT(30 + 12, 30)
          \\
            =&\,  \ggT(30, 12)
            =     \ggT(2 \cdot 12 + 6, 12)
          \\
            =&\,  \ggT(12,6)
            =     6 \,.
          \end{align*}
          Dabei gilt
          \begin{equation}
            \label{equation: first gcd}
              6
            = 30 - 2 \cdot 12
            = 30 - 2 \cdot (42 - 30)
            = 3 \cdot 30 - 2 \cdot 42 \,.
          \end{equation}
      \end{enumerate}
    \item
      Es gilt nun
      \begin{enumerate}
        \item
          Es gilt
          \begin{align*}
             &\,  \ggT(70, 6)
            =     \ggT(11 \cdot 6 + 4, 6)
          \\
            =&\,  \ggT(6, 4)
            =     \ggT(4 + 2, 4)
          \\
            =&\,  \ggT(4, 2)
            =     2 \,.
          \end{align*}
          Dabei gilt
          \[
              2
            = 4 - 2
            = 4 - (6 - 4)
            = 2 \cdot 4 - 6
            = 2 \cdot (70 - 11 \cdot 6) - 6
            = 2 \cdot 70 -  23 \cdot 6 \,.
          \]
          Durch Einsetzen von Gleichung~\eqref{equation: first gcd} ergibt sich damit, dass
          \[
              2
            = 2 \cdot 70 - 23 \cdot (3 \cdot 30 - 2 \cdot 42)
            = 2 \cdot 70 + 46 \cdot 42 - 69 \cdot 30 \,.
          \]

      \end{enumerate}
  \end{enumerate}
\end{solution}

\begin{question}
  Es sei $R$ ein kommutativer Ring und $P \ideal R$ ein Primideal.
  Zeigen Sie, dass
  \[
              S
    \defined  R \smallsetminus P
    =         \{ r \in R \suchthat r \notin P \}
  \]
  ein multiplikative Teilmenge von $R$ ist.
\end{question}

\begin{solution}
  Es gilt
  \[
          1 \in S
    \iff  1 \notin P
    \iff  P \neq R \,,
  \]
  wobei $P \neq R$ gilt, da $P$ prim ist.
  Es gilt außerdem, dass
  \[
    \forall x, y \in R:
    (xy \in P \implies x \in P \vee y \in P) \,,
  \]
  was äquivalent zu
  \[
    \forall x, y \in R:
    ( x \notin P \wedge y \notin P \implies xy \notin P)
  \]
  ist, was sich wiederum zu
  \[
    \forall x, y \in R:
    ( x \in S \wedge y \in S \implies xy \in S )
  \]
  umschreiben lässt.
  Für alle $s, t \in S$ gilt also auch $st \in S$.
\end{solution}

\begin{question}
  Zeigen Sie, dass der Ring $\Integer[\sqrt{-2}] = \Integer[i\sqrt{2}]$ euklidisch ist.
\end{question}

\begin{solution}
  Für jedes $z = a + i b \sqrt{-2} \in \Integer[\sqrt{-2}]$ mit $a, b \in \Integer$ definieren wir die \emph{Norm} von $z$ als $N(z) \defined \abs{z}^2 = a^2 + 2 b^2 \in \Natural$.
  Dann ist $\Integer[\sqrt{-2}]$ zusammen mit $N$ ein euklidischer Ring:
  
  Man bemerke, dass $\Integer[\sqrt{-2}]$ in der komplexen Zahlenebene $\Complex$ ein \enquote{Gitter} von Breite $1$ und Höhe $\sqrt{2}$ bildet, weshalb es für jedes Element $z \in \Complex$ ein $w \in \Integer[\sqrt{-2}]$ mit
  \[
          \abs{z - w}
    \leq  \frac{\sqrt{1^2 + \sqrt{2}^2}}{2}
    =     \frac{\sqrt{3}}{2} < 1
  \]
  gibt.
  Für $f, g \in \Integer[\sqrt{-2}]$ mit $g \neq 0$ lässt sich in $\Complex$ der Quotient $f/g$ bilden, und es gibt ein $q \in \Integer[\sqrt{-2}]$ mit $\abs{f/g - w} < 1$ gibt.
  Für $r \defined f - qg$ gilt dann $f = qg + r$, wobei
  \[
      N(r)
    = \abs{r}^2
    = \abs{f - qg}^2
    = \underbrace{\abs{f/g - q}^2}_{<1} \abs{g}^2
    < \abs{g}^2
    = N(g) \,.
  \]
\end{solution}

\begin{question}
  Es seien $n, m \geq 1$.
  Zeigen Sie, dass
  \[
          \Integer/n \times \Integer/m
    \cong \Integer/{\ggT(n,m)} \times \Integer/{\kgV(n,m)} \,.
  \]
\end{question}

\begin{solution}
  Es sei $\mathcal{P} \subseteq \Natural$ die übliche Menge der Primzahlen.
  Aus den Primfaktorzerlegungen
  \begin{align*}
      n
    = \prod_{p \in \mathcal{P}} p^{\nu_p}
    \qquad&\text{und}\qquad
      m
    = \prod_{p \in \mathcal{P}} p^{\mu_p} \,.
  \intertext{ergeben sich die Primfaktorzerlegungen}
      \ggT(n,m)
    = \prod_{p \in \mathcal{P}} p^{\min(\nu_p,\mu_p)}
    \qquad&\text{und}\qquad
      \kgV(n,m)
    = \prod_{p \in \mathcal{P}} p^{\max(\nu_p,\mu_p)} \,.
  \end{align*}
  Nach dem chinesischen Restklassensatze gelten deshalb
  \begin{align*}
            \Integer/n
    &\cong  \prod_{p \in \mathcal{P}} \Integer/p^{\nu_p} \,,  \\
            \Integer/m
    &\cong  \prod_{p \in \mathcal{P}} \Integer/p^{\mu_p} \,,  \\
            \Integer/\ggT(n,m)
    &\cong  \prod_{p \in \mathcal{P}} \Integer/p^{\min(\nu_p,\mu_p)} \,,  \\
            \Integer/\kgV(n,m)
    &\cong  \prod_{p \in \mathcal{P}} \Integer/p^{\max(\nu_p,\mu_p)} \,,
  \shortintertext{und somit}
            \Integer/n \times \Integer/m
    &\cong  \prod_{p \in \mathcal{P}} ( \Integer/p^{\nu_p} \times \Integer/p^{\mu_p}) \,, \\
            \Integer/\ggT(n,m) \times \Integer/\kgV(n,m)
    &\cong  \prod_{p \in \mathcal{P}} \left(
                                                \Integer/p^{\min(\nu_p,\mu_p)}
                                        \times  \Integer/p^{\max(\nu_p,\mu_p)}
                                      \right) \,.
  \intertext{Dabei gilt für jedes $p \in \mathcal{P}$, dass}
            \Integer/p^{\nu_p} \times \Integer/p^{\mu_p}
    &\cong  \Integer/p^{\min(\nu_p,\mu_p)} \times  \Integer/p^{\max(\nu_p,\mu_p)} \,,
  \intertext{denn es gilt}
            \{\nu_p, \mu_p\}
    &=      \{ \min(\nu_p, \mu_p), \max(\nu_p, \mu_p) \} \,.
  \end{align*}
\end{solution}

\begin{question}
  Für alle $n \geq 1$ sei $\varphi(n) = \card{\{1 \leq k \leq n \suchthat \text{$k$ und $n$ sind teilerfremd}\}}$.
  \begin{enumerate}
    \item
      Begründen Sie, dass $\varphi(n) = \card{\unitgroup{(\Integer/n)}}$ gilt.
    \item
      Zeigen Sie, dass $\varphi(nm) = \varphi(n)\varphi(m)$ gilt, wenn $n, m \geq 1$ teilerfremd sind.
    \item
      Zeige Sie für $p$ prim und $\ell \geq 1$, dass $\varphi(p^\ell) = p^{\ell-1} (p-1) $ gilt.
    \item
      Bestimmen Sie $\varphi(42)$, $\varphi(57)$ und $\varphi(144)$.
  \end{enumerate}
\end{question}

\begin{solution}
  \begin{enumerate}
    \item
      Es ist $\class{k} \in \Integer/n$ genau dann eine Einheit, wenn $k$ und $n$ teilerfremd sind.
      Dies ergibt sich etwa dadurch, dass
      \begin{align*}
            &\, \class{k} \in \unitgroup{(\Integer/n)}  \\
        \iff&\, \exists \class{a} \in \Integer/n : \class{a} \cdot \class{k} = \class{1}  \\
        \iff&\, \exists \class{a} \in \Integer/n : \class{ak} = \class{1}  \\
        \iff&\, \exists a, b \in \Integer : ak + bn = 1 \\
        \iff&\, 1 \in \genideal{k,n} = \genideal{\ggT(k,n)} \\
        \iff&\, \text{$\ggT(k,n)$ ist eine Einheit} \,.
      \end{align*}
      Es gilt somit
      \begin{align*}
            \card{\unitgroup{(\Integer/n)}}
        &=  \card{\{1 \leq k \leq n \suchthat \text{$\class{k}$ ist eine Einheit in $\Integer/n$}\}}  \\
        &=  \card{\{1 \leq k \leq n \suchthat \text{$k$ und $n$ sind teilerfremd}\}}
         =  \varphi(n) \,.
      \end{align*}
    \item
      Nach dem chinesischen Restklassensatz gilt $\Integer/(nm) = \Integer/n \times \Integer/m$, und somit
      \begin{align*}
            \varphi(nm)
        &=  \card*{\unitgroup{(\Integer/(nm))}}
         =  \card*{\unitgroup{(\Integer/n \times \Integer/m)}} \\
        &=  \card*{\unitgroup{(\Integer/n)} \times \unitgroup{(\Integer/m)}}
         =  \card*{\unitgroup{(\Integer/n)}} \card*{\unitgroup{(\Integer/m)}}
         =  \varphi(n) \varphi(m) \,.
      \end{align*}
    \item
      Es ist $1 \leq k \leq p^\ell$ genau dann teilerfremd zu $p^\ell$, wenn $k$ den Primfaktor $p$ nicht enthält.
      Da jede $p$-te Zahl durch $p$ teilbar ist, gilt somit
      \[
          \varphi(p^\ell)
        = p^\ell - \frac{p^\ell}{p}
        = p^\ell - p^{\ell-1}
        = p^{\ell-1}(p-1) \,.
      \]
    \item
      Es gelten
      \begin{align*}
            \varphi(42)
        &=  \varphi(2 \cdot 3 \cdot 7)
         =  \varphi(2) \varphi(3) \varphi(7)
         =  1 \cdot 2 \cdot 6
         =  12 \,,
        \\
            \varphi(57)
        &=  \varphi(3 \cdot 19)
         =  \varphi(3) \varphi(19)
         =  2 \cdot 18
         =  36 \,,
        \\
            \varphi(144)
        &=  \varphi(2^4 \cdot 3^2)
        =   \varphi(2^4) \varphi(3^2)
        =   2^3 \cdot 1 \cdot 3^1 \cdot 2
        =   48 \,.
      \end{align*}

  \end{enumerate}
\end{solution}





\pagebreak





\section*{Lösungen}

\printsolutions





\end{document}
