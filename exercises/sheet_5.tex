\documentclass[a4paper, 10pt]{scrartcl}
\usepackage{../styles/generalstyle}
\usepackage{../styles/exercisestyle}

\subject{Repetitorium Einführung in die Algebra, WS 2017/18}
\title{Tag 5}
\author{}
\date{}

\begin{document}

\begin{question}
  Es sei $K(a)/K$ eine einfache algebraische Körpererweiterung.
  \begin{enumerate}
    \item
      Zeigen Sie, dass $\sepindex{K(a)}{K} \leq \fieldindex{K(a)}{K}$ gilt.
    \item
      Zeigen Sie, dass genau dann Gleichheit gilt, wenn $a$ separabel ist.
  \end{enumerate}
\end{question}

\begin{question}
  Zeigen Sie, dass der Frobenius-Automorphismus $\sigma \colon \closure{\Finite_p} \to \closure{\Finite_p}$ unendliche Ordnung hat.
\end{question}

\begin{question}
  Zeigen Sie mithilfe der Galoiskorrespondenz, dass es genau dann eine Einbettung $\Finite_{p^n} \hookrightarrow \Finite_{p^m}$ gibt, wenn $n \divides m$ gilt.
\end{question}

\begin{question}
  Es sei $\zeta_3 \defined e^{2 \pi i /3}$.
  Auf dem gestrigen Aufgabenzettel wurde bereits gezeigt, dass $\fieldindex{\Rational(\sqrt[3]{2}, \zeta_3)}{\Rational} = 6$ gilt;
  dies darf im Folgenden ohne erneuten Beweis genutzt werden.
  \begin{enumerate}
    \item
      Zeigen Sie, dass die Erweiterung $\Rational(\sqrt[3]{2}, \zeta_3)/\Rational$ galoissch ist.
    \item
      Folgern Sie, dass $\Gal{\Rational(\sqrt[3]{2}, \zeta_3)/\Rational} \cong S_3$.
      \\
      (\emph{Hinweis}:
       Konstruieren Sie zunächst eine Einbettung $\Gal{\Rational(\sqrt[3]{2}, \zeta_3)/\Rational} \hookrightarrow S_3$.)
  \end{enumerate}
\end{question}

\begin{question}
  Es sei $p$ prim und $\zeta_p \defined e^{2 \pi i / p}$.
  \begin{enumerate}
    \item
      Bestimmen Sie den Grad $\fieldindex{\Rational(\sqrt[p]{p}, \zeta_p)}{\Rational}$.
    \item
      Zeigen Sie, dass die Erweiterung $\Rational(\sqrt[p]{p}, \zeta_p)/\Rational$ galoissch ist.
  \end{enumerate}
\end{question}

\begin{solution}
  \begin{enumerate}
    \item
      Es handelt sich bei $\Rational(\sqrt[p]{p}, \zeta_p)$ um das Kompositum von $\Rational(\sqrt[p]{p})$ und $\Rational(\zeta_p)$.
      \begin{itemize}
        \item
          Das Minimalpolynom von $\sqrt[p]{p}$ über $\Rational$ ist $t^p - p \in \Rational[t]$, wobei sich die Irreduziblität aus dem Eisenstein-Kriterium ergibt.
          Es gilt deshalb $\fieldindex{\Rational(\sqrt[p]{p})}{\Rational} = p$.
        \item
          Es ist $\zeta_p$ eine primitive $p$-te Einheitswurzel.
          Das Minimalpolynom von $\zeta_p$ ist deshalb das $p$-te Kreisteilungspolynom $\Phi_p = t^{p-1} + \dotsb + t + 1 \in \Rational[t]$.
          Es gilt deshalb $\fieldindex{\Rational(\zeta_p)}{\Rational} = p-1$.
      \end{itemize}
      Da $p$ und $p-1$ teilerfremd sind, gilt somit
      \[
          \fieldindex{\Rational(\sqrt[p]{p}, \zeta_p)}{\Rational}
        = \fieldindex{\Rational(\sqrt[p]{p}) \Rational(\zeta_p)}{\Rational}
        = \fieldindex{\Rational(\sqrt[p]{p})}{\Rational} \fieldindex{\Rational(\zeta_p)}{\Rational}
        = p(p-1) \,.
      \]
    \item
      Es gilt
      \[
          \Rational(\sqrt[p]{p}, \zeta_p)
        = \Rational(\sqrt[p]{p}, \zeta_p \sqrt[p]{p}, \dotsc, \zeta_p^{p-1} \sqrt[p]{p}) \,,
      \]
      wobei $\sqrt[p]{p}, \zeta_p \sqrt[p]{p}, \dotsc, \zeta_p^{p-1} \sqrt[p]{p}$ genau die $p$-ten Einheitswurzeln sind.
      Es ist deshalb $\Rational(\sqrt[p]{p}, \zeta_p)$ ein Zerfällungskörper des Poylnoms $t^p - p \in \Rational[t]$, und die Erweiterung $\Rational(\sqrt[p]{p}, \zeta_p)/\Rational$ ist somit normal.
      Die Erweiterung ist separabel, da $\Rational$ perfekt ist, da $\ringchar{\Rational} = 0$ gilt.
  \end{enumerate}
\end{solution}

\begin{question}
  Es sei $L \defined \Rational(\sqrt{2}, \sqrt{3}, \sqrt{5})$.
  \begin{enumerate}
    \item
      Zeigen Sie, dass die Erweiterung $\Rational(\sqrt{2}, \sqrt{3}, \sqrt{5})/\Rational$ galoissch ist.
  \end{enumerate}
  Sie dürfen im Folgenden ohne Beweis verwenden, dass $\sqrt{5} \notin \Rational(\sqrt{2}, \sqrt{3})$ liegt.
  \begin{enumerate}[resume]
    \item
      Zeigen Sie, dass $\fieldindex{L}{\Rational} = 8$ gilt.
    \item
      Konstruieren Sie einen Gruppenisomorphismus $\Gal{L/\Rational} \to (\Integer/2)^3$.
    \item
      Betrachten Sie die folgenden Unterkörper von $L$:
      \begin{enumerate}
        \item
          $\Rational(\sqrt{2})$
        \item
          $\Rational(\sqrt{6})$
        \item
          $\Rational(\sqrt{3}, \sqrt{5})$
      \end{enumerate}
      Bestimmen Sie die Untergruppen von $\Gal{L/\Rational}$, denen diese Unterkörper unter der Galoiskorrespondenz entsprechen, sowie die Ordnungen dieser Untergruppen.
  \end{enumerate}
\end{question}





\pagebreak





\section*{Lösungen}

\printsolutions





\end{document}
