\documentclass[a4paper, 10pt]{scrartcl}
\usepackage{../styles/generalstyle}
\usepackage{../styles/exercisestyle}

\subject{Repetitorium Einführung in die Algebra, WS 2017/18}
\title{Tag 4}
\author{}
\date{}

\begin{document}

\begin{question}
  \begin{enumerate}
    \item
      Zeigen Sie, dass für alle $n, m \geq 1$ der Körper $\Rational(\!\sqrt[n]{2})$ genau dann ein Unterkörper von $\Rational(\!\sqrt[m]{2})$ ist, wenn $n \divides m$ gilt.
    \item 
      Es sei $p$ prim und $L/K$ eine Erweiterung vom Grad $\fieldindex{L}{K} = p$.
      Zeigen Sie, dass die Erweiterung $L/K$ einfach ist, und bestimmen Sie alle $a \in L$ mit $L = K(a)$.
    \item
      Es sei $K(a)/K$ eine einfache Erweiterung mit ungeraden Grad $\fieldindex{K(a)}{K}$.
      Zeigen Sie, dass $K(a) = K(a^2)$ gilt.
    \item
      Es sei $L/K$ eine Körpererweiterung vom Grad $\fieldindex{L}{K} = 2^n$ und $f \in K[t]$ ein kubisches Polynom, das eine Nullstelle in $L$ hat.
      Zeigen Sie, dass $f$ bereits eine Nullstelle in $K$ hat.
  \end{enumerate}


\end{question}

\begin{question}
  Zeigen Sie, dass es für alle $n \geq 1$ ein Element $a \in \closure{\Rational}$ mit $\fieldindex{\Rational(a)}{\Rational} = n$ gibt.
  Folgern Sie, dass $\fieldindex{\closure{\Rational}}{\Rational} = \infty$ gilt.
\end{question}

\begin{question}
  \begin{enumerate}
    \item
      Bestimmen Sie das Minimalpolynom von $\sqrt[3]{2}$ über $\Rational$.
    \item
      Bestimmen Sie das Minimalpolynom von  $\zeta_3 \defined e^{2 \pi i/3}$ über $\Rational$.
    \item
      Bestimmen Sie den Grad der Erweiterung $\Rational(\sqrt[3]{2}, \zeta_3)/\Rational$.
  \end{enumerate}
\end{question}

\begin{question}
  Es sei $a \defined \sqrt[4]{2}$.
  \begin{enumerate}
    \item
      Zeigen Sie, dass $\fieldindex{\Rational(a)}{\Rational} = 4$ gilt.
    \item
      Zeigen Sie, dass es genau zwei Körperisomorphismen $\Rational(a) \to \Rational(a)$ gibt.
    \item
      Entscheiden Sie, ob die Erweiterung $\Rational(a)/\Rational$ normal ist.
  \end{enumerate}
\end{question}

% \begin{question}
%   Zeigen Sie, dass algebraisch abgeschlossene Körper unendlich sind.
% \end{question}

\begin{question}
  Es sei $L/K$ eine algebraische Körpererweiterung, so dass jedes Polynom $f \in K[t]$ über $L$ in Linearfaktoren zerfällt.
  Zeigen Sie, dass $L$ bereits ein algebraischer Abschluss von $K$ ist.
\end{question}

\begin{solution}
  Da die Körpererweiterung $L/K$ nach Annahme algebraisch ist, genügt es zu zeigen, dass $L$ algebraisch abgeschlossen ist.
  Hierfür zeigen wir, dass für jede algebraische Körpererweiterung $L'/L$ bereits $L' = L$ gilt:
  
  Es sei $a \in L'$.
  Dann sind $L(a)/L$ und $L/K$ algebraisch, und somit ist auch $L(a)/K$ algebraisch.
  Nach Annahme zerfällt das Minimalpolynom $m_a \in K[t]$ bereits über $L$ in Linearfaktoren;
  insbesondere muss deshalb die Nullstelle $a$ von $m_a$ bereits in $L$ liegen.
\end{solution}

\begin{question}
  \begin{enumerate}
    \item
      Es sei $L/K$ eine Körpererweiterung vom Grad $\fieldindex{L}{K} = 2$.
      Zeigen Sie, dass $L$ der Zerfällungskörper eines quadratischen Polynoms $f \in K[t]$ ist, und die Erweiterung $L/K$ somit normal ist.
    \item
      Folgern Sie, dass die Erweiterungen $\Rational(\sqrt[4]{3})/\Rational(\sqrt{3})$ und $\Rational(\sqrt{3})/\Rational$ normal sind.
    \item
      Zeigen Sie, dass die Erweiterung $\Rational(\sqrt[4]{3})/\Rational$ allerdings nicht normal ist.
  \end{enumerate}
\end{question}

\begin{solution}
  \begin{enumerate}
    \item
      Es sei $a \in L$ mit $a \notin K$.
      Dann gilt
      \[
          2
        = \fieldindex{L}{K}
        = \fieldindex{L}{K(a)} \fieldindex{K(a)}{K}
      \]
      mit $\fieldindex{K(a)}{K} > 1$;
      es gilt deshalb $\fieldindex{L}{K(a)} = 1$ und $\fieldindex{K(a)}{K} = 2$, und somit insbesondere $L = K(a)$.
      (Wir haben hier den Beweis von Aufgabe 1, 2. wiederholt.)
      Das Minimalpolynom $m_a \in K[t]$ ist quadratisch, da
      \[
          \deg(m_a)
        = \fieldindex{K(a)}{K}
        = \fieldindex{L}{K}
        = 2
      \]
      gilt.
      Da das quadratische Polynom $m_a$ eine Nullstelle in $L$ hat, zerfällt $m_a$ über $L$ bereits in Linearfaktoren.
      Zusammen mit $L = K(a)$ folgt, dass $L$ ein Zerfällungskörper von $m_a$ ist.
    \item
      Für alle $n \geq 1$ ist $t^n - 3 \in \Rational[t]$ das Minimalpolynom von $\sqrt[n]{3}$ über $\Rational$, wobei sich die Irreduziblität aus dem Eisenstein-Kriteraum ergibt.
      Es gilt also stets $\fieldindex{\Rational(\sqrt[n]{3})}{\Rational} = n$.
      \begin{itemize}
        \item
          Es gilt somit $\fieldindex{\Rational(\sqrt{3})}{\Rational} = 2$.
        \item
          Es gilt $\fieldindex{\Rational(\sqrt[4]{3})}{\Rational} = 4$, und aus der Multiplikativität des Grades ergibt sich
          \[
              4
            = \fieldindex{\Rational(\sqrt[4]{3})}{\Rational}
            = \fieldindex{\Rational(\sqrt[4]{3})}{\Rational(\sqrt{3})}
              \fieldindex{\Rational(\sqrt{3})}{\Rational}
            = 2\fieldindex{\Rational(\sqrt[4]{3})}{\Rational(\sqrt{3})} \,,
          \]
          und somit $\fieldindex{\Rational(\sqrt[4]{3})}{\Rational(\sqrt{3})} = 2$.
      \end{itemize}
      Nach dem vorherigen Aufgabenteil sind $\fieldindex{\Rational(\sqrt[4]{3})}{\Rational(\sqrt{3})}$ und $\fieldindex{\Rational(\sqrt{3})}{\Rational}$ somit normal.
    \item
      Das irreduzible Polynom $f \defined t^4 - 3 \in \Rational[t]$ besitzt in $\Rational(\sqrt[4]{3})$ genau zwei Nullstellen, nämlich $\sqrt[4]{3}$ und $-\sqrt[4]{3}$.
      Also besitzt $f$ zwar eine Nullstelle in $\Rational(\sqrt[4]{3})$, zerfällt dort aber noch nicht.
      Deshalb ist $\Rational(\sqrt[4]{3})/\Rational$ nicht normal.
  \end{enumerate}
\end{solution}





\pagebreak





\section*{Lösungen}

\printsolutions





\end{document}
