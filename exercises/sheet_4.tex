\documentclass[a4paper, 10pt]{scrartcl}
\usepackage{../styles/generalstyle}
\usepackage{../styles/exercisestyle}

\subject{Repetitorium Einführung in die Algebra, WS 2017/18}
\title{Tag 4}
\author{}
\date{}

\begin{document}

\begin{question}
  \begin{enumerate}
    \item
      Zeigen Sie, dass für alle $n, m \geq 1$ der Körper $\Rational(\!\sqrt[n]{2})$ genau dann ein Unterkörper von $\Rational(\!\sqrt[m]{2})$ ist, wenn $n \divides m$ gilt.
    \item 
      Es sei $p$ prim und $L/K$ eine Erweiterung vom Grad $\fieldindex{L}{K} = p$.
      Zeigen Sie, dass die Erweiterung $L/K$ einfach ist, und bestimmen Sie alle $a \in L$ mit $L = K(a)$.
    \item
      Es sei $K(a)/K$ eine einfache Erweiterung mit ungeraden Grad $\fieldindex{K(a)}{K}$.
      Zeigen Sie, dass $K(a) = K(a^2)$ gilt.
    \item
      Es sei $L/K$ eine Körpererweiterung vom Grad $\fieldindex{L}{K} = 2^n$ und $f \in K[t]$ ein kubisches Polynom, das eine Nullstelle in $L$ hat.
      Zeigen Sie, dass $f$ bereits eine Nullstelle in $L$ hat.
  \end{enumerate}


\end{question}

\begin{question}
  Zeigen Sie, dass es für alle $n \geq 1$ ein Element $a \in \closure{\Rational}$ mit $\fieldindex{\Rational(a)}{\Rational} = n$ gibt.
  Folgern Sie, dass $\fieldindex{\closure{\Rational}}{\Rational} = \infty$ gilt.
\end{question}

\begin{question}
  \begin{enumerate}
    \item
      Bestimmen Sie das Minimalpolynom von $\sqrt[3]{2}$ über $\Rational$.
    \item
      Bestimmen Sie das Minimalpolynom von  $\zeta_3 \defined e^{2 \pi i/3}$ über $\Rational$.
    \item
      Bestimmen Sie den Grad der Erweiterung $\Rational(\sqrt[3]{2}, \zeta_3)/\Rational$.
  \end{enumerate}
\end{question}

\begin{question}
  Es sei $a \defined \sqrt[4]{2}$.
  \begin{enumerate}
    \item
      Zeigen Sie, dass $\fieldindex{\Rational(a)}{\Rational} = 4$ gilt.
    \item
      Zeigen Sie, dass es genau zwei Körperisomorphismen $\Rational(a) \to \Rational(a)$ gibt.
    \item
      Entscheiden Sie, ob die Erweiterung $\Rational(a)/\Rational$ normal ist.
  \end{enumerate}
\end{question}

% \begin{question}
%   Zeigen Sie, dass algebraisch abgeschlossene Körper unendlich sind.
% \end{question}

\begin{question}
  Es sei $L/K$ eine algebraische Körpererweiterung, so dass jedes Polynom $f \in K[t]$ eine Nullstelle in $L$ hat.
  Zeigen Sie, dass $L$ bereits ein algebraischer Abschluss von $K$ ist.
\end{question}

\begin{question}
  \begin{enumerate}
    \item
      Es sei $L/K$ eine Körpererweiterung vom Grad $\fieldindex{L}{K} = 2$.
      Zeigen Sie, dass $L$ der Zerfällungskörper eines quadratischen Polynoms $f \in K[t]$ ist, und die Erweiterung $L/K$ somit normal ist.
    \item
      Folgern Sie, dass die Erweiterungen $\Rational(\sqrt[4]{3})/\Rational(\sqrt{3})$ und $\Rational(\sqrt{3})/\Rational$ normal sind.
    \item
      Zeigen Sie, dass die Erweiterung $\Rational(\sqrt[4]{3})/\Rational$ allerdings nicht normal ist.
  \end{enumerate}
\end{question}


\end{document}
