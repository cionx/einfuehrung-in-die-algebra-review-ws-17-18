\documentclass[a4paper, 10pt]{scrartcl}
\usepackage{../styles/generalstyle}
\usepackage{../styles/exercisestyle}

\subject{Repetitorium Einführung in die Algebra, WS 2017/18}
\title{Tag 2}
\author{}
\date{}

\begin{document}

\begin{question}
  Bestimmen Sie bis auf Isomorphie alle abelschen Gruppen der Ordnung $8$.
\end{question}

\begin{question}
  Es sei $p$ prim und $G$ eine endliche Gruppe der Ordnung $\card{G} = p^r m$ mit $p \notdivides m$.
  Dabei gelte $r \geq 1$, $m \geq 2$, und die Gruppe $G$ sei zudem einfach, d.h.\ $G$ und $1$ seien die einzigen beiden Normalteiler von $G$.
  Zeigen Sie für die Anzahl der $p$-Sylowuntergruppen $n_p$, dass $n_p \geq p + 1$ gilt.
\end{question}

\begin{question}
  Es sei $R$ ein kommutativer Ring.
  \begin{enumerate}
    \item
      Es sei $I \ideal R$ ein Ideal.
      Zeigen Sie, dass
      \[
                  I[t]
        \defined  \left\{
                    \sum_i a_i t^i \in R[t]
                  \suchthat*
                    \text{$a_i \in I$ für alle $i$}
                  \right\}
      \]
      ein Ideal in $R[t]$ ist, so dass $R[t]/I[t] \cong (R/I)[t]$ gilt.
    \item
      Entscheiden Sie, ob $P[t] \ideal R[t]$ prim ist, wenn $P \ideal R$ prim ist.
    \item
      Entscheiden Sie, ob $M[t] \ideal R[t]$ maximal ist, wenn $M \ideal R$ maximal ist.
  \end{enumerate}
\end{question}


\end{document}
