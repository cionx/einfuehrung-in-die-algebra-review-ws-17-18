\documentclass[a4paper, 10pt]{scrartcl}
\usepackage{../styles/generalstyle}
\usepackage{../styles/exercisestyle}

\subject{Repetitorium Einführung in die Algebra, WS 2017/18}
\title{Tag 2}
\author{}
\date{}

\begin{document}

\begin{question}
  \begin{enumerate}
    \item
      Bestimmen Sie für jede Primzahl $p$ eine $p$-Sylowuntergruppe von $S_4$.
    \item
      Es sei $p$ prim.
      Geben Sie eine $p$-Sylowuntergruppe von $\upper{n}{\Finite_p}$, der Gruppe der invertierbaren oberen $(n \times n)$-Matrizen mit Einträgen in $\Finite_p$, an.
  \end{enumerate}
\end{question}

\begin{question}
  Zeigen Sie, dass jede Gruppe der Ordnung $231$ mindestens zwei echte, nicht-triviale Normalteiler enthält.
\end{question}

\begin{question}
  Es sei $p$ prim, $G$ eine endliche Gruppe und $n_p$ die Anzahl der $p$-Sylowuntergruppen von $G$.
  Es sei $\card{G} = p^r m$ mit $p \notdivides m$, wobei $r \geq 1$, $m \geq 2$ gelte, und die Gruppe $G$ sei zudem einfach, d.h.\ $G$ und $\{1\}$ seien die einzigen beiden Normalteiler von $G$.
  Zeigen Sie, dass $n_p \geq p + 1$ gilt.
\end{question}

% \begin{question}
%   Es sei $p$ prim, $G$ eine endliche Gruppe und $n_p$ die Anzahl der $p$-Sylowuntergruppen von $G$.
%   Es sei $S$ eine solche $p$-Sylowuntergruppe.
%   Zeigen Sie für den Normalisator $\normalizer{G}{S}$, dass $n_p = \groupindex{G}{\normalizer{G}{S}}$ gilt.
%   \newline
%   (\emph{Hinweis}:
%   $G$ wirkt auf der Menge der $p$-Sylowuntergruppen durch Konjugation.)
% \end{question}

\begin{question}
  Bestimmen Sie alle abelschen Gruppen der Ordnung $8$ bis auf Isomorphie.
\end{question}

\begin{question}
  Zeigen Sie, dass jeder endliche Integritätsbereich ein Körper ist.
  \newline
  (\emph{Hinweis}:
  Betrachten Sie für $r \in R$ die Abbildung $R \to R$, $x \mapsto rx$.)
\end{question}

\begin{question}
  Es sei $R$ ein kommutativer Ring.
  \begin{enumerate}
    \item
      Es sei $I \ideal R$ ein Ideal.
      Zeigen Sie, dass
      \[
                  I[t]
        \defined  \left\{
                    \sum_{i=0}^n a_i t^i \in R[t]
                  \suchthat*
                    \text{$a_i \in I$ für alle $i$}
                  \right\}
      \]
      ein Ideal in $R[t]$ ist, so dass $R[t]/I[t] \cong (R/I)[t]$ gilt.
    \item
      Entscheiden Sie, ob $P[t] \ideal R[t]$ prim ist, wenn $P \ideal R$ prim ist.
    \item
      Entscheiden Sie, ob $M[t] \ideal R[t]$ maximal ist, wenn $M \ideal R$ maximal ist.
  \end{enumerate}
\end{question}

\begin{question}
  Es sei $R$ ein kommutativer Ring und es gebe ein $n > 1$, so dass $x^n = x$ für alle $x \in R$ gilt.
  Zeigen Sie, dass jedes Primideal $P \ideal R$ bereits maximal ist.
\end{question}

\begin{solution}
  Der Quotientenring $R/P$ ist ein Integritätsbereich, da $P$ prim ist.
  Die Maximalität von $M$ ist äquivalent dazu, dass $R/P$ bereits ein Körper ist.
  Hierfür ist noch zu zeigen, dass jedes $x \in R/P$ mit $x \neq 0$ eine Einheit ist.
  Dabei gibt es nach Annahme ein $n > 1$ mit $x^n = x$.
  Da $x \neq 0$ gilt, und $R/P$ ein Integritätsbereich ist, lässt sich diese Gleichung durch $x$ kürzen;
  es gilt nämlich
  \[
          x^n = x
    \iff  x^n - x = 0
    \iff  x(x^{n-1} - 1) = 0
    \iff  x^{n-1} - 1 = 0
    \iff  x^{n-1} = 1 \,.
  \]
  Da $n > 1$ gilt, lässt sich $x^{n-1} = 1$ zu $x \cdot x^{n-2} = 1$ umschreiben.
  Es ist also $x$ eine Einheit mit $1/x = x^{n-2}$.
\end{solution}





\pagebreak





\section*{Lösungen}

\printsolutions





\end{document}
